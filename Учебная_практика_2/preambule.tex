% !TeX spellcheck = ru_RU
%Настройка страниц

\usepackage[left=30mm, right=15mm, top=20mm, bottom=20mm, headsep=1cm, footskip=1cm]{geometry}
%twoside, openany,%

%Настройка языка и отображения
\usepackage{cmap}				%Ссылки в PDF
\usepackage[T2A]{fontenc}		%Шрифты
\usepackage[utf8]{inputenc}		%Кодировка
%\usepackage{lmodern}			%Хз, не трогать
\usepackage[russian]{babel}		%Грамматика
\usepackage{graphicx,svg}		%Пикчи
\DeclareGraphicsExtensions{.pdf,.png,.jpg}
\frenchspacing					%Отключает большой пробел между предложениями

\usepackage[nodisplayskipstretch]{setspace}
\usepackage{indentfirst}		%Красная строка
\setlength{\parindent}{1.25cm}	%Настройка отступа красной строки для шрифта
%\linespread{1.3} 				%Межстрочный интервал
\usepackage{setspace}
\onehalfspacing

\usepackage{microtype}          %Пиздатый графон

%\usepackage{parskip} 			%Интервал между абзацами
%\setlength{\parindent}{1.5} 	%Настройка интервала между абзацами, по умолчанию будет 0

\usepackage{enumitem}           %Фикс списков
\setlist{noitemsep}             %Убираем расстояние между элементами списка

%\pagestyle{empty} 				%Использование стандартных колонитулов

\usepackage{fancyhdr}			%Колонтитулы
%\pagestyle{fancy} 				%Использование кастомных колонитулов
%\fancyhf{} 					%Отчистить все колонтитулы
%\lhead{} 						% левый верхний колонтитул
%\chead{} 						% центральный верхний
%\rhead{} 						% правый верхний
%\lfoot{} 						% левый нижний
%\cfoot{\thepage} 				% центральный нижний
%\rfoot{} 						% правый нижний


%Оглавление
\setcounter{tocdepth}{3}        %Уровни оглавления
%4 это chapter, section, subsection, subsubsection и paragraph;
%3 это chapter, section, subsection и subsubsection;
%2 это chapter, section, и subsection;
%1 это chapter и section;
%0 это chapter.

%Фикс таблиц и рисунков
\usepackage[tableposition=top]{caption}
\usepackage{subcaption}
\DeclareCaptionLabelFormat{gostfigure}{Рисунок #2}
\DeclareCaptionLabelFormat{gosttable}{Таблица #2}
\DeclareCaptionLabelSeparator{gost}{~---~}
\captionsetup{labelsep=gost}
\captionsetup[figure]{labelformat=gostfigure}
\captionsetup[table]{labelformat=gosttable}
\renewcommand{\thesubfigure}{\asbuk{subfigure}}
%По стандарту название рисунка располагается под рисунком, а название таблицы — над таблицей.
%Следить за этими расположениями вам придется самостоятельно, но пакету можно подсказать об этом
%законе, чтобы он оптимизировал выделение пустого места соответствующим образом.


\newcommand{\ii}{\textit}
%\newcommand{\bb}{\textbf}
\newcommand{\uu}{\underline}
%\newcommand{\m}{\texttt}
%\newcommand{\box}{\fbox}


%Переопределение заголовков
\usepackage{titlesec}

\titleformat{\chapter}
    {\filcenter\bfseries}
    {\thechapter}
    {1em}{}
 
\titleformat{\section}
    {\normalsize\bfseries}
    {\thesection}
    {1em}{}
 
\titleformat{\subsection}
    {\normalsize\bfseries}
    {\thesubsection}
    {1em}{}

% Настройка вертикальных и горизонтальных отступов
\titlespacing*{\chapter}{0pt}{-30pt}{8pt}
\titlespacing*{\section}{\parindent}{*1.3}{*1.3}
\titlespacing*{\subsection}{\parindent}{*1.3}{*1.3}


%\usepackage{blindtext}          %Затычка для тестов

\usepackage{tocloft} %регулировка расположения TableOfContent (Оглавления) на странице
\newcommand{\renametibleofcontent}{\renewcommand\contentsname{СОДЕРЖАНИЕ}}
\renewcommand{\cfttoctitlefont}{\hspace{0.38\textwidth}\MakeUppercase} %уменьшаем размер шрифта и ровняем по центру
%
%\let\LaTeXStandardTableOfContents\tableofcontents
%\renewcommand{\tableofcontents}{%
%\begingroup%
%\renewcommand{\bfseries}{\relax}%
%\LaTeXStandardTableOfContents%
%\endgroup%
%}%%Отключаем жирный шрифт в оглавлении

 % Межстрочные отступы в Оглавлении:
\setlength{\cftbeforetoctitleskip}{-10mm} %отступ Оглавления от верхнего поля страницы.
\setlength{\cftbeforechapskip}{\medskipamount} %отступ между главами
\setlength{\cftbeforesecskip}{\medskipamount} %отступ между секциями \section{title}

% % Отступы от левого поля:
\setlength{\cftchapindent}{0em} %отступ между левым полем и \chapter{}
\setlength{\cftsecindent}{1.25cm} %отступ между левым полем и \section{title}
\setlength{\cftsubsecindent}{2.5cm} %отступ между левым полем и \subsection{title}
% % Отточия в Оглавлении
\renewcommand\cftchapdotsep{\cftdot} %добавляет отточия после \chapter{title}
%\renewcommand{\cftchapleader}{\cftdotfill{\cftchapdotsep}} %делает отточия после \chapter{title} тонкими, (по умолчанию жирные).
\renewcommand\cftsecdotsep{\cftdot} %делает отточия после \section{title} частыми.
\renewcommand\cftsubsecdotsep{\cftdot} %делает отточия после \subsection{title} частыми.
\cftsetpnumwidth{1em} %Отступ точек от номера страницы
\makeatletter \renewcommand{\@dotsep}{1} \makeatother

\newcommand{\renamebiblio}{\renewcommand{\bibname}{СПИСОК ИСПОЛЬЗОВАННЫХ ИСТОЧНИКОВ}}



%Красивые настройки
\usepackage{footmisc}           %Пакет для постраничной а не сквозной нумерации сносок
\usepackage{makecell}
\usepackage{multirow}           %Улучшенное форматирование таблиц
\usepackage{ulem}               %Подчеркивания

\usepackage{enumitem}
\setlist{noitemsep}