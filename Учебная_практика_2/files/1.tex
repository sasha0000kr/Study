\chapter*{ВВЕДЕНИЕ}
\addcontentsline{toc}{chapter}{ВВЕДЕНИЕ}

Практика по профессиональному модулю \textbf{\module}
направлена на формирование у обучающегося общих компетенций:
\\


\begin{tabular}{|l|m{125mm}|}
    \hline
    \textbf{Код} & \textbf{Наименование общих компетенций}\\

    \hline
    ОК 01. & Выбирать способы решения задач профессиональной деятельности, применительно к различным контекстам.\\

    \hline
    ОК 02. & Осуществлять поиск, анализ и интерпретацию информации, необходимой для выполнения задач профессиональной деятельности.\\

    \hline
    ОК 03. & Планировать и реализовывать собственное профессиональное и личностное развитие.\\

    \hline
    ОК 05. & Осуществлять устную и письменную коммуникацию на государственном языке с учетом особенностей социального и культурного контекста.\\

    \hline
    ОК 08. & Использовать средства физической культуры для сохранения и укрепления здоровья в процессе профессиональной деятельности и поддержание необходимого уровня физической подготовленности.\\

    \hline
    ОК 09. & Использовать информационные технологии в профессиональной деятельности.\\

    \hline
    ОК 10. & Пользоваться профессиональной документацией на государственном и иностранном языке.\\

    \hline
\end{tabular}

\clearpage
\textbf{профессиональных компетенций:}
\\


\begin{tabular}{|l|m{125mm}|}
    \hline
    \textbf{Код} & \textbf{Наименование общих компетенций}\\

    \hline
    ВД 1 & Монтаж, программирование и пуско-наладка мехатронных систем и мобильных робототехнических комплексов.\\

    \hline
    ПК 1.1. & Выполнять монтаж компонентов и модулей мехатронных систем и мобильных робототехнических комплексов в соответствии с технической документацией.\\

    \hline
    ПК 1.2. & Осуществлять настройку и конфигурирование программируемых логических контроллеров и микропроцессорных систем в соответствии с принципиальными схемами подключения.\\

    \hline
    ПК 1.3. & Разрабатывать управляющие программы мехатронных систем и мобильных робототехнических комплексов в соответствии с техническим заданием.\\

    \hline
    ПК 1.4. & Выполнять работы по наладке компонентов и модулей мехатронных систем и мобильных робототехнических комплексов в соответствии с технической документацией.\\

    \hline
\end{tabular}
\\


-- и приобретение практического опыта по виду профессиональной
деятельности по профессиональному модулю \module .

\clearpage
В ходе освоения программы учебной практики студент должен: \textbf{иметь практический опыт:}
\\


\section*{Иметь практический опыт}
\begin{itemize}
\item Выполнять сборку узлов и систем, монтажа, наладки оборудования, средств измерения и автоматизации, информационных устройств мехатронных систем;
\item составлять документацию для проведения работ по монтажу оборудования мехатронных систем;
\item программировать мехатронные системы с учетом;
\item программировать мехатронные системы с учетом специфики технологических процессов;
\item программировать мехатронные системы с учетом специфики технологических процессов;
\item проводить контроль работ по монтажу оборудования мехатронных систем с использованием контрольно-измерительных приборов;
\item осуществлять пуско-наладочные работы и испытания мехатронных систем;
\item распознавание сложных проблемных ситуаций различных контекстах;
\item проведение анализа сложных ситуаций при решении задач профессиональной деятельности;
\item определение этапов решения задачи;
\item определение потребности в информации;
\item осуществление эффективного поиска;
\item выделение всех возможных источников нужных ресурсов, в том числе неочевидных;
\item разработка детального плана действий;
\item оценка рисков на каждом шагу;
\item оценка плюсов и минусов полученного результата, своего плана и его реализации, предложение критериев оценки и рекомендации по улучшению плана;
\item планирование информационного поиска из широкого набора источников, необходимого для выполнения профессиональных задач;
\item проведение анализа полученной информации, выделение в ней главных аспектов;
\item структурирование отобранной информации в соответствии с параметрами поиска;
\item интерпретация полученной информации в контексте профессиональной деятельности;
\item использование актуальной нормативно-правовой документации по \\ профессии (специальности);
\item применение современной научной профессиональной терминологии;
\item определение траектории профессионального развития и самообразования;
\item грамотно устно и письменно излагать свои мысли по профессиональной тематике на государственном языке;
\item проявление толерантность в рабочем коллективе;
\end{itemize}

\section*{Уметь}
\begin{itemize}
    \item применять технологии бережливого производства при организации и выполнении работ по монтажу и наладке мехатронных систем;
    \item читать техническую документацию на производство монтажа;
    \item читать принципиальные структурные схемы, схемы автоматизации, схемы соединений и подключений;
    \item готовить инструмент и оборудование к монтажу;
    \item осуществлять предмонтажную проверку элементной базы мехатронных систем;
    \item осуществлять монтажные работы гидравлических, пневматических, электрических систем и систем управления;
    \item контролировать качество проведения монтажных работ мехатронных систем;
    \item настраивать и конфигурировать ПЛК в соответствии с принципиальными схемами подключения;
    \item читать принципиальные структурные схемы, схемы автоматизации, схемы соединений и подключений;
    \item методы непосредственного, последовательного и параллельного программирования;
    \item алгоритмы поиска ошибок управляющих программ ПЛК;
    \item разрабатывать алгоритмы управления мехатронными системами;
    \item программировать ПЛК с целью анализа и обработки цифровых и аналоговых сигналов и управления исполнительными механизмами мехатронных систем;
    \item визуализировать процесс управления и работу мехатронных систем;
    \item применять специализированное программное обеспечение при разработке управляющих программ и визуализации процессов управления и работы мехатронных систем;
    \item проводить отладку программ управления мехатронными системами и визуализации процессов управления и работы мехатронных систем;
    \item использовать промышленные протоколы для объединения ПЛК в сеть;
    \item производить пуско-наладочные работы мехатронных систем;
    \item выполнять работы по испытанию мехатронных систем после наладки и монтажа;
    \item распознавать задачу и/или проблему в профессиональном и/или социальном контексте;
    \item анализировать задачу и/или проблему и выделять её составные части;
    \item правильно выявлять и эффективно искать информацию, необходимую для решения задачи и/или проблемы;
    \item составлять план действия;
    \item определять необходимые ресурсы;
    \item владеть актуальными методами работы в профессиональной и смежных сферах;
    \item реализовать составленный план;
    \item оценивать результат и последствия своих действий (самостоятельно или с помощью наставника);
    \item определять задачи поиска информации;
    \item определять необходимые источники информации;
    \item планировать процесс поиска;
    \item структурировать получаемую информацию;
    \item выделять наиболее значимое в перечне информации;
    \item оценивать практическую значимость результатов поиска;
    \item оформлять результаты поиска;
    \item определять актуальность нормативно-правовой документации в профессиональной деятельности;
    \item выстраивать траектории профессионального и личностного развития;
    \item излагать свои мысли на государственном языке;
    \item оформлять документы;
    \item использовать физкультурно-оздоровительную деятельность для укрепления здоровья, достижения жизненных и профессиональных целей;
    \item применять рациональные приемы двигательных функций в профессиональной деятельности;
    \item пользоваться средствами профилактики перенапряжения, характерными для данной профессии (специальности);
    \item применять средства информационных технологий для решения профессиональных задач;
    \item использовать современное программное обеспечение;
    \item понимать общий смысл четко произнесенных высказываний на известные темы (профессиональные и бытовые);
    \item понимать тексты на базовые профессиональные темы;
    \item участвовать в диалогах на знакомые общие и профессиональные темы;
    \item строить простые высказывания о себе и о своей профессиональной деятельности;
    \item кратко обосновывать и объяснить свои действия (текущие и планируемые);
    \item писать простые связные сообщения на знакомые или интересующие профессиональные темы.
\end{itemize}

\section*{Знать}
\begin{itemize}
    \item правила техники безопасности при проведении монтажных и пуско-наладочных работ и испытаний мехатронных систем;
    \item концепцию бережливого производства;
    \item перечень технической документации на производство монтажа мехатронных систем;
    \item нормативные требования по проведению монтажных работ мехатронных систем;
    \item порядок подготовки оборудования к монтажу мехатронных систем;
    \item технологию монтажа оборудования мехатронных систем;
    \item принцип работы и назначение устройств мехатронных систем;
    \item теоретические основы и принципы построения, структуру и режимы работы мехатронных систем;
    \item правила эксплуатации компонентов мехатронных систем;
    \item принципы связи программного кода, управляющего работой ПЛК, с действиями исполнительных механизмов;
    \item промышленные протоколы для объединения ПЛК в сеть;
    \item языки программирования и интерфейсы ПЛК;
    \item технологии разработки алгоритмов управляющих программ ПЛК;
    \item языки программирования и интерфейсы ПЛК;
    \item технологии разработки алгоритмов управляющих программ ПЛК;
    \item основы автоматического управления;
    \item методы визуализации процессов управления и работы мехатронных систем;
    \item методы отладки программ управления ПЛК;
    \item методы организации обмена информацией между устройствами мехатронных систем с использованием промышленных сетей;
    \item последовательность пуско-наладочных работ мехатронных систем;
    \item технологию проведения пуско-наладочных работ мехатронных систем;
    \item нормативные требования по монтажу, наладке и ремонту мехатронных систем;
    \item технологии анализа функционирования датчиков физических величин, дискретных и аналоговых сигналов;
    \item правила техники безопасности при отладке программ управления мехатронными системами;
    \item актуальный профессиональный и социальный контекст, в котором приходится работать и жить;
    \item основные источники информации и ресурсы для решения задач и проблем в профессиональном и/или социальном контексте;
    \item алгоритмы выполнения работ в профессиональной и смежных областях;
    \item методы работы в профессиональной и смежных сферах;
    \item структура плана для решения задач;
    \item порядок оценки результатов решения задач профессиональной деятельности;
    \item номенклатура информационных источников, применяемых в профессиональной деятельности;
    \item приемы структурирования информации;
    \item формат оформления результатов поиска информации;
    \item содержание актуальной нормативно-правовой документации;
    \item современная научная и профессиональная терминология;
    \item возможные траектории профессионального развития и самообразования;
    \item особенности социального и культурного контекста;
    \item правила оформления документов;
    \item роль физической культуры в общекультурном, профессиональном и социальном развитии человека;
    \item основы здорового образа жизни;
    \item условия профессиональной деятельности и зоны риска физического здоровья для профессии (специальности);
    \item средства профилактики перенапряжения;
    \item современные средства и устройства информатизации;
    \item порядок их применения и программное обеспечение в профессиональной деятельности;
    \item правила построения простых и сложных предложений на профессиональные темы;
    \item основные общеупотребительные глаголы (бытовая и профессиональная лексика);
    \item лексический минимум, относящийся к описанию предметов, средств и процессов профессиональной деятельности особенности произношения;
    \item правила чтения текстов профессиональной направленности.
\end{itemize}