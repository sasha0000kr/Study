\chapter*{УСТРАНЕНИЕ НЕИСПРАВНОСТЕЙ}
Независимо от применяемых средств процесс поиска неисправностей имеет две стадии: выбор последовательности проверки
элементов; выбор методики (способа) проведения отдельных операций проверки.

Поиск может проводиться по заранее определенной последовательности проверок или ход каждой последующей проверки определяется
результатом предыдущей. В зависимости от этого различают следующие методы проверок: последовательных поэлементных,
последовательных групповых и комбинационных.

Выбор той или иной последовательности проверок зависти от конструкции изделий в целом или их части, в которой появилась
неисправность, и может изменяться в процессе накопления информации по надежности и трудоемкости проверки элементов.

Для поиска неисправностей мехатронной системы могут применяться следующие методы:
\begin{itemize}
    \item Внешний осмотр. Наибольший эффект дает внешний осмотр включенного электрооборудования при отсутствии аварийных признаков отказа и соблюдения правил безопасности труда. Признаками неисправности в этом случае (кроме тех, которые можно обнаружить при включенном электрооборудовании) являются: появление искрений,\\дыма, нагрев отдельных деталей, появление треска и т.п. Однако внешний осмотр не позволяет обнаружить скрытые неисправности.
    \item Метод замены. Если после замены исчезают неисправности, то был заменен действительно поврежденный элемент.
    \item Метод вносимой неисправности. В этом случае в проверяемый блок вносятся искусственные повреждения, вызывающие определенные\\логические взаимодействия элементов. Контроль за параметрами схемы и анализ их изменений позволяют определить или локализовать неисправность.
    \item Метод половинного разбиения. Этот метод успешно может быть применен в том случае, если показатели надежности отдельных узлов и блоков схем электрооборудования одинаковы. Для поиска неисправности можно проверить один узел, например, по напряжению, а затем по току. Деление может быть выполнено и внутри блока или узла, что позволяет оперативно локализовать, а затем и обнаружить неисправность.
    \item Метод контрольного сигнала. Использование подобного метода обусловлено широким распространением логических элементов и микросхем в системах регулирования и управления. Для обнаружения неисправности с помощью контрольного сигнала целесообразно представить контрольную цепь диаграммой прохождения сигнала через исправную систему. Контрольному сигналу заданной формы будет соответствовать определенная реакция, анализируя которую, можно выявить работоспособность проверяемого узла или электрической цепи.
    \item Метод промежуточных измерений. Метод предусматривает осциллографирование характерных процессов, измерение напряжений на контрольных точках, контроль сопротивления отдельных элементов и электрических цепей и другие контрольно-диагностические действия, позволяющие определить место неисправности в электрооборудовании или обнаружить неисправный элемент.
    \item Метод сравнения с неисправным объектом. Метод сравнения заключается в том, что сигналы неисправности узла или блока схемы сравнивают с сигналами другого исправного или неисправного узла или блока.
\end{itemize}

Для устранения неисправностей роботов собранных из набора
TETRIX могут применяться следующие методы:
\begin{itemize}
    \item Если у вас возникли проблемы с отправкой сообщений с вашего хост-компьютера в my ROOM, убедитесь, что IP-адрес на передней панели хост-устройства совпадает с IP-адресом myRIO wireless.
    \item Если ровер движется в неправильном направлении, поменяйте полярность подключения двигателей или активируйте их реверс в программе управления.
    \item Убедитесь, что и myRIO, и аккумулятор надежно закреплены на конструкции робота, так как их вес способствует устойчивости и может менять центр масс робота.
    %\item Для балансирующих роботов в случае неправильной может потребоваться повторная калибровка. Проведение процедуры описано ниже:
    %    \begin{itemize}
    %        \item Держите робота вертикально так, чтобы его центральная линия была перпендикулярна земле
    %        \item Нажмите <<Button 0>> в нижней части myRIO, все еще удерживая робота вертикально, и быстро отпустите
    %    \end{itemize}
\end{itemize}