%60 часов, из них 20 часов теория и 40 часов практических занятий, 6 занятий по 12 часов
%Форма контроля дифф. зачет

%книга охрана труда в машиностроении

%Учебник: 



%Раздел
%\chapter{Основные понятия и терминологии безопасности труда. Основные задачи охраны труда. Основные стадии идентификации негативных производственных факторов. Классификация опасных и вредных производственных факторов. Источники опасных и вредных производственных факторов. Источники негативных факторов. Воздействия негативных факторов на человека. Нормирование и предельно допустимые уровни негативных (вредных) факторов. Опасные механические факторы. Физические негативные факторы. Химические негативные факторы. Опасные факторы комплексного характера. Опасные электрические факторы.}
%\newpage
\chapter{Идентификация и воздействие на человека негативных факторов производственной среды}
Знание методов и средств обеспечения безопасных и здоровых условий труда во многом определяет уровень профессиональной подготовленности любого специалиста, а приобретение таких знаний является целью изучения дисциплины «Охрана труда».

Предметом изучения дисциплины «Охрана труда» является система человек —-- машина --- среда, рассматриваемая с позиций весьма важного практического требования --- обеспечения безопасного функционирования.


Главными задачами изучения дисциплины «Охрана труда» являются:
\begin{enumerate}
    \item приобретение знаний, позволяющих идентифицировать (выявлять) опасные и вредные производственные факторы (ОВПФ) в машиностроении
    \item освоение системы технических мероприятий и средств по обеспечению безопасности труда в машиностроении
    \item освоение содержания основных вопросов управления промышленной безопасностью и охраной труда
    \item освоение основных направлений обеспечения пожарной безопасности на предприятиях и в организациях машиностроения
\end{enumerate}


\section{Основные понятия и терминологии безопасности труда}
%сделать команду для терминов
Охрана труда --- это система сохранения жизни и здоровья работников в процессе трудовой деятельности, включающая в себя правовые, социально-экономические, организационно-технические, санитарно-гигиенические, лечебно-профилактические, реабилитационные и другие мероприятия.

Техника безопасности --- это система организационных мероприятий и технических средств, предотвращающих воздействие на работников опасных производственных факторов.

Производственная санитария --- это система организационных мероприятий и технических средств, предотвращающих или уменьшающих воздействие на работников вредных производственных факторов.

Рабочая среда --- это пространство, в котором осуществляется трудовая деятельность человека.

Промышленная безопасность --- это состояние защищенности жизненно важных интересов личности и общества от аварий на опасных производственных объектах и послествий указаных аварий.

Опасные производственные объекты --- это цехи, участки, площадки, на которых обращаются опасные вещества, используется оборудование, работающее под давлением \(0.7MPa\) и выше или при температуре воды выше \(115^\circ C\), получаются расплавы черных и цветных металлов, ведутся горные работы, используются стационарно установелнные грузоподъемные механизмы, эскалаторы, канатные дороги.

Опасный производственный фактор --- это производственный фактор, воздействие которого на работника в определенных условиях приводит к травме или другому внезапному резкому ухудшению здоровья.

Вредный производственный фактор --- это производственных фактор, воздействие которого на работников в определенных условиях приводит к заболеванию или снижению работоспособности, может перейти в опасный фактор.

Безопасность труда --- это состояние условий труда, при котором исключено воздействие на работников ОВПФ\footnote{Опасных вредных производственных факторов} либо уровни их воздействия не превышают установленных нормативов.

Предельно допустимый уровень производственного фактора --- уровень производственного фактора, воздействие которого при работе установленной продолжительности в течение всего трудового стажа не приводит к травме, заболеванию или отклонению в состоянии здоровья в процессе работы или в отдаленные сроки жизни настоящего и последующего поколений.

Опасная зона --- пространство, в котором возможно воздействие на работника опасного и (или) вредного производственных факторов.

Средство защиты на производстве --- средство, применение которого предотвращает или уменьшает воздействие на одного или более работников ОВПФ. Если указанные средства предназначены для защиты одного работника, то их называют индивидуальными, если для двух и более -- коллективными.

\section{Классификация опасных и вредных производственных факторов}
%с 41 по 52 стр написать
Идентификация ОВПФ --- процесс выявления и оценки характеристик этих факторов. В ходе идентификации нужно выявить наименования возможных ОВПФ, их источники, фактические значения факторов, степень превышения допустимых значений.


По ГОСТ 12.0.003 все ОВПФ с учетом их материальной сущности и природы действия подразделяют на физические, химические, биологические, психофизиологические.


\subsection*{Физические ОВПФ}
Физические ОВПФ включают в себя такие факторы, как:
\begin{multicols*}{2}
\begin{enumerate}
    \item движущиеся машины и механизмы
    \item подвижные части производственного оборудования
    \item передвигающиеся изделия, заготовки,материалы
    \item разрушающиеся конструкции
    \item острые кромки, заусенцы, шероховатость на поверхностях заготовок, инструментов, материалов
    \item расположение рабочих мест на значительной высоте относительно поверхности земли (пола)
    \item обрушивающиеся горные породы, водные массы
    \item качка, невесомость
    \item повышенная запыленность и загазованность воздуха рабочей зоны
    \item повышенные или пониженные температуры поверхностей воздуха рабочей зоны
    \item повышенные или пониженные ионизация влажность, скорость движения воздуха, барометрическое давление в рабочей зоне и его резкое изменение
    \item повышенные уровни шума, вибрации, инфразвуковых колебаний, ультразвука, ионизирующих излучений в рабочей зоне, статического электричества, электромагнитных излучений, ультразвуковой и инфракрасной радиации
    \item повышенная напряженность электрического и магнитного полей
    \item повышенное значение напряжения в электрической цепи, замыкание которой может произойти через тело человека
    \item отсутствие или недостаток естественного света, недостаточная освещенность рабочей зоны, повышенная яркость света, пониженная контрастность, прямая и отраженная блесткость, повышенная пульсация светового потока
\end{enumerate}
\end{multicols*}


\subsection*{Химические и биологические ОВПФ}
Химические ОВПФ --- это вредные вещества и их соединения. По характеру действия на организм человека они могут быть токсическими, раздражающими, сенсибилизирующими, канцерогенными, мутагенными, влияющими на репродуктивную функцию. Вредные вещества могут проникать в организм человека через органы дыхания, желудочно-кишечный тракт, кожные покровы и слизистые оболочки.


Биологические ОВПФ включают в себя патогенные микроорганизмы (бактерии, вирусы, риккетсии, спирохеты, грибы, простейшие), а также опасные и вредные макроорганизмы (растения и животные).


\subsection*{Психофизиологические ОВПФ}
Психофизиологические ОВПФ с учетом природы их возникновения и характера действия подразделяют на физические перегрузки и нервно-психические перегрузки.


Физические перегрузки возможны при перемещении вручную каких-либо грузов и измеряются в единицах работы --- джоуль (Дж) или килограмм-сила-метр \(\frac{kg\times}{m}\).

Физические перегрузки могут быть статическими и динамическими. Статические перегрузки возникают при удержании инструмента, удержании тела в наклонном положении, тяге или толкании ручную, например грузовых тележек, и измеряются в единицах усилия, умноженных на время приложения усилия, т.е. в ньютон секунда \(H\times c\).


Нервно-психические перегрузки подразделяют на умственное перенапряжение, перенапряжение анализаторов, монотонность труда и эмоциональные перегрузки.