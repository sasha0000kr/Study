\chapter{Философия Милетской школы}

Милетская школа философии берет свое название от города Милет, который
был расположен в Малой Азии на Ионическом побережье (ныне -- территория
Турции). Милетская школа являлась первой школой древнегреческой
философии.

\emph{Центральный вопрос Милетской школы -- вопрос об Архе\footnote{Архе´
    (от греч. arche - начало, в лат. переводе - принцип) - термин
    древнегреческой философии, употреблявшийся в двух основных значениях:
    1-первоначало мира; 2-гносеологический принцип, отправная точка
    познания.} (первоначале мира). Первый философ и основатель школы --
  Фалес, началом всего считал воду. Из воды путем сгущения или разрежения
  возникли твердые тела и воздух.}

\section{Философия Фалеса}

Но Фалес был не только философом, но еще и астрономом, он достаточно
точно пред­сказал солнечное затмение, предположил, что Луна светит не
своим, а отраженным светом, составил первую карту неба. В ис­тории же
философии этот мудрец известен как основатель пер­вой философской школы
в европейской истории. Он сделал предположение, что началом всего
существующего является вода, влага. Дело не только в том, что, по
Фалесу, наша земля буквально плавает в воде, но и в том, что без водного
начала невозможна жизнь.

Из своих представлений о мире Фалес сделал вывод, что вода является
первоначалом всего. У Фалеса вода являлась разумной и божественной.
Философ отмечал, что вода необходима как живым организмам, так и неживой
природе, под влиянием воды тела меняют размеры.

Источником самодвижения Фалес считал душу. Он утверждал, что душа есть
как в одушевленных, так и в неодушевленных телах (приводил в пример -
магнит и янтарь). Фалес был первым философом, который объявил душу
бессмертной.

По-своему, Фалес совершил настоящую научную революцию. Он не только
нашёл мировое первоначало не где-то за пределами земного мира (в виде
божественных эманаций и т. п.), а совсем рядом, но и посмотрел на мир
рационально, не примешивая к своей теории сверхъестественные силы. По
сути, это была первая попытка посмотреть на мир как на материальную
систему, состоящую из взаимосвязанных элементов.


\section{Философия
  Анаксимандра}

Анаксимандр, второй представитель Милетской школы и ученик Фалеса,
первоначалом мира называл апейрон (по-гречески -- беспредельное) --
особое вещество, в котором материальные элементы -- вода, воздух, огонь
-- смешаны воедино. Под воздействием сил притяжения и отталкивания из
апейрона выделяются противоположности - сухое и влажное, холодное и
горячее, которые, сочетаясь, образуют все многообразие материального
мира.

Так же, как Фалес, Анаксимандр ставил вопрос о начале мира. Он
утверждал, что первоначалом и основой является апейрон (беспредельное
по-гречески), не определяя его ни как воздух, ни как воду, ни как
что-либо иное. Фалес сводил все материальное разнообразие мира к воде,
Анаксимандр же уходит от этой вещественной определенности, его мысль
более абстрактна. Его апейрон характеризуется как безграничное,
неопределенное материальное начало. Очень похоже на современные
определения понятия материя\footnote{Мате́рия (от лат. materia
  «вещество») -- физическое вещество, в отличие от психического и
  духовного.В классическом значении всё вещественное, «телесное»,
  имеющее массу, протяжённость, локализацию в пространстве, проявляющее
  корпускулярные свойства. В материалистической философской традиции
  категория «материя» обозначает субстанцию, обладающую статусом
  первоначала (объективной реальностью) по отношению к сознанию
  (субъективной реальности).} -- объективная, существующая вне нас
реальность, которая дается нам в ощущениях. Значит, материальными будут
и дерево за окном, И парта, за которой вы сидите, и солнце на небе, ведь
они существуют объективно (то есть независимо от субъекта, от нас) и мы
знаем об их существо­вании благодаря своим ощущениям. Поэтому взгляды
Анакси­мандра напоминают взгляды современных материалистов: он тоже
считал, что мир имеет своим началом нечто материальное, но не
конкретизировал, в какой форме это материальное начало существовало. Он
учил, что `части изменяются, целое же остается неизменным', то есть
алейрон может проявляться в разных стихиях, формах, вещах. В этом смысле
можно сказать, что Анаксимандр опередил свое время: его мысль гораздо
дальше от­ стоит от образного мифологического мышления, чем у Фалеса.


\section{Философия Анаксимена}

Третьим выдающимся милетским философом является Анаксимен (585-524 до н.
э.). Он был учеником и последователем Анаксимандра. Подобно Фалесу и
Анаксимандру, Анаксимен искал материальную первооснову мира. Такой
основой он считал воздух. Из воздуха затем возникло все остальное.
Разряжение воздуха приводит к возникновению огня, а сгущение вызывает
ветры, тучи, воду, землю, камни. Сгущение и разряжение понимаются здесь
как противоположные процессы, участвующие в образовании различных
состояний материи. Естественное объяснение возникновения и развития мира
Анаксимен распространяет и на происхождение богов: боги тоже произошли
из воздуха и поэтому не отличаются от других природных явлений, они тоже
материальны. При таком понимании в них остается очень мало
божественного, не правда ли?


\section{Заключение}

Посмотрите, \emph{перед нами совсем другой стиль мышления, не похожий на
  антропоморфные мифологические объяснения.} Фалес ищет причины
происходящего в самом мире, а не вне его, он обращается к рациональным
доводам и рассуждениям, а не к слепой вере и не к художественным
ассоциациям. Дело не в том, что первые философы обладали большей
истиной, чем люди с мифологическим мировоззрением, --- с точки зрения
современного человека, представлять себе мир произошедшим из воды не
менее забавно, чем верить (согласно одному из древнегреческих мифов),
что небо поддерживают своими плечами атланты. Поэтому вопрос не в том,
что мифология -- это ложь, а философия -- истина. Разумеется, это не
так. В мифах содержится значительное количество интуиций и догадок,
подтвержденных затем развитием человеческого познания, а философских
системах, как мы с вами уже отмечали, объективность и доказательность не
всегда присутствуют. Дело в другом. Изменилось познавательное отношение
к миру, действительность начали объяснять не образами, а теориями (пусть
даже наивными и неверными).

Произошел переход на другой уровень мировоззрения --
\emph{теоретический, рациональный}. В литературе такой переход часто
называют переходом «От мифа к логосу». Вы уже знаете, что «логос» на
греческом языке означает «слово», «закон», «учение». Поэтому переход
сознания человечества от мифа к логосу бьm связан с попытками поиска
закономерностей в окружающем мире, с созданием первых теорий, с
выработкой абстрактных понятий, которые начали использоваться вместо
мифологических образов в объяснении действительности. На место веры,
свойственной мифологическому и религиозному мировоззрениям, постепенно
пришло рациональное убеждение. Признаком и симптомом подобного перехода
стало возникновение философии.

\subsection{Итог}

Таким образом, для всех представителей милетской школы основным вопросом
был вопрос о первоначале и сущности мира. И хотя ответы на этот вопрос
давались разные -- будут ли считаться первоосновой вода, апейрон или
воздух -- установка на теоретическое, рациональное объяснение природы
свидетельствует о том, что мы имеем дело уже не с мифологией, а с
принципиально новым отношением к миру, которое и положило начало
формированию не только философии, но и науки.
