\newcommand{\labor}[1]{\chapter{#1}}
\newcommand{\purpose}{\paragraph{Цель работы:}}
\newcommand{\units}{{\section{Описание оборудования}}
\newcommand{\theory}{\section{Теоретические сведения}}
\newcommand{\practice}{\section{Выполнение лабораторной работы}}}
\newcommand{\tasks}{\subsection*{\normalsize{Ход работы}}}
\newcommand{\conclusion}{\paragraph{Вывод:}}


\newcommand{\scheme}{\subsection{Электрическая схема}}
\newcommand{\demo}{\subsection{Проверка работоспособности}}
\newcommand{\mod}{\subsection{Внешний вид устройства}}
\newcommand{\prog}{\subsection{Управляющая программа}}


\newcommand{\standartunits}{\input{files/standartunits.tex}}
\newcommand{\standarttasks}{\input{files/standarttasks.tex}}


\newcommand{\makepractice}{\practice\tasks\input{files/standarttasks.tex}}
\newcommand{\makeconclusion}{\conclusion{В ходе работы были применены знания о базовом программировании на Arduino. Составлена схема и программа для управления ею. Также проведена работа по оптимизации программного кода. Практическое задание выполнено успешно!}}


\newcommand{\makeunits}{\section{\large{Описание оборудования}} \input{files/standartunits.tex}}

\makeatletter
\renewcommand{\@makechapterhead}[1]{
{\parindent=0pt \centering \normalfont\large\bfseries
\center{Доклад №~\thechapter}
\center{\normalfont\Large\bfseries <<#1>>} \par
\nopagebreak \vspace{0.1cm} } }
\renewcommand{\@makeschapterhead}[1]{
{\parindent=0pt
\center{\normalfont\large\bfseries #1} \par
\nopagebreak \vspace{0.1cm} }}
\makeatother