\chapter{ФомаАквинский}

Философ и богослов Фома Аквинский (Аквинат) -- наиболее известный
представитель схоластики. Он использовал труды Аристотеля для того,
чтобы систематизировать богословские идеи. Основные его работы -- «Сумма
теологии» и «Сумма против язычников» (или «Сумма философии»).

\emph{Фома Аквинский (1225/26-1274) родился в Акуино, близ Неаполя в
богатой и влиятельной аристократической семье. Вопреки желанию
родственников Фома вступает в нищенствующий орден доминиканцев. Пытаясь
образумить Фому, родители заточили его в башне фамильного замка, где тот
провел больше года, но не отказался от своего выбора. Фома учится и
работает в Париже, Кёльне и Риме, где пишет ряд трактатов и комментариев к Библии и трудам Аристотеля. В 1274 г. по пути на собор он
умирает. В 1323 г. Фому причислили клику святых.}

В фокусе исследований Аквината -- проблема соотношения веры и разума. По
мнению философа, и вера, и разум ведут к истинному знанию, однако в
случае противоречия между ними следует отдавать предпочтение вере:
\emph{``Философия -- служанка теологии''}. Так, при помощи разума можно
доказать то, что бог существует и он един. Что касается троичности бога,
первородного греха и других идей, то их нельзя понять без откровения
(Библии) и веры.


\section{Пять доказательств
бога}

Фоме принадлежат пять доказательств существования бога:

\begin{enumerate}
\def\labelenumi{\arabic{enumi}.}
\item
  \begin{description}
  \item[\textbf{Движение --}]
  \emph{все, что движется, имеет источник движения в чем-то другом,
  следовательно, должен быть перводвигатель, т.е. бог.}
  \end{description}
\end{enumerate}

Не подлежит сомнению и подтверждается чувствами, что в этом мире нечто
движется. Но все, что движется, имеет источник движения. Следовательно,
должен быть перводвигатель, так как не может быть бесконечной цепи
движущих предметов. А перводвигатель -- это Бог. (Это доказательство
Аквината прямо опирается на философию Аристотеля и его учение о
перводвигателе.)

\begin{enumerate}
\def\labelenumi{\arabic{enumi}.}
\setcounter{enumi}{1}
\item
  \begin{description}
  \item[\textbf{Причина --}]
  \emph{все имеет причину, цепь причин не может уходить в бесконечность,
  следовательно, существует первопричина, т.е. бог.}
  \end{description}
\end{enumerate}

Каждое явление имеет причину. Но у причины тоже есть причина и так
далее. Значит, должна быть верховная причина всех реальных явлений и
процессов, а это Бог. (И опять видна перекличка с аристотелевским
учением о «форме всех форм».)

\begin{enumerate}
\def\labelenumi{\arabic{enumi}.}
\setcounter{enumi}{2}
\item
  \begin{description}
  \item[\textbf{Необходимость --}]
  \emph{случайное зависит от необходимого, значит, существует высшая,
  божественная необходимость, т.е. бог.}
  \end{description}
\end{enumerate}

Люди видят, что вещи возникают и гибнут. Рано или поздно они перестают
существовать. То, что стул, на котором вы сейчас сидите, существует --
случайность с точки зрения мироздания, его могло бы и не быть. Но если
все может быть, а может и не быть, то ко­ гда-нибудь в мире ничего не
будет. Если это так, то уже сейчас ничего не должно быть. Но так как мир
не исчез, значит, существующее случайно «Подпитывается» чем-то
необходимым. Такой абсолютно необходимой сущностью является Бог.

\begin{enumerate}
\def\labelenumi{\arabic{enumi}.}
\setcounter{enumi}{3}
\item
  \begin{description}
  \item[\textbf{Качество --}]
  \emph{все имеет разные степени качеств (хуже, лучше), значит, должен
  быть эталон -- высшее совершенство, т.е. бог.}
  \end{description}
\end{enumerate}

Люди считают одни вещи лучше других. Эта девушка красивее своей соседки,
а этот юноша -- умнее приятеля. Но с чем мы сравниваем? Где «масштаб»
для сравнения? Мы должны чувствовать, что есть «пресовершенная красота,
совершенный ум и т. д. Чем ближе вещь к этому пределу, тем она кажется
нам лучше. Значит, есть то, что обладает этим предельным качеством, --
это Бог. (А это доказательство напоминает по логике рассуждения диалоги
Сократа.)

\begin{enumerate}
\def\labelenumi{\arabic{enumi}.}
\setcounter{enumi}{4}
\item
  \begin{description}
  \item[\textbf{Цель --}]
  \emph{все в мире имеет цель, следовательно, существует высшее разумное
  начало, которое и направляет все в мире к цели, т.е. бог.}
  \end{description}
\end{enumerate}

Все предметы, лишенные разума, устроены целесообразно. Крьmья бабочки
пригодны для того , чтобы летать с цветка на цветок, а орла - чтобы
парить в вышине. Тигры полосатые, и их окраска помогает им скрываться в
джунглях. Почему все так устроено? Поскольку сами предметы лишены
разумения, постольку их должен направлять некто , одаренный разумом.
Значит, есть разумное существо, полагающее цель для всего, что
происходит в природе. Это Бог.


\section{Томизм}

Учение Фомы известно как томизм \emph{(по-латыни Фома -- Thomas)}. Идеи
Аквината популярны до сих пор, и современная католическая философия
известна как неотомизм.

Ватикан в Средние века, а сегодня католическая
стоит на позициях неотомизма, то есть видоизмененного, «Продолженного»
учения Фомы Аквинского. За свои заслуги перед церковью Фома Аквинат в
1323 г. был причислен к лику святых, а в 1567
г. бьm признан пятым великим учителем церк­ви. В 1879 г. его учение было
объявлено «единственно истин­ной» философией католицизма.


\section{Итог}

Целью учения Аквината было показать, что разум и философия не
противоречат вере. Двигаясь к истине, разум может вступить в
противоречие с догматом веры. По мнению Фомы, в этом случае ошибается
разум, так как в божественном Откровении ошибок нет. Но философия и
религия имеют общие положения, поэтому лучше понимать и верить, чем
просто верить. Есть истины, которые недоступны рациональному познанию, а
есть истины, которые оно может постичь. Например, разум может
свидетельствовать, что Бог есть (и в своих сочинениях Фома Аквинский
пытался именно с помощью разумных, логических доводов доказать Его
существование).

