\documentclass[a4paper, 14pt]{extreport}
%Настройка страниц
\usepackage[left=1.5cm, right=1.5cm, top=2cm,
    bottom=2cm, bindingoffset=0.5cm]{geometry}
\usepackage{cmap}				%Ссылки в PDF
\usepackage[T2A]{fontenc}		%Шрифты
\usepackage[utf8]{inputenc}		%Кодировка
\usepackage[russian]{babel}		%Грамматика
%\frenchspacing					%Отключает большой пробел между предложениями
\usepackage[nodisplayskipstretch]{setspace}
\usepackage{indentfirst}		%Красная строка
\setlength{\parindent}{1.25cm}	%Настройка отступа красной строки для шрифта
%\linespread{1.3} 				%Межстрочный интервал
\usepackage{setspace}
\onehalfspacing
\usepackage{microtype}          %Микротипографические эффекты
%\usepackage{parskip} 			%Интервал между абзацами
%\setlength{\parindent}{1.5} 	%Настройка интервала между абзацами
\usepackage{enumitem}           %Фикс списков
\setlist{noitemsep}             %Убираем расстояние между элементами списка
\sloppy

\clubpenalty=5000
\widowpenalty=5000

\usepackage{pscyr}
\title{Основы философии}
\author{Набор и верстка: Краснов Александр Сергеевич}
\date{\today}

\begin{document}
\maketitle
\tableofcontents
\newpage
\chapter{Общие сведения о философии}
\section{История философии}

Философом впервые назвал себя древнегреческий математик и мыслитель
Пифагор.

\emph{Термин «философия» был составлен им из двух слов \textbf{phileo}
-- люблю + \textbf{sophia} -- мудрость и дословно обозначал
«\textbf{любовь к мудрости}».}

Только постоянное стремление к мудрости делает человека философом.
Поэтому философом, в большей или меньшей мере, можно назвать всякого
самостоятельно мыслящего человека.

\subsection{Предмет и функции философии}

Крупнейший мыслитель античности Платон сказал, что философия есть не
просто размышления о мире и человеке, а особый образ жизни,
заключающийся в стремлении к совершенству. \emph{Философия --- это сама
жизнь, т.е. свободное формирование своей судьбы в сложном и
неоднозначном мире.}

Платон в своем определении указывает на то назначение философии, которое
состоит в постоянном возвышении человека, в его выходе за узкие пределы
времени и досужих мнений к фундаментальному и всеобщему.

Основатель немецкой классической философии Имануил Кант называет
главными целями человеческого разума высшие вопросы о возможностях
человека. \textbf{Вопрос о человеке для философии -- решающий}. Все
составляющие философии и все предварительные вопросы должны
рассматриваться через призму человека: все, что не способствует
достижению человеческих целей, есть спекулятивное знание, а не философия
в собственном смысле.

Немецкий мыслитель Георг Гегель полагал, что философия отражает не
только личные идеи отдельного мыслителя: в нее впечатано само
историческое время его жизни со всеми особенностями и противоречиями,
ценностями и жизненными смыслами.

Философия, размышляя о культуре в целом (религии, искусстве, науке,
морали), выделяет все самое существенное и отображающее «живую душу
человечества» в наиболее концентрированном виде. В этом отношении
философия есть квинтэссенция культуры и всех достижений человечества, а
изучение философии --- самый короткий путь к пониманию культуры.

Немецкий философ Фридрих Энгельс полагал, что философия есть наука о
наиболее общих законах. Это значит, что она не интересуется ничем
сиюминутным и преходящим и не разменивается на частности и случайности,
ее предмет - фундаментальные основы знаний, действий и ценностей, их
закономерные, повторяющиеся и устойчивые связи. Поэтому можно сказать,
что философия лежит в основании всех других наук, рассматривая мир
обобщенно, как единое целое, в то время как любая другая наука имеет
дело лишь с тем или иным фрагментом окружающего мира.

Для Энгельса философия есть не просто «любомудрие», а наука в полном
смысле этого слова - строгая система знаний со своими методами и
категориями. Всеобщие законы, которые открывает философия, могут быть
применены как для научного обоснования настоящей и будущей деятельности
человечества, так и для изменения самого мира и человека.

Таким образом, в различных исторических традициях философия понимается
по-разному. Каждый мыслитель выделяет свой аспект ее рассмотрения.

\section{Происхождение философии}

Философия одна из самых древних областей человеческого знания. Оно
насчитывает около 25 столетий своей истории. Философия зародилась в
период становления человеческой цивилизации, где-то на рубеже 8--9 века
до нашей эры в древней Индии, Китае, но свою классическую форму обрела в
древней Греции.

Термин философия впервые объяснил греческий философ Пифагор в 6-5 веке до нашей эры.

\subsection{Что такое мудрость?}

В древности понятие мудрости означало стремление к особому
интеллектуальному, рациональному познанию мира основанному на знании.
Причем мудрость означала не только накопление междисциплинарных знаний
об отдельных видах или частях мира, а о стремлении понять мир как
целостных в совей основе.

\begin{quote}
\emph{Мудрость -- поиск истины.}
\end{quote}

Философия как любовь к мудрости и истине стала синонимом зарождения
теоретического рационального знания о мире.

По своему содержанию она представляла собой синкретитеское\footnote{
\emph{Синкретизм (др.-греч.объединение)} ---
  сочетание разнородных философских начал в одну систему. В отличие от
  эклектизма, синкретизм сочетает эти начала без их объединения.
  Слитность, нерасчлененность, характеризуемая для первоначального
  состояния в развитии чего-нибудь.
  } нерасчлененное знание и включала в себя всю
совокупность знаний о мире и человеке: начало математики, астрономии,
механики, медицины, истории, психологии, этики. Но на ряду с зачатками
различных наук совокупность знаний древних мыслителей включало также
обобщенных размышления людей о мире и о себе, о тайнах мироздания и о
судьбе человечества.

\section{Становление философии}

Но наряду с зачатками различных наук совокупное знание древних
мыслителей в своих размышлениях они искали ответы на вопросы:

\begin{itemize}

\item
  Как возник мир?
\item
  В чем сущность мира?
\item
  Как устроен мир?

  \begin{itemize}
  
  \item
    А есть ли что-то общее, что свойственно всем без исключения явлениям
    мира и вещам, в чем оно состоит, т.е. что является первоосновой
    мира?
  \end{itemize}
\item
  Конечен или бесконечен мир?
\item
  Какого место человека в мире?
\item
  Познаваем ли мир?
\item
  Что такое сознание?
\item
  Что есть истина?
\item
  Какова природа человека?
\item
  Что такое общество и как прийти к лучшему его устройству?
\item
  Что такое добро и зло?
\item
  В чем счастье человека?
\item
  Что такое прогресс и каковы его критерии?
\end{itemize}

\emph{Сущность} ---
совокупность качеств и свойств, присущих предмету, определяющее именно
этот предмет.


\subsection{Мировоззрение}

\emph{1. Мировоззрение} ---
это совокупность наиболее общих взглядов на мир в целом и место человека
в нем.
\emph{2. Мировоззрение} ---
это система взглядов, определяющая отношение человека к миру, дающая
ориентиры и регулирующая его поведение; совокупность знаний, убеждений,
ценностей и норм.


\subsection{Мифология}

\begin{quote}
\emph{Миф} -- это придание, сказание + \emph{logos} -- от
древнегреческого слово, учение.
\end{quote}

\emph{Мифология} ---
исторически первый тип мировоззрения, форма общественного сознания
древнего общества.
\emph{Миф} ---
единая синкретическая форма сознания, в которой соединены зачатки
знаний, элементы верований, политических взглядов, различных видов
искусства, собственно философии.


Мифы пытаются дать ответы на вопросы:

\begin{enumerate}
\def\labelenumi{\arabic{enumi}.}

\item
  Происхождение вселенной, Земли и человека
\item
  Объяснение природных явлений
\item
  Жизнь, судьба, смерть человека, деятельность человека и его достижения
\item
  Вопросы чести, долга и нравственности
\end{enumerate}

Для мифологии характерны черты:

\begin{enumerate}
\def\labelenumi{\arabic{enumi}.}

\item
  Эмоционально-образная форма -- не жесткие разграничения мыслей и
  эмоций, художественных образов и знаний
\item
  Очеловечивание природы -- перенос человеческих черт на окружающий мир,
  олицетворение и одушевление природных сил
\item
  Отсутствие рефлексии\footnote{
    
    \emph{Рефлексия} ---
    работа сознания, мысли человека над собственным сознанием,
    размышление над своими взглядами, психическим состоянием и их
    оценка.
    }
\item
  Утилитарная направленность мифологии, которая проявляется в том, что
  решаемые ею мировоззренческие проблемы тесно связаны с практическими
  задачами: с хозяйством, с счастливой жизнью, с защитой от голода,
  болезней, нужды и т.д.
\end{enumerate}

\subsection{Отличие философского мировоззрения}

\begin{quote}
\emph{Философия ---}
особый, научно--теоретический вид мировоззрения

\end{quote}

Философское мировоззрение отличается от мифологического и религиозного
тем, что оно:

\begin{enumerate}
\def\labelenumi{\arabic{enumi}.}

\item
  Основано на знании, а не на вере или вымысле
\item
  Оно рефлексивно -- в философии содержаться размышления над
  собственными представлениями о мире и о человеке в этом мире.
  Философия требует сомнения, допускает критику идей, отказ от веры в
  догматы и постулаты.
\item
  Философское мышление логично -- положение философии не просто
  утверждаются, а выводятся, доказываются в систематизированной
  логически упорядоченной форме
\item
  Опирается на четкие понятия и категории
\end{enumerate}

Таким образом философия представляет собой вид мировоззрения,
отличающийся рациональностью, системностью, логикой и теоретической
оформленностью.

\section{Предмет и функции философии}

В любой теории различаются предметы и объекты. Объект составляет всю
реальность, попадающую в поле внимания. Предмет представляет те стороны,
свойства реальности, которые выявляются в связи со специфическими целями
изучения.

\emph{Цель философии} ---
понять принципы и закономерности по которым живет и развивается мир,
человек, общество.


\begin{quote}
\emph{Основная цель философии} --- сформировать мировоззрение.
\end{quote}

Философия -- это учение о сущности мира и о человеке, но не мир, и не
человек, взятые сами по себе, а система отношений \textbf{мир --
человек}.

\subsection{Определение философии}

\begin{quote}
\emph{ФИЛОСОФИЯ} ---
(греч. phileo -- люблю, sophia -- мудрость; любовь к мудрости) ---
особая форма познания мира, вырабатывающая систему знаний о
фундаментальных принципах и основах человеческого бытия, о наиболее
общих сущностных характеристиках человеческого отношения к природе,
обществу и духовной жизни во всех ее основных проявлениях.


\emph{Всемирная энциклопедия философии, М., 2001.Всемирная энциклопедия
философии, М., 2001.}
\end{quote}

\emph{Философия} (представляет из себя и является):

\begin{enumerate}
\def\labelenumi{\arabic{enumi}.}

\item
  Формой общественного сознания, направленная на выработку целостного
  взгляда на мир и место в нем человека.
\item
  Исследование самых общих и фундаментальных вопросов о сущности
  природы, человека и общества.
\item
  Наука о всеобщих законах развития природы, общества и человеческого
  мышления.
\end{enumerate}

\subsection{Функции философии}

\textbf{Кому нужна философия?} Философия -- это выбор интеллектуально
независимого, внутренне свободного человека. Философия не обещает
достижения истины, но обеспечивает тому, кто ее ищет, полную свободу
поиска. Поэтому философия необходима обществу в целом: как яркое
проявление свободомыслия, она постоянно ищет и предугадывает пути
социальных изменений, формируя новые идеи или предупреждая о будущих
сложностях. Философия -- осмысление изменений, без которых развитие
невозможно.

\subsubsection{Основные функции философии:}

\begin{enumerate}
\def\labelenumi{\arabic{enumi}.}

\item
  \emph{Мировоззренческая} -- служит выработке системы взглядов на
  объективный мир и место в нем человека, на отношение человека к
  окружающей его действительности и самому себе, а также обусловленные
  этими взглядами основные жизненные позиции людей, их убеждения,
  идеалы, принципы познания и деятельности, ценностные ориентации.
\item
  \emph{Методологическая} -- вырабатывает общие принципы и способы
  теоретического и практического освоения действительности.
\item
  \emph{Гуманистическая} -- рассматривает мир через призму человека и
  его целей, обосновывает его самоценность, его права и свободы.
\item
  \emph{Аксиологическая} -- дает оценку миру и человеку, обосновывает
  значимость духовных ценностей (истины, добра, красоты и т.д.).
\item
  \emph{Воспитательная} -- заключается в рекомендации следовать
  положительным нормам и идеалам нравственности.
\item
  \emph{Общекультурная} -- знакомит человека с высшими достижениями
  мировой интеллектуальной культуры, является показателем уровня
  образованности и культуры человека.
\item
  \emph{Прогностическая} -- прогнозирует тенденции развития научного
  знания и социальных изменений.
\item
  \emph{Интегрирующая} -- систематизирует и обобщает данные всех наук,
  т.е. выступает как наиболее общая «наука наук».
\item
  \emph{Критическая} -- подвергает критике устаревшие взгляды, нормы,
  учения; формирует свободное, недогматическое мышление.
\end{enumerate}

Основными из перечисленных функций являются мировоззренческая и
методологическая: они характерны не только для философии, формирующей
как общую картину мира, так и методы для других наук. Остальные функции
являются дополняющими: гуманистическая, аксиологическая и обще
культурная конкретизируют мировоззрение (и человека, и общества),
прогностическая, интегрирующая и критическая важны для научного освоения
мира и встроены в методологию этого процесса.

Таким образом, определение функций философии показывает, что она
выступает в двух основных качествах: во-первых, как форма мировоззрения
и, во-вторых, как особая наука.

\chapter{Основные направления Философии}

\textbf{\emph{Основной вопрос философии ---}}
основной вопрос философии это вопрос о бытие и мышления к бытию а
сознания к материи и бытию.


Он имеет две стороны:

\begin{itemize}
\item
  Первая связанна с представление о том что является первичным,
  материя(бытие) или сознание(мышление).
\item
  Вторая сторона философии это вопрос о том способно ли человеческое
  мышление познать мир.
\end{itemize}

В зависимости от решения первой стороны ОВФ, философы делятся на два
больших лагеря -- \emph{Материалистов} и \emph{Идеалистов}.

\begin{quote}
Материализм исходит из того что первичным является материя, природа а
мышление, сознание, духовное является вторичным производным от материи.
\end{quote}

\subsection{Идеализм}

Идеализм утверждает что материя это продукт, производное сознание
мышление.

Идеализм имеет две формы -- \emph{Обьективный}, \emph{Субьективный}.

Различные формы идеализма связанны с различием представлений об
особенностях того исходного сознания которое порождает материю.

\subsubsection{Обьективный идеализм}

Рассматривает в качестве такого исходного сознание надличностной то есть
не связанной ни с каким то конкретным человеком идеальную субстанцию.

\begin{itemize}

\item
  Мир идей -- Платон
\item
  Обсолютная идея
\item
  Мировая воля
\end{itemize}

\subsubsection{Субьективный идеализм}

Рассматривает в качестве такого исходного считает сознание отдельного
конкретного человека его ощущения, восприятия, формы самопознания и т.д.
Субьективные идеалисты считают что материя, природа есть продукт
сознания каждого отдельного человека.

\begin{itemize}

\item
  Джордж Беркли (Мир есть комплекс моих ощушений, существовать значит
  быть воспринимаемым).
\end{itemize}

\subsection{Вторая сторона основы философии}

Вторая сторона основы философии -- это сторона о том как наши мысли об
окружающен нас мире относотся к самому этому миру (познаваем ли мир).

Исходя из того как философы отвечают на этот вопрос, существуют
следующие направления:

\subsubsection{Агностицизм (Агностос (Agnostos) -- недоступные познания)
--- ученье отрицающее познание мира}

\emph{Агностики не верят, либо возможность человека познать мир либо в
позноваемость самого мира или же допускают ограниченную возможность
познания}

Иммануил Кант выдвинул последовательную теорию Агностицизма согласно
которой: * Человек обладает ограничинными познавательными возможностями
так как ограниченны позновательные возможности знания. * Сам окружающий
мир не познаваем впринципе - человек может познать внешнюю сторону
предметов и явлений, но не когда не познает внутреннюю сущность данных
предметов и явлений ``вещей в себе''.

\subsubsection{Скептицизм --- отрицает возможность достоверного знания}

В скептицизме мир познаваем но человеческому знанию он не доступен.

\begin{itemize}

\item
  Позновательные возможности человека ограниченны: эмоции, логика,
  корыстные интересы мешают познанию. Сознание ограниченно а мир
  бесконечно сложен.
\item
  Скептики опираются на релятивизм(отностильный) который подчеркивает
  изменчивасть реальности и отрицает её устойчивость. Познание релятивно
  и следовательно не достоверно(изменчего).
\end{itemize}

Существует направление философии которое считает что человек обладает
потонциально безграничными возможностями познания.

\subsection{Филосовские направления по методам познания}

В зависимости от метода используемого познания существуют филосовские
направления Философии:

\begin{itemize}

\item
  Рационализм -- признает основы и познание разума.
\item
  Сенцуализм -- за основу берутся чуства и следовательно знания являются
  результатом деятельности органов чуств -- ощущений.
\item
  Эмпиризм -- источник знаний и критерий их истинности лежит в
  импирческом опыте, анализе процессов и их истинности.
\item
  Ирационализм -- отрицает разума, эмпирического опыта, чувств познаний,
  основное внимание уделяется иррациональным способам познания. Акцент
  делается на субьективном опыта человека -- на интуиции (Ницще и
  Шоппенгауэр).
\end{itemize}

\chapter{Античная философия}
\section{Древнегреческая философия}

Античная философия -- первая в истории попытка рационального постижения
окружающего мира.

Античная философия существовала в течении более тысячелетия с \emph{6
века до нашей эры по 6 век нашей эры}. За это время она прошла, как и
вся античная культура, замкнутый цикл от зарождения к расцвету, а через
него к упадку и гибели.

\subsection{Периоды древнегреческой филосовии}

\begin{enumerate}
\def\labelenumi{\arabic{enumi}.}
\item
  Протофилософия (\emph{protos} от лат. --- первый) -- период зарождения
  и формирования, 6 век до нашей эры. Это первая попытка понять
  первоосновы мира, от Фаллеса до Аннаксагора.
\item
  Зрелость и расцвет (5-4 век до нашей эры).

  \begin{itemize}
  
  \item
    Софисты\footnote{Софи́сты -- древнегреческие платные преподаватели
      красноречия, представители одноимённого философского направления,
      распространённого в Греции во 2-й половине V --- 1-й половине IV
      веков до н. э. Изначально термин «софист» служил для обозначения
      искусного или мудрого человека, однако уже в древности приобрёл
      уничижительное значение: Платон указывал на высокие гонорары за
      обучение у софистов, их самовосхваление и не всегда честные приёмы
      полемики. Сейчас софистами называют демагогов, которые стараются
      убедить людей в нужном им мнении.} -- философская школа
    (направление)
  \item
    Сократ, до Платона и Аристотеля
  \end{itemize}
\item
  Закат -- греческая философия эпохи оккультизма и философия периода
  Римской республики (3-1 век дол нашей эры).

  \begin{itemize}
  
  \item
    Эпикур -- философ
  \item
    Древняя стоя\footnote{В переносном смысле стоицизм --- твёрдость и
      мужество в жизненных испытаниях. (\emph{4 -- середина 3 в до н.
      э.}) Представители: Зенон Китийский, Клеанф, Хрисип, Зенон
      Тарсский, Диоген Вавилонский и Антипатр из Тарса. Представителями
      школы стоицизма в Древнем Риме были Марк Аврелий, Эпиктет, Сенека.}
  \item
    Скептицизм\footnote{Скептици́зм (от др.-греч. σκεπτικός ---
      рассматривающий, исследующий) --- философское направление,
      выдвигающее сомнение в качестве принципа мышления, особенно
      сомнение в надёжности истины.}
  \end{itemize}
\item
  Упадок и гибель (в эпоху Римской империи) (\emph{1-6 век нашей эры}).

  \begin{itemize}
  
  \item
    Римские стоики
  \item
    Неоплатоники\footnote{Неоплатониками называют тех мыслителей под
      конец древнего мира, которые пытались начертать новую систему
      философии, соединивши философию Платона с некоторыми положениями
      восточных верований.}
  \end{itemize}
\end{enumerate}

\subsection{Основные черты древнегреческой философии}

Древнегреческая философия характеризуется стремлением познать мир в
целом, природу, космос, а затем уже и человека. \emph{Отсюда
космоцентризм\footnote{Космоцентризм --- философское направление
  античности, система философских взглядов, появившаяся в Древней
  Греции, согласно которой мир воспринимается как космос, разнообразный,
  гармоничный и одновременно способный вселить ужас. Все явления
  окружающего мира рассматривались через призму космоса. Космос
  охватывает Землю, человека, небесные светила. Он замкнут, имеет
  сферическую форму и в нем происходит постоянный круговорот --- все
  возникает, течет и изменяется. Из чего возникает и к чему
  возвращается, согласно данному учению - неясно.} ранний греческой
философии.}

В представлении древних Греков космос охватывал фактически землю и небо,
человека и общество, животных и растений. Он воспринимался видимой
замкнутой сферой, конечной по своим размерам, но в ней все двигалось,
текло и изменялось, возникало и исчезало. Все в космосе было текучим,
изменчивым, и философы искали ответы на вопросы:

\begin{enumerate}
\def\labelenumi{\arabic{enumi}.}

\item
  Из чего все возникает?
\item
  Во что превращается?
\item
  Где первооснова мира?
\item
  В чем состоит субстанция первоосновы?
\end{enumerate}

\textbf{Главным вопросом древнегреческой философии был вопрос о
первоначале мира.} И в этом смысле философия пересекается с мифологией,
но если мифология решала этот вопрос по принципу -- кто родил сущее, то
философии стоит искать субстанциональное начало из чего все произошло.

\subsection{Философские школы Древней греции}

Представители милетской школы считали,, что основой мира выступают
некоторые чувственно-воспринимаемые элементы: вода, воздух, земля, огонь
или апейрон (нечто неопределенное, но материальное).

Пифагорийцы видели первооснову мира в математических элементах --
числах, соотношения которых и определяет мировые процессы.

Атомисты видели основу мира в чувственно не воспринимаемых частицах --
неделимых атомах. Платон и его школа считали, что мир, вещи, лишь тени
идей, результаты их временного воплощения.

\subsubsection{Недостатки}

\begin{enumerate}
\def\labelenumi{\arabic{enumi}.}

\item
  Объяснение мира были наивными и противоречивыми
\item
  Отсутствие доказательств
\end{enumerate}

\subsubsection{Преимущества}

Однако, громадным шагом вперед явилось следующее: 1. Присуще общее,
единое: все они стремились объяснить мир, космос из него самого. 2.
Сверх естественым силам, богам они отводили нерешающее место и роль. В
их космосе все двигалось и изменялось, было взаимосвязано и
взаимообусловлено.

\section{Философия Милетской школы}

Милетская школа философии берет свое название от города Милет, который
был расположен в Малой Азии на Ионическом побережье (ныне -- территория
Турции). Милетская школа являлась первой школой древнегреческой
философии.

\emph{Центральный вопрос Милетской школы -- вопрос об Архе\footnote{Архе´
  (от греч. arche - начало, в лат. переводе - принцип) - термин
  древнегреческой философии, употреблявшийся в двух основных значениях:
  1)первоначало мира; 2)гносеологический принцип, отправная точка
  познания.} (первоначале мира). Первый философ и основатель школы --
Фалес, началом всего считал воду. Из воды путем сгущения или разрежения
возникли твердые тела и воздух.}

\subsection{Философия Фалеса}

Но Фалес был не только философом, но еще и астрономом, он достаточно
точно пред­сказал солнечное затмение, предположил, что Луна светит не
своим, а отраженным светом, составил первую карту неба. В ис­тории же
философии этот мудрец известен как основатель пер­вой философской школы
в европейской истории. Он сделал предположение, что началом всего
существующего является вода, влага. Дело не только в том, что, по
Фалесу, наша земля буквально плавает в воде, но и в том, что без водного
начала невозможна жизнь.

Из своих представлений о мире Фалес сделал вывод, что вода является
первоначалом всего. У Фалеса вода являлась разумной и божественной.
Философ отмечал, что вода необходима как живым организмам, так и неживой
природе, под влиянием воды тела меняют размеры.

Источником самодвижения Фалес считал душу. Он утверждал, что душа есть
как в одушевленных, так и в неодушевленных телах (приводил в пример -
магнит и янтарь). Фалес был первым философом, который объявил душу
бессмертной.

По-своему, Фалес совершил настоящую научную революцию. Он не только
нашёл мировое первоначало не где-то за пределами земного мира (в виде
божественных эманаций и т. п.), а совсем рядом, но и посмотрел на мир
рационально, не примешивая к своей теории сверхъестественные силы. По
сути, это была первая попытка посмотреть на мир как на материальную
систему, состоящую из взаимосвязанных элементов.

\subsection{Философия Анаксимандра}

Анаксимандр, второй представитель Милетской школы и ученик Фалеса,
первоначалом мира называл апейрон (по-гречески -- беспредельное) --
особое вещество, в котором материальные элементы -- вода, воздух, огонь
-- смешаны воедино. Под воздействием сил притяжения и отталкивания из
апейрона выделяются противоположности - сухое и влажное, холодное и
горячее, которые, сочетаясь, образуют все многообразие материального
мира.

Так же, как Фалес, Анаксимандр ставил вопрос о начале мира. Он
утверждал, что первоначалом и основой является апейрон (беспредельное
по-гречески), не определяя его ни как воздух, ни как воду, ни как
что-либо иное. Фалес сводил все материальное разнообразие мира к воде,
Анаксимандр же уходит от этой вещественной определенности, его мысль
более абстрактна. Его апейрон характеризуется как безграничное,
неопределенное материальное начало. Очень похоже на современные
определения понятия материя\footnote{Мате́рия (от лат. materia
  «вещество») -- физическое вещество, в отличие от психического и
  духовного.В классическом значении всё вещественное, «телесное»,
  имеющее массу, протяжённость, локализацию в пространстве, проявляющее
  корпускулярные свойства. В материалистической философской традиции
  категория «материя» обозначает субстанцию, обладающую статусом
  первоначала (объективной реальностью) по отношению к сознанию
  (субъективной реальности).} -- объективная, существующая вне нас
реальность, которая дается нам в ощущениях. Значит, материальными будут
и дерево за окном, И парта, за которой вы сидите, и солнце на небе, ведь
они существуют объективно (то есть независимо от субъекта, от нас) и мы
знаем об их существо­вании благодаря своим ощущениям. Поэтому взгляды
Анакси­мандра напоминают взгляды современных материалистов: он тоже
считал, что мир имеет своим началом нечто материальное, но не
конкретизировал, в какой форме это материальное начало существовало. Он
учил, что ``части изменяются, целое же остается неизменным'', то есть
алейрон может проявляться в разных стихиях, формах, вещах. В этом смысле
можно сказать, что Анаксимандр опередил свое время: его мысль гораздо
дальше от­ стоит от образного мифологического мышления, чем у Фалеса.

\subsection{Философия Анаксимена}

Третьим выдающимся милетским философом является Анаксимен (585-524 до н.
э.). Он был учеником и последователем Анаксимандра. Подобно Фалесу и
Анаксимандру, Анаксимен искал материальную первооснову мира. Такой
основой он считал воздух. Из воздуха затем возникло все остальное.
Разряжение воздуха приводит к возникновению огня, а сгущение вызывает
ветры, тучи, воду, землю, камни. Сгущение и разряжение понимаются здесь
как противоположные процессы, участвующие в образовании различных
состояний материи. Естественное объяснение возникновения и развития мира
Анаксимен распространяет и на происхождение богов: боги тоже произошли
из воздуха и поэтому не отличаются от других природных явлений, они тоже
материальны. При таком понимании в них остается очень мало
божественного, не правда ли?

\subsection{Заключение}

Посмотрите, \emph{перед нами совсем другой стиль мышления, не похожий на
антропоморфные мифологические объяснения.} Фалес ищет причины
происходящего в самом мире, а не вне его, он обращается к рациональным
доводам и рассуждениям, а не к слепой вере и не к художественным
ассоциациям. Дело не в том, что первые философы обладали большей
истиной, чем люди с мифологическим мировоззрением, --- с точки зрения
современного человека, представлять себе мир произошедшим из воды не
менее забавно, чем верить (согласно одному из древнегреческих мифов),
что небо поддерживают своими плечами атланты. Поэтому вопрос не в том,
что мифология -- это ложь, а философия -- истина. Разумеется, это не
так. В мифах содержится значительное количество интуиций и догадок,
подтвержденных затем развитием человеческого познания, а философских
системах, как мы с вами уже отмечали, объективность и доказательность не
всегда присутствуют. Дело в другом. Изменилось познавательное отношение
к миру, действительность начали объяснять не образами, а теориями (пусть
даже наивными и неверными).

Произошел переход на другой уровень мировоззрения --
\emph{теоретический, рациональный}. В литературе такой переход часто
называют переходом «От мифа к логосу». Вы уже знаете, что «логос» на
греческом языке означает «слово», «закон», «учение». Поэтому переход
сознания человечества от мифа к логосу бьm связан с попытками поиска
закономерностей в окружающем мире, с созданием первых теорий, с
выработкой абстрактных понятий, которые начали использоваться вместо
мифологических образов в объяснении действительности. На место веры,
свойственной мифологическому и религиозному мировоззрениям, постепенно
пришло рациональное убеждение. Признаком и симптомом подобного перехода
стало возникновение философии.

\subsubsection{Итог}

Таким образом, для всех представителей милетской школы основным вопросом
был вопрос о первоначале и сущности мира. И хотя ответы на этот вопрос
давались разные -- будут ли считаться первоосновой вода, апейрон или
воздух -- установка на теоретическое, рациональное объяснение природы
свидетельствует о том, что мы имеем дело уже не с мифологией, а с
принципиально новым отношением к миру, которое и положило начало
формированию не только философии, но и науки.

\section{Философия и атомистическая теория Демокрита}

\subsection{Краткие биографические данные}

Демокрит (460-370 г. до н.э) -- натурфилософ, материалист, человек
энциклопедических знаний.~Он работал во всех областях: этике,
математике, физике, астрономии, медицине, теории музыки.~Он создал
учение об атоме.~

Первым, что понял Демокрит была Восточная мудрость, но решающее влияние
на формирование идей Демокрита изобразили не восточные мудрецы, а
Левкипп.

Левкипп (5 век до н.э) -- древнегреческий философ, один из
основоположников атомистики (ввел в рассмотрение понятие атома, как
неделимой частицы материи). Он учил, что мир состоит из атомов
(``atomos'' -- неделимый).

\subsection{Античный атомизм}

Демокрит был атомистом и считал, что любая вещь состоит из атомов и
пустоты. Вещи возникают и уничтожаются, но атомы, из которых они состоят
- вечны, ведь им не на что распадаться.

Он принимает тезис о том, что бытие - это нечто простое, понимая под ним
неделимое - атом.~Он дает материалистическую интерпретацию этой
концепции, считая атом наименьшей, больше не делимой физической
частицей.~Демокрит допускает бесконечное количество таких атомов, тем
самым отвергая утверждение, что бытие - это одно.~

На первый взгляд, учение об атомизме чрезвычайно просто.~Начало всего
сущего - неделимые частицы - атомы и пустота.~Ничто не возникает из
несуществующего и не аннигилирует в несуществующее, но возникновение
вещей - это комбинация атомов, а разрушение - это распад на части, в
пределе на атомы.~

\begin{quote}
Атомы и пустота - вот два вечных начала мироздания.
\end{quote}

Этим разным по размерам, но абсолютно плотным частицам присуще движение
(«они трясутся во всех направлениях», писал Демокрит), поэтому с
необходимостью в мире должна существовать пустота.

Что касается числа атомов в мире, Демокрит признает его бесконечным, они
отличаются друг от друга по форме, порядку, положению (Демокрит писал о
вогнутых, выпуклых, угловатых, шарообразных и других атомах).
~Следовательно, пустота также должна быть бесконечной, поскольку
конечное пространство не может вместить бесконечное количество атомов и
бесконечное количество миров, состоящих из них.~Трудно сказать, что
здесь оказывается первым предположением - бесконечность числа атомов или
бесконечность пустоты.~Оба основаны на аргументе, что и количество
атомов, и количество пустоты не более того.~Этот аргумент также
распространяется на число форм атомов.

Причем сами атомы не обладают качествами какого-то конкретного вещества
- качество вещи возникает лишь при восприятии сочетаний тех или иных
атомов.

\begin{quote}
Атомы вечны и неизменны, а вот вещи неустойчивы и преходящи.
\end{quote}

Но сами движения атомов не случайны, они подчиняются
причинно-следственной связи, объективным законам, необходимости.

\subsection{Необходимость и случайность}

Для Демокрита нет случайных событий. Случайным человеку кажется событие,
причину которого он не знает.

Пример Демокрита: по дороге идет человек, вдруг с неба на голову ему
падает черепаха и убивает его. Вот, казалось бы, пример случайного
события! Но Демокрит объяснял: нет, это событие не случайно. Орел,
схватив черепаху, обычно бросает ее на камень, чтобы расколоть панцирь и
полакомиться. Человек, шедший по дороге, был лыс, его голова напоминала
камень, потому орел бросил на него черепаху.

Правда, с современной точки зрения, Демокрит здесь смешивает два
различных понятия: причинность и необходимость. Конечно, существует
причина того, что черепаха упала на голову идущему человеку. С другой
стороны, то, что орел уронил черепаху на голову именно этому человеку.
случайность. Демокрит же, не делая различия между причинностью и
необходимостью, приходит к фатализму.

\emph{Фатализм (от латинского «fatalis» - роковой,
предопределенный судьбой)} ---
означает такой взгляд на мир, когда все происходящее рассматривается как
заранее предначертанное; все происходит так, как и должно произойти
согласно судьбе, року, мировому закону или какой-то высшей воле.


С одной стороны, атомисты отказались от поиска каких-либо божественных
причин происходящего, с другой -- считали, что в мире все предопределено
движениями атомов в пустоте.

\subsection{Происхождение мира}

Мир тоже возник в результате взаимодействия атомов. Пустота заполнена
атомами неравномерно. В тех частях пространства, где их больше, они чаще
сталкиваются друг с другом, и из такого движения атомов возникает вихрь.
В результате вихревого движения более крупные и тяжелые атомы собираются
в центре и образуют землю. Более легкие выталкиваются на периферию и
образуют там небо. Поскольку атомы земли продолжают двигаться (движение
атомов неуничтожимо!), то земля как будто бы утрамбовывается к центру и
выдавливает из себя воду. Вода, заполнив самые низкие места и впадины,
образует озера и моря.

Демокрит объяснял возникновение мира только физическими причинами, не
прибегая к действиям богов. Но самым удивительным является то, что, по
его мнению, миров бесконечно много, причем каждый из них имеет
шарообразную форму. В каждый момент времени какие-то миры возникают,
другие гибнут, а Вселенная - беспредельна.

Естественными причинами Демокрит объяснял и происхождение жизни на
земле. Только что образовавшаяся в результате атомных вихрей земля была
еще мягкой, «грязеобразной». На ней вздулись пузыри, как на лужах после
дождя. Эти пузыри нагревались солнцем и когда они лопнули, из них и
вышли первые животные и люди. Потом земля затвердела, а животные и люди
стали воспроизводить себя другим, привычным нам способом.

Кстати, Демокрит объяснял наличие полов тем, что в «допеченных» пузырях
вызрели самцы, в «недопеченных» -- самки.

Почему он разделял «допеченные» и «недопеченные» пузыри, как наличие
пола? Демокрит считал женщин глупыми и болтливыми существами, о развитии
которых и говорить не приходится. Поэтому и пузырь был «недопеченный»,
ведь ему не нужно было полностью нагреваться для того, чтобы стать
женщиной.

Таким образом, по мнению Демокрита, жизнь зародилась самопроизвольно, с
чем вряд ли согласятся наши современники. Но надо сказать, что всего два
века назад мысль о том, что в луже под действием тепла могут
самопроизвольно зародиться мелкие животные черви, насекомые, казалась
вполне научной, с ней -- никто не спорил.

\subsection{Что такое душа?}

Души людей тоже состоят из атомов, шарообразных по форме. Они более
подвижны, чем, скажем, угловатые, и не цепляются друг за друга. Атомы
души разлиты по всему телу человека и являются источником его
подвижности.

Интересно, что во время дыхания происходит обмен атомов нашей души с
окружающим миром: с выдохом мы выпускаем наружу некоторое количество
шарообразных атомов души, но, вдыхая затем воздух, мы большинство из них
втягиваем в себя вновь, а атомы воздуха, попадающие внутрь организма,
своим давлением мешают вылететь наружу остальным подвижным атомам души.

Когда же человек умирает, за выдохом перестает следовать вдох, подвижные
круглые атомы уже ничто не удерживает, - и душа вылетает наружу.

\emph{Получается, что душа смертна: после смерти тела все атомы
рассеиваются в окружающем мире.}

\subsection{Теория познания атомистов}

Но как мы можем знать, что все состоит из атомов и пустоты? Можем ли мы
увидеть атомы или пощупать их? Атомы не даны нам в ощущениях. Об их
существовании мы можем знать только благодаря усилиям разума.
Получается, что \emph{Демокрит четко различает две ступени познания:
чувственную и рациональную.}

\subsubsection{Чувственная ступень познания (сюда относится познание
посредством слуха, зрения. осязания, вкуса, обоняния)}

Человек познает лишь грубую видимость явлений, не понимая их причин.
Демокрит считал, что сладость свойственна предметам, атомы которых имеют
округлую форму, острый же вкус имеют вещества, атомы которых имеют
угловатую, заостренную форму и т.д.

Зрение же возможно потому, что от всех материальных тел происходит
истечение атомов (своеобразные копии этих тел). Истекающие от тел атомы
отпечатываются в нашем глазу, создавая образы предметов. Чувственная
ступень познания дает нам истинную информацию о мире, но мы не в
состоянии понять ее правильно, если будем руководствоваться только
ощущениями. По преданию, Демокрит ослепил себя, потому что «атомы
глазами не увидеть».

\subsubsection{Рациональная ступень познания (познание посредством ума)}

Только умом мы можем постичь существование атомов, ведь мы не в
состоянии их увидеть или ощутить. Разум идет дальше чувств, он способен
найти истинные причины происходящего, но без чувств у него не было бы
материала для последующего размышления.

Таким образом, различая две ступени познания, атомисты не
противопоставляли их, а рассматривали в единстве.

\subsection{Заключение}

Учение атомистов стало своеобразным завершением первого этапа
древнегреческой философии, для которого было характерно внимание не
только человеку, сколько природе и её объяснению. Их натурфилософия
носила материалистический характер.

Задача научного познания, согласно Демокриту, состоит в том, чтобы
свести наблюдаемые явления к сфере истинного существования и дать им
объяснение, основанное на общих принципах атомизма.~Этого можно достичь
за счет совместной деятельности ощущений и ума.~

\subsubsection{Черты научного метода Демокрита:}

\begin{enumerate}
\def\labelenumi{\arabic{enumi}.}

\item
  В познании исходить из единственного числа
\item
  Любой объект и явление можно разложить на простейшие элементы (анализ)
  и на их основе объяснить (синтез)
\item
  Различать существование по правде и по мнению
\item
  Явления реальности -- это отдельные фрагменты упорядоченного космоса,
  которые возникли и функционируют в результате действия чисто
  механической причинности
\end{enumerate}

\subsubsection{Итог}

Учение Демокрита было естественным результатом всего предшествующего
развития философской мысли. Согласно учению Демокрита, Вселенная,
состоящая из атомов вещества и пустоты, является движущейся материей.
Это так же реально, как и есть. Атомы, находящиеся в постоянном
движении, объединяются, чтобы образовать все. Когда атомы разделяются,
вещи умирают. Атомы - носители множественности, а пустота - воплощение
единства. Пустота безгранична, но атом имеет определенную форму и
размер. Атом не воспринимается чувственно и выглядит как пылинка,
плавающая в воздухе, только гораздо меньше. Число атомов и их
конфигураций бесконечно. Атомы находятся в постоянном движении, что
является основой формирования мира. Вечное движение состоит в постоянном
столкновении, отталкивании, сцеплении, расщеплении, падении атомов. Это
движение вызвано вихрем. Это движение вызвано начальным вихрем. Теория
атомного строения мира Демокрита оказалась наиболее убедительной по
своим принципам из созданных ранее. Согласно Демокриту, все во Вселенной
происходит по закону необходимости. Под этой концепцией он понимал
бесконечную цепочку причинно-следственных связей. А найти причины
различных явлений, по мнению Демокрита, и есть главная задача мудреца.

\section{Философия и диалектика Гераклита}

\emph{Диалектика (от греческого dialektike - искусство вести
спор)} ---
учение о всеобщей связи явлений и развитии, источником которого
признается наличие противоречий в окружающем человека мире и в его
сознании.


\subsection{Философия Гераклита}

Согласно диалектической точке зрения, в мире все развивается и не стоит
на месте. До нас дошло ставшее широко известным изречение Гераклита
Эфесского: \emph{``В одну и ту же реку нельзя войти дважды''}, где он
уподобил мир течению реки. По мнению мыслителя, все изменчиво и
переходяще. Почему? Да потому что мир, кажущийся на первый взгляд
устойчивым и однозначным, содержит в себе противоречия. Каждая вещь
обладает противоположными качествами: ``морская вода -- чистейшая и
грязнейшая: рыбам она питье и спасение, людям же -- гибель и отрава'',
``прекраснейшая обезьяна безобразна, если ее сравнить с человеком'' и
т.д., из чего следует вывод \textbf{об относительности всех свойств
вещей}. В мире нет никакого вечного и абсолютного качества, все
взаимосвязано и относительно. Аристотель приводил образное высказывание
Гераклита по этому поводу: ``Ослы солому предпочли бы золоту''. Значит,
ценность богатства и золота -- тоже относительна.

Одна противоположность обязательно предполагает другую: мы можем понять,
что такое добро, если столкнемся со злом, начинаем ценить красоту,
увидев безобразное и т. д. Ночь переходит в день, а день -- в ночь, за
зимой приходит лето, но за летом -- зима, война сменяется миром и
наоборот, после голода мы испытываем насыщение, но его вновь сменяет
чувство голода. Более того, Гераклит делает \emph{вывод о совпадении
противоположностей}: ``одно и то же в нас живое и мертвое, бодрствующее
и спящее, ведь это, изменившись, есть то, а то, изменившись, есть это''.
Такое совпадение (тождество) противоположностей означает их вечную
борьбу, которая является причиной возникновения всего нового в мире.

\begin{quote}
\emph{Борьба противоположностей --- «отец всего и царь над всем».}
\end{quote}

Такой образ мира, вечным законом которого является борьба
противоположностей, вовсе не означал для Гераклита хаоса и беспорядка.
Напротив, в мире есть гармония, но эта гармония скрыта и проявляется как
раз через борьбу. В конечном счете закон мира описывается им как
«логос\footnote{В философии, означающее одновременно «слово»
  (высказывание, речь) и «понятие» (суждение, смысл). Гераклит, впервые
  использовавший его в философском смысле и, по существу,
  отождествлявший его с огнём как основой всего (согласно Гераклиту,
  огонь является первоосновой мира (архэ) и его основным элементом или
  стойхейоном), называл «логосом» вечную и всеобщую необходимость. Хотя
  в последующем значение этого понятия неоднократно изменялось, тем не
  менее, \textbf{под логосом понимают наиболее глубинную, устойчивую и
  существенную структуру бытия, наиболее существенные закономерности
  мира.}}». Мы уже встречались с этим греческим словом, означающим
«слово», «понятие», «учение». Для Гераклита логос -- объективный закон
мира, принцип устройства космоса, это та сила, которая и определяет
движение космического огня. Один из сохранившихся фрагментов начинается
словами: «Не мне, но Логосу внимая \ldots{} ». То есть Гераклит пытался
приоткрыть людям закон развития мира, а отнюдь не свою личную точку
зрения. В мире все изменчиво, кроме логоса, вызывающего к жизни сами
изменения.

Но можно ли познать логос? Это непросто. Гераклит писал, что ``природа
любит скрываться'', поэтому требуется усилие для постижения ее законов.
Люди в своей жизни постоянно сталкиваются с проявлениями логоса, но не
понимают этого. Не все способны познать логос. Как же так? Ведь Гераклит
признавал, что все люди разумны: ``размышление всем свойственно, всем
людям дано познавать себя и размышлять''. Но размышляют-то люди
по-разному. Некоторые накапливают бесполезные знания, становятся
ходячими справочниками, но у них нет своих взглядов на окружающее, они
довольствуются чужими.

\begin{quote}
\emph{``Многознание не научает уму'', --- писал Гераклит.}
\end{quote}

Люди равны по своей природе, равны в том, что все они разумны . Но
используют люди свой разум в зависимости от интересов и целей, что
ставят перед собой. ``Плохие свидетели глаза и уши у людей, которые
имеют грубые души'' --- предупреждал Гераклит. Большинство людей
находятся во власти своих сиюминутных желаний,они хотят получать
удовольствия. Такие люди не слышат логоса; они, подобно ослам,
``предпочитают солому золоту''. Гераклит убежден, что такие люди не
стали бы счастливее, даже если бы все их желания исполнились, но сами
они этого не понимают, их жизнь похожа на «детские игры» . Лишь самые
достойные и мудрые способны жить в соответствии с логосом , а не со
своими преходящими животными желаниями: ``самые достойные предпочитают
одно: вечную славу смертным вещам. Толпа же набивает свое брюхо подобно
скоту''. Поэтому хотя в принципе все люди от рождения обладают
способностями познать логос, удается это единицам, которые действительно
используют полученные способности, а не растрачивают свою жизнь в
поисках мимолетных удовольствий.

\subsection{Основы диалектики в философии Гераклита}

В античности диалектика воспринималась в первую очередь как манера
ведения диалога, ораторское искусство, направленное на выявление
противоречий в позиции собеседника, и поиски истины через их обнажение.
Диалектический метод широко использовался Сократом в его знаменитой
майевтике.

Однако в учении Гераклита было сформулировано основное положение
современной диалектики, которые было развито спустя более чем две тысячи
лет в трудах Гегеля и Маркса. Это положение о единстве и противостоянии
противоположностей. Гераклит выделяет несколько видов противоположности:

\begin{itemize}

\item
  противоположности как следствия одного начала
\item
  противоположности как противоречивые свойства одного предмета
\item
  противоположности как антагонисты, необходимые для существования друг
  друга
\item
  противоположности как разные полюсы одного и того же континуума
\end{itemize}

В первом случае противоположностей Гераклит указывает на то, что один и
тот же предмет может одновременно производить два противоположных
эффекта, так морская вода хороша для рыб и плоха для людей. Этот род
противоположностей был прекрасно отражен в утверждении Парацельса о том,
что любое лекарство есть яд, разница лишь в дозе.

Ко второму случаю относятся предметы которые можно охарактеризовать
противоположными чертами, как например, металл который одновременно
твердый и гибкий. Противоположности антагонисты -- это веселья и уныние,
смелость и трусость, работа и отдых -- явления которые существуют
независимы друг от друга, но в тоже время невозможно познать одно из них
без познания другого.

И наконец, противоположность, как диалектическая противоречивость
предмета -- жар и холод есть проявления температуры, уродство и
великолепие есть проявление красоты и т.д.

Гераклит не только указывает на противоречивую природу мира, вещей и
процессов, но также раскрывает характер взаимоотношения
противоположностей, их постоянного преобразования одной в другую
(влажное высыхает, а сухое увлажняется), но также борьбы, которую
Гераклит превозносит как главный источник движения, развития и
изменений, саму основу существования мира.

\subsection{Заключение}

\subsubsection{Основные идеи Гераклита}

Гераклит Эфесский является продолжателем философии милетцев. Поиски
первоначальной субстанции привели его к убеждению, что первоначалом
может быть только огонь -- сосредоточение превращений, самое изменчивое
из первоначал, нечто,вечно находящееся в движении. В этих особых
качествах огня содержится предпосылка для развития диалектических идей
Гераклита --- \emph{о всеобщем развитии и изменении вещей, об их борьбе
и переходе в свои противоположности.} самые известные высказывания
Гераклита:

\begin{itemize}

\item
  Все течет, все изменяется.
\item
  В одну реку нельзя войти дважды.
\item
  Все происходит через борьбу: борьба - отец всего и царь над всем.
\end{itemize}

\subsubsection{Итог}

Положение о всеобщей изменчивости связывается Гераклитом с идеей
внутренней раздвоенности вещей и процессов на противоположные стороны.
``Всё едино и всё состоит из противоположностей''. Гармония тоже состоит
из противоположностей, представляя собой их единство. В изображении
Гераклита борьба, распря, война имеют глубинное отношение к рождению,
возникновению, расцвету, т.е. к самой жизни. ``Борьба всеобща и всё
рождается благодаря борьбе и по необходимости''. «Война» (так нередко
Гераклит называл борьбу) есть отец всего и всего царь. Или: ``Путь вверх
и путь вниз один и тот же''. В роде бы всё ясно, это один и тот же путь,
но он и в правда состоит из 2 противоположных путей.

Во всяком явлении он ищет противоположное ему, как бы рассекая всякое
целое на составляющие его противоположности. А за рассечением, анализом
следует (по главному правилу диалектики) синтез -- борьба, «война» как
источник и смысл любого процесса: ``Война есть отец всего и мать всего;
одним она определила быть богами, другим людьми; одних она сделала
рабами, других свободными''.

\section{Объективный идеализм Платона}

\subsection{Краткая биография}

Великий древнегреческий мыслитель Платон является основоположником
одного из двух главных направлений в философии -- идеализма\footnote{
  \emph{Идеализм} ---
  термин для обозначения широкого спектра философских концепций и
  мировоззрений, в основе которых лежит утверждение о первичности идеи
  по отношению к материи (см. Основной вопрос философии) в сфере бытия.
  Во многих историко-философских трудах проводится дихотомия, считающая
  противопоставление идеализма материализму как сущность философии.
  Категории материализма и идеализма во все эпохи являются историческими
  категориями. Применяя их, всегда нужно учитывать ту историческую их
  окраску и, в частности, ту эстетическую значимость, которую они
  получают в связи с разными периодами исторического развития, в связи с
  отдельными философами и культурологами и в связи с бесконечно
  разнообразной пестротой результатов и произведений философов и
  культурологов. Абстрактный идеализм в чистом виде и абстрактный
  материализм в чистом виде являются крайними противоположностями
  философского мировоззрения, не отвергающими, но предполагающими
  бесчисленное количество их совмещений с бесконечно разнообразной
  дозировкой. \emph{Идеальное} --- духовное, не воспринимаемое с помощью
  органов чувств.
  }. Платон родился в знатной семье, достаточно
богатого рода в 427 г. до н. э. на острове Эгина, входившем в Афинское
государство, а умер в 347 г. до н. э. в Афинах. Настоящее имя философа
-- Аристокл. Слово ``Платон'' в переводе с греческого означает
``широкий'', ``обширный''. В начале своего жизненного пути Платон был
учеником и последователем Сократа с 407 до 399 г. до н.э. Платон много
путешествовал по своей стране и другим странам. После смерти Сократа
более 10 лет Платон провел в путешествиях. Во время своих путешествий
Платон посещал Сицилию, где правил Дионисий старший. Платон поделился с
правителем своими представлениями по поводу социального и
государственного устройства, за что был отдан в рабство, откуда был
выкуплен своими друзьями. Много лет спустя уже на закате своих лет
Платон вновь посетил Сицилию, где правил Дионисий младший и снова
оказался на невольничьем рынке, откуда вновь его выкупили друзья.

Большинство произведений Платона сохранились до наших дней. Диалоги
Платона отличаются легкостью слога и прекрасным языком, даже в наши дни
мы восхищаемся культурой языка и мышления произведений Платона. Это
настоящий памятник литературы и философской мысли. Его произведения
написаны в форме монолога (\emph{``Апология Сократа''}), а остальные в
форме диалогов (более 30). К ним относится \emph{``Лахес'',
``Протагор'', ``Гиппий Меньший'', ``Горгий'', ``Кратил'', ``Гиппий
Больший''} (в котором впервые появляется термины \emph{эйдос} и
\emph{идея} ), начало \emph{``Государство'', ``Менон''. ``Пир'',
``Федр'', ``Федон''}. Более поздние диалоги -- \emph{``Софист'',
``Парменид'', ``Тимей''}. Самое позднее произведение называется
\emph{``Законы''}.

\subsection{Онтология}

Главной характеристикой философии Платона является идеалистическое
решение основного вопроса философии. Если с трудов софистов и Сократа
начинается, классический период античной философии, то в трудах Платона
и его ученика Аристотеля философская мысль Древней Греции достигает
расцвета. Как мы помним по философии Древнего Востока, идеалистическое
разрешение основного вопроса философии говорит о зрелости философской
мысли. Однако в отношении теории развития идеалистическая концепция
Платона носит метафизический характер. Платон относится к явлениям
природы, как к отдельным, изолированным друг от друга и неизменным
явлениям. Это метафизический аспект его философский концепции оказался
недостаточно перспективным.

Гносеологическую позицию Платона можно определить как рационализм. Идеи
невозможно постичь чувственным опытом, мир идей умопостигаем. Но если мы
не можем постичь идеи на уровне чувственного опыта, то как, в таком
случае, мы их осознаём? Дело в том, что идеи присутствуют в нашей душе
изначально. Платон создал миф, по которому человеческие души изначально
находились на небесах, где они передвигались по небосводу на колеснице,
запряжённой двумя лошадьми, одна из которых исполнена влечений к истине
и добродетели, а другая склонна к буйству, под её напором колесница
срывается с небосвода и падает вниз.

\subsection{Идеи Платона}

Платон воспринимал материальный окружающий нас мир как ``тень''.
Материальный мир производный от мира идей. Мир идей -- это объективно
существующая, независимая от материи идеалистическая реальность. Такая
позиция делает \textbf{философскую концепцию Платона объективным
идеализмом}.

Идея в философии Платона -- это подлинная сущность. Все предметы в
материальном мире, воспринимаемые человеком есть лишь тени идей
объективной идеалистической реальности. Таким образом, Платон
разграничивал Мир идей, кажущееся бытие и материю. Предметы в кажущемся
бытии (т.е. том бытии, которое воспринимает человек) - это соединение
идеи, существующей в идеальном мире и материи.

\emph{Основа философии Платона -- это учение об идеях. Он утверждал, что
идеи выступают не только основой человеческих качеств, но и основой всей
действительности. Идеи -- это подлинная реальность, прообразы вещей,
отношений ценностей. Они связаны между собой и неизменны. Каждому классу
чувственных предметов соответствует некоторая идея. По отношению к
чувственным вещам идеи одновременно и их причины и образцы, по которым
были созданы эти вещи.}

Но одного существования идей недостаточно для объяснения вещей
чувственно воспринимаемого мира. Так как эти вещи изменчивы, то они
должны быть обусловлены не только бытием, но и небытием. Это небытие
Платон отождествляет с материей, которая есть область непрекращающегося
движения, возникновения и изменения. Материя принимает на себя идеи и
превращает каждую идею во множество чувственных вещей, обособленных друг
от друга по месту, занимаемому ими в пространстве. В то время как
атомисты считали телесными атомы, а небытие приравнивали к пустоте, у
Платона небытие есть «материя», а бытие -- бестелесные идеи.

Платон выделял иерархию в мире идей. Главная идея -- это Бог. Эта идея
главенствует над всеми другими идеями. На следующей степени стоят идеи
блага, красоты и добра. Таким образом, идеи \emph{блага}, \emph{красоты}
и \emph{добра} -- это основные идеи, которые стоят на вершине иерархии
идеалистической философии Платона. Эти идеи не уничтожимы и неизменны.
Они существуют вне времени. Но полностью выстроить иерархию идей у
мыслителя не получилось, так как чувственное бытие слишком сложно, и
нелегко воссоздать все взаимоотношения вещей в некой иерархической
пирамиде.

\subsubsection{Разделение философских концепций по Платону}

Следовательно, истинный подлинный мир - это мир идей. Наш мир, доступный
для восприятия человека - это лишь кажущийся мир. Таким образом,
существует вечный конфликт между миром идей и материей, между бытием и
небытием. В царстве идей также существует конфликт бытия и не-бытия.
Таким образом, в идеальный мир идей Платон вводит диалектическое
развитие. Введение идеи Бога делает философию Платона
персонифицированным идеализмом. Платон первый разделил философские
концепции на материалистические и идеалистические.

Материалистические концепции объясняют первооснову мира через стихийные
начала -огонь, землю, воду, воздух. Идеалисты признают первичность идеи,
души и мышления по отношению к материальному миру и природе. Философия
Платона была не просто идеализмом, но воинствующим объективным
персонифицированным идеализмом. Платон считал, что не только
материалистические течения в философии неверными, но и что они подлежат
запрету, так как вводят людей в заблуждение. Самым большим невежеством
Платон считал отрицание существования Богов. Такие люди, по мнению
Платона должны быть заключены в тюрьму, а после смерти их тела должны
быть выброшены за пределы государства.

Платон говорил, что существует 3 степени атеизма:

\begin{itemize}

\item
  Непризнание существования богов
\item
  Признание невмешательства богов в человеческое существование
\item
  Убеждение в том, что путем религиозных манипуляций (жертвоприношений и
  пр.), человек может умилостивить богов
\end{itemize}

Таким образом, подлинное истинное бытие, по мнению Платона - это мир
идей. Это бытие постигается только рациональным способом. Идеальное
сущее состоит из множества идей - эйдосов ( (перевод. идея - внешний
вид, наружность, тип, вид, род, форма). Эйдосы составляют умопостигаемое
духовное множество. Эти идеи неизменны и вечны, неуничтожимы и
несотворимы.

\subsubsection{Государство Платона и классификация людей}

Платон утверждает, что людей можно разделить на три категории:

\begin{enumerate}
\def\labelenumi{\arabic{enumi}.}
\item
  \textbf{Люди, живущие своими потребностями, они должны работать.} У
  них вожделеющая (чувственная) душа. Эта категория включает
  земледельцев, ремесленников и торговцев, именно им дозволено иметь
  личную собственность. Однако от управления общественными делами они
  полностью отстраняются и получают минимальное образование.
\item
  \textbf{Люди со страстной (аффективной) душой.} Это класс стражей, они
  охраняют и защищают общество. Стражи нуждаются в продуманном
  образовании, которое включает гимнастику и музыку (поэзия, танец и
  музыка в современном значении этого слова). Таким путём стражи должны
  учиться мужеству, презрению к смерти и нравственному благородству. У
  стражей не должно быть никакой собственности, их личная жизнь
  находится под строгим контролем со стороны правителей.
\item
  \textbf{Люди с разумной душой. Они подчиняют всю свою жизнь созерцанию
  идей и постижению знаний.} Это слой правителей-мудрецов и философов.
  Образование этих правителей - философов является особенно длительным и
  основательным. Оно включает овладение геометрией, а также другими
  отраслями знаний, и в первую очередь, диалектикой.
\end{enumerate}

В платоновском государстве феномен частной жизни приносится в жертву
социальному целому. Не существует брака, все жёны и дети общие. Платон
своеобразно истолковал принцип социальной справедливости (каждый человек
делает то, к чему он наиболее приспособлен). Правителям, как и стражам
запрещалось иметь семью и личную собственность. Детей предполагалось с
трёх лет помещать в государственные воспитательные заведения.

Социальная утопия Платона кажется возвышенной и отвратительной
одновременно. Каждый человек в этом обществе значим лишь как
функциональный элемент социального целого. Человек как личность
растворяется в социуме.

\subsubsection{Объяснение происхождения мира}

Объясняя происхождение окружающего нас мира, Космоса, как упорядоченного
мироздания, Платон создаёт миф о Демиурге. Демиург -- это мастер,
творец. Для того, чтобы Космос получился совершенным, демиург устроил ум
в душе, а душу в теле, и таким образом построил Вселенную. Мировое тело
было построено с соблюдением строгих пропорций между четырьмя элементами
-- землёй, водой, воздухом и огнём. К нему была присоединена мировая
душа, и результатом этого явилось небо, включающее звёзды, планеты и
солнце.

Далее, в Космосе были созданы живые существа, в том числе и человек.
Объяснение природной действительности -- не самая сильная сторона
философии Платона. Он не смог убедительно объяснить происхождение
материального мира, не объяснил существование идей зла, лжи,
преступления, уродства. Высшей идеей он провозглашает идею добра. Это
позволяет заключить, что мир устроен наилучшим образом. Как мы видим,
Платон столкнулся с очень сложными вопросами.

\subsection{Основы идеалистической философии Платона}

\textbf{Гносеологическая.} Платон разделил понятие и мышление человека,
которые вырабатывает определения и понятия. Эти понятия и определения
Платон назвал -- эйдосами, которые, по его мнению, составляют идеальное
бытие. Платон был последователем и учеником Сократа. Сократ говорил, что
знать - значит иметь определение. В философии Платона носителями
определений и понятий являются идеи.

\textbf{Онтологическая.} Окружающий мир, который человек воспринимает
своими органами чувств Платон называет кажущимся бытием. В этом
кажущемся бытии одни вещи рождаются другими. Платон постулировал идею о
том, что предметы кажущегося материального мира порождаются в
соответствии с идеей идеального мира -- эйдосом\footnote{Слово «эйдос» в
  буквальном переводе с древнегреческого означает «форма». Когда мы
  посмотрим на солнце, а потом закрываем глаза, на сетчатке какое-то
  время еще остается образ солнечного диска. Платоновские эйдосы --- это
  такие образы, которые мы находим в своей душе и сознании, через
  которые воспринимаем и постигаем окружающий мир.}. Таким образом,
материальные вещи есть слепок с мира идей.

Так как существует пирамида идей, то существуют взаимоотношения между
ними. Это есть отношения соподчинения. Идеи соотносятся друг с другом на
основе понятий более общего и более частного. Одна идея более
конкретная, другая более общая, следовательно, первая будет подчиняться
второй. Однако, идеи -- это лишь образцы, по которым строитcя
материальный мир. Для того, чтобы было возможно чувственное
воспринимаемое бытие Платон вводит понятие хора. Хора в переводе с
греческого - материя, материальная субстанция. Материя и идея -- это два
рода сущего. Материя есть потенциальный источник многообразия
чувственного кажущегося бытия. Эйдос -- это пассивное начало мироздания.
Хора обладает пластикой и динамикой, она может принять любую форму. Так,
изначально существует идея стола, например, потом эта идея соединяется с
материей - хорой, которая принимает соответствующую форму стола. Однако,
важно отметить, что хора - это только материальная субстанция, источник
физической материи. Сама физическая материя состоит из 4 основных
стихийных первоэлементов, хорошо известных философии древней Греции -
огонь, воздух, земля, вода. Однако, и эти первоэлементы существуют
только потому, что в идеальном мире есть эйдосы, воплощающие идею воды,
огня, воздуха, земли. Физическая материя может изменяться, уничтожаться,
пропадать и возникать. Мир эйдосов неизменен и вечен.

Философской иллюстрацией концепции Платона является миф о Пещере,
который Платон использовал для обоснования своего учения. Платон
говорил, что материалисты, утверждающие, что в основе нашего мира лежит
материя и есть лишь чувственно-воспринимаемые предметы, подобны
человеку, сидящему в пещере. Этот человек сидит спиной к выходу пещеры,
в который проходит слабый свет. На стене пещеры человек видит, как
проходят тени. Материалисты верят, что тени на стенах пещер - есть
подлинное бытие. Однако, эти тени лишь отбрасываются подлинным бытием,
которое находится вне стен этой пещеры. Таким образом, сама пещера есть
кажущееся бытие, тени на стенах пещеры -- это образцы идей, а вот то,
что отбрасывает эти тени и есть сами эйдосы. Платон сравнивает
большинство людей с узниками пещеры, которые верят в чувственное
материальное бытие и не хотят выбраться из пещеры, чтобы познать эйдосы.
Философов и мудрецов Платон сравнивает со свободными разумными людьми,
которые выбрались из пещеры и узрели подлинный идеальный мир идей.

Миф о пещере Платон приводит для психологического доказательства своей
концепции.

Таким образом, Платон сравнивает мир чувственно-воспринимаемых идей с
миром теней, а идеи -- отбрасывают эти тени и являются образцами вещей.

Платон выделяет несколько родов сущего. Самым высшим является Бог, далее
идет мировая душа, затем идеи, и только потом хора и материальные вещи,
которая она образует. Платон вводит понятие мировой души для того, чтобы
обосновать движение. Ведь хора и эйдосы -- пассивные начала,
следовательно, необходим источник движения. Этим источником движения и
активности в космосе в философии Платона стала мировая душа.

\subsubsection{Функции мировой души}

\begin{itemize}

\item
  источник движения
\item
  источник одушевленности тел
\item
  источник сознания и познания
\item
  При этом мировая душа причастна как к миру эйдосов, так и миру
  материальных вещей. Для того, чтобы это было возможно, мировая душа
  состоит из противоположностей -- тождественного (воплощение идей) и
  иного (воплощение хоры). Мировая душа есть причина как внешнего
  движения, так и внутреннего движения. Следовательно, иное познает
  вещи, а тождественное познает идеи.
\end{itemize}

Высшим родом сущего является Бог. Бог в философии Платона - это и творец
хоры и творец эйдосов. Платон вводит понятие Демиург (перевод. --
устроитель, ремесленник, архитектор мира). Демиург создатель мироздания.
Платон говорит, что было время, когда не существовало космоса и даже
самого времени. Демиург сотворил и космос, и время, а также других
богов.

Следовательно, каждый род сущего выполняет определенную роль в
мироздании. Так Бог был создателем и материи, идей, и мировой души. Идеи
являются прообразами вещей, мировая душа выполняет роль активного
начала, которая приводит в движение хору и эйдосы, хора есть источник
многообразия материального мира, наконец, чувственно воспринимаемы вещи
и составляют окружающий мир.

\subsubsection{Гносеология}

Платон отрывает чувственное познание от рационального, как это делали
еще представители элиты. С помощью чувственного познания можно узнать
кажущееся бытие, т.е. наш материальный мир. Истинное бытие - мир идей -
не поддается чувственному познанию.

Платон разработал теорию воспоминания (анамнезиса), что является
наиболее интересной идеей в его гносеологии. Как представитель идеализма
Платон признавал наличие бессмертной души в человеке. Платон считал, что
до того как воплотиться в человеческом теле, человеческая душа обитала в
мире идей. Таким образом, душа воспринимала идеи, когда она еще не была
соединена с человеческим телом. Когда душа попадает в тело, то она
забывает то, что она видела в мире идей. Но у души есть способность
припоминания. Воспоминания души тем сильнее, чем больше душа
отстраняется от тела. Таким образом, вершина познания есть полное
припоминание идеального мира идей. Платон считал, что вещи могут
способствовать припоминанию, так как они есть слепок с идей, но основное
средство познания -- это диалектическое припоминание. Таким образом
Платон постулирует идею врожденности знания.

Платон выделил 3 ступени знания:

\begin{enumerate}
\def\labelenumi{\arabic{enumi}.}

\item
  \textbf{Абсолютно достоверное знание} -- это знание, которое душа
  получила, когда была в мире идей, до воплощения в теле
\item
  \textbf{Менее достоверное знание} -- это научное знание, получено за
  счет научного обоснования и доказательств
\item
  \textbf{Физическое знание} -- чувственное познание
\end{enumerate}

\subsection{Историческое значение философии Платона}

\begin{enumerate}
\def\labelenumi{\arabic{enumi}.}
\item
  Впервые философом оставлено целое собрание фундамен­тальных
  произведений;
\item
  Положено начало идеализму как крупному философскому направлению (так
  называемая «линия Платона» --- противоположность материалистической
  «линии Демокрита»);
\item
  Впервые глубоко исследованы проблемы не только природы, но и общества
  --- государство, законы и т. д.;
\item
  Была создана философская школа (Академия), просуществовавшая около
  1000 лет, где выросли многие видные последователи Платона (Аристотель
  и др.). Академия была закрыта в 529 г. византийским императором
  Юстинианом как рассадник язычества и «вредных» идей, однако за свою
  историю успела добиться того, что платонизм и неоплатонизм стали
  ведущими направлениями европейской философии.
\end{enumerate}

\subsubsection{Заключение}

Таким образом, учение Платона -- это объективный идеализм, так как
материя рассматривается как производное от нематериальных идей. Идеи
вечны, они не зависят от условий пространства и времени. Напротив, мир
чувственных вещей -- это мир вечного возникновения и гибели. Все вещи
существуют только потому, что существуют идеи этих вещей. Вещи сами по
себе несовершенны, они портятся и меняются. Идея же характеризуется
абсолютным совершенством.

Платон отрывает чувственное познание от рационального, как это делали
еще представители элиты. С помощью чувственного познания можно узнать
кажущееся бытие, т.е. наш материальный мир. Истинное бытие - мир идей -
не поддается чувственному познанию.

\emph{Истинное познание в философии Платона - это познание
рациональное.}

\section{Кратко о философии Платона}

\subsection{Основные идеи}

\subsubsection{Мир вещей иллюзорен}

Платон находился под сильным влиянием древнегреческого философа
Парменида. У него он позаимствовал идею о том, что видимый нами мир
вещей в каком-то смысле иллюзорен, и попытался ее развить. Платон не
отрицал в прямом смысле материальные вещи. Он хотел показать, что они
существуют для наших чувств, но их нет для нашего разума.

\subsubsection{Пути к реальности}

Деление мира на иллюзорный и подлинный связано с представлением о двух
видах доступа к реальности --- через чувства и через разум. Платон
позаимствовал у Парменида концепцию о том, что разум открывает нам
подлинный путь --- путь к настоящей истине. Чувства тоже открывают путь
к реальности, но к иллюзорной, неполной. Задача философа --- вовремя
разобраться, что привычный мир похож на пену, он не такой надежный и
устойчивый, как может показаться, а по ту сторону мира вещей есть
другая, более глубокая реальность, до которой можно добраться при помощи
разума.

\subsubsection{Пещера Платона}

Рассуждая о подлинной реальности и иллюзорной, Платон приводит аллегорию
пещеры. Предлагаем пересмотреть видео о ней. Платон представляет пещеру,
в которой заключены уники. Все, что они могут видеть --- это свет от
огня и тени --- фигуры людей, которые переносят предметы. У узников даже
не может возникнуть сомнение в том, является ли видимая ими
действительность настоящей, они не могут предположить иллюзорность этой
реальности.

Жизнь в пещере -- рабское положение. За много веков до Платона в Древней
Греции действительно существовала такая практика --- рабы, захваченные
во время войны, жили в ямах похожих на землянки. Рисуя образ пещеры с
узниками, Платон невольно напоминает, что состояние, когда человек живет
будто в пещере, считает эту жизнь настоящей и доволен ей --- рабское
положение. Настоящий философ должен освободиться от этого.

\subsection{Теория эйдосов}

Слово «эйдос» в буквальном переводе с древнегреческого означает «форма».
Когда мы посмотрим на солнце, а потом закрываем глаза, на сетчатке
какое-то время еще остается образ солнечного диска. Платоновские эйдосы
--- это такие образы, которые мы находим в своей душе и сознании, через
которые воспринимаем и постигаем окружающий мир.

\subsubsection{Функции эйдоса}

\paragraph{Познавательная}

Эйдосы -- это своего рода смыслы, в них выражается сущность вещей. При
помощи эйдосов мы получаем доступ сквозь поверхностный слой явлений
вглубь, к сути вещей.

\paragraph{Онтологическая, бытийная}

Эйдосы придают вещам бытие. Эйдос --- это внутренний принцип
существования вещи, тот стержень, на котором держится бытие вещи. Вещь
существует в той степени, в которой она соответствует своему эйдосу.
Если мы представим, что у вещи нет никакого эйдоса, то это будет
означать, что она по-настоящему и не существует.

\paragraph{«Царство идей» и «царство вещей»}

Часто говорят, что Платон делит реальность на два мира. Это верно лишь
отчасти. Если рассмотреть эйдосы сами по себе, в отрыве от вещей, то они
действительно будут похожи на самодостаточную, вечную, неизменную,
божественную реальность, к которой применимо название «царство идей» --
особая область бытия. Тогда известный нам мир вещей --- это другое
«царство», которое расположено ниже. Еще ниже располагается сама по себе
материя.

\paragraph{Материя}

Если идеи обладают бытием в максимальной степени, а вещи частично, то
сама материя -- это небытие, ничто. Платон описывает материю очень
презрительно, он называет ее «меон», что в буквальном переводе означает
«не-сущее», то есть несуществующее, чистая материя -- это просто
небытие.

\subsubsection{Структура мира}

Получается, что мир делится на три уровня:

\begin{enumerate}
\def\labelenumi{\arabic{enumi}.}

\item
  Подлинное бытие -- «царство идей»
\item
  Иллюзорное полубытие -- «царство вещей»
\item
  Небытие -- это материя
\end{enumerate}

Эйдосы, вещи и материя едины. Это деление условно, на самом деле эйдосы,
вещи и материя у Платона абсолютно неразрывны, они связаны сложными и
напряженными отношениями. Главное отношение, которым они связаны ---
отношение стремления-становления. Вещи не просто подражают своим идеям,
но стремятся к ним. Идеи-эйдосы как магниты притягивают вещи к себе.
Наличие «царства идей» заставляет весь вещественный мир находиться в
устремлении вверх. Хотя сами вещи этого не осознают, весь материальный
мир и процессы в нем устремлены к вечности. Материальный мир буквально
любит эйдосы, эта любовь неосознаваемая, но разлитая по всему
мирозданию.

\emph{Противопоставлять «царство идей» и «царство вещей» нельзя, они
неразрывно связаны, одно без другого лишается смысла.}

\subsection{Космология}

\emph{Космология --}
это учение об устройстве мира и его происхождении, которое Платон
излагает в трактате «Тимей». Рассуждая об устройстве мироздания, Платон
переосмысляет Пифагора.


\subsubsection{Основа мира}

У Пифагора в основе всего лежат числа, у Платона -- эйдосы. Платон
пытается показать, что в основе мироздания лежат неделимые, вечные,
неуничтожимые сущности --- эйдосы. Но как эйдосы воплощаются в материю?
Здесь Платон осознанно заимствует главную концепцию Пифагора --- это
пять правильных многогранников, которые открыл Пифагор: тетраэдр, куб,
октаэдр, икосаэдр, додекаэдр. Правильные многогранники соответствуют
стихиям --- огню, земле, воздуху, воде и эфиру. Эйдосы образуют
правильные многогранники, а из них, как из атомов, складываются все
вещи. Такая концепция похожа на теорию атомов, только «атомы» у Платона
являются не частицами материи, а элементарными абстрактными
формами-эйдосами.

\subsubsection{Движущая сила мира}

Какова движущая сила, соединяющая эйдосы-идеи с материей? Какая сила
заставила эйдосы воплотиться в материю и породить видимый мир вещей?
Здесь впервые в системе Платона появляется бог. Платон называет этого
своего бога термином «демиург», что буквально означает
создатель-ремесленник, мастер. Демиург создает мир не из ничего, как
христианский бог, а из уже подготовленной материи, он просто придает ей
форму.

\subsubsection{Воплощение вещей в этот мир}

Работа демиурга механическая. Можно представить, что в одной руке он
держит подготовленную материю, геометрически расчерченную благодаря
четырем первоэлементам, а с другой -- смотрит на эйдосы как на чертежи и
воплощает эти идеальные формы в материальных объектах. Это механическая,
а не творческая работа. Еще до появления христианства и в первые века
христианства многие восточные мистические учения, например, гностики
склонялись к платоновскому представлению о богах.

\section{Философия Аристотеля}

Аристотель -- философ Древней Греции, живший в 384 г. до н. э.- 322 г.
до н. э. Ученик выдающегося мыслителя того времени, Платона. Аристотель
известен тем, что был наставником Александра Македонского. Знания,
переданные Аристотелем Александру, были для полководца путеводной
звездой всей его жизни. Философия Аристотеля достойна пристального
внимания. Она и до сих пор несет в себе пользу и ценные знания.

\subsection{Основы философии Аристотеля}

Аристотеля интересовали как основы мироустройства, так и вопросы
сущности человеческой личности. Эти исследования он отражал в своих
работах, дошедших до наших дней. Мыслитель много трудов посвятил
искусству риторики -- обучал красноречию.

Вплотную Аристотель начал изучать философию еще в 17 лет. В этом
возрасте он поступил в Академию Платона, где обучался 20 лет.
Впоследствии, основал собственную философскую школу в городе Пеле,
которая получила название «Ликей» (прототип современного лицея), где
преподавал до конца жизни.

Аристотель -- автор работ, которые легли в основу современной философии.
И самые известные из них --- «Риторика», «Метафизика», «Политика»,
«Поэтика», «Органон».

\subsection{Составляющие философии Аристотеля}

Учение философа делится на 4 части:

\begin{itemize}

\item
  \textbf{Теорию} -- изучение проблем бытия и его граней, происхождения
  и сущности явлений
\item
  \textbf{Практику} -- модель государственного устройства и деятельность
  людей
\item
  \textbf{Поэтику} -- изучение средств художественного выражения в
  литературе
\item
  \textbf{Логику} -- науку об истинном представлении окружающей
  действительности
\end{itemize}

В вопросах сущности бытия, Аристотель критиковал труды своего учителя,
Платона. Он был противником однозначных теорий о мироустройстве, и
считал, что каждая идея будет зависеть от обстановки в окружающем мире,
а каждая вещь уникальна. Подробно остановимся на этих моментах.

\subsection{Понятие метафизики}

Суть метафизики Аристотеля -- критика трудов Платона и его концепции о
разделении мира идей и мира вещей. Ученый считает, что форма и материя
неотделимы друг от друга. В материи заложено стремление воплотить в
жизни те возможности, которые она заключает в себе.

Рассматривая проблему бытия, Аристотель выступил с критикой философии
Платона, согласно которой, окружающий мир делился на «мир вещей» и «мир
чистых (бестелесных идей), и «мир вещей» в целом, как и каждая вещь в
отдельности, являлся лишь материальным отображением соответствующей
«чистой идеи».

Ошибка Платона, по Аристотелю, в том что он оторвал «мир идей» от
реального мира и рассматривал «чистые идеи» вне всякой связи с
окружающей действительностью, которая имеет и свои собственные
характеристики -- протяженность, покой, движение и др.

\emph{Не существует «чистых идей», не связанных с окружающей
действительностью, отображением которых являются все вещи и предметы
материального мира.}

\emph{Существуют только единичные и конкретно определенные вещи.}

\emph{Данные вещи называются индивидуумы (в переводе -- «неделимые»), то
есть существует только конкретная лошадь в конкретном месте, а не «идея
лошади», воплощением которой данная лошадь является, конкретный стул,
находящийся в конкретном месте и имеющий свои признаки, а не «идея
стула», конкретный дом, имеющий точно определенные параметры, а не «идея
дома», и т.д.}

\emph{Индивидуумы являются первичной сущностью, а виды и роды
индивидуумов (кони вообще, дома вообще и т.д.) -- вторичной}

Понятие «формы» по Аристотелю включает в себя три момента: сущность
предмета «в настоящем времени», и потенциально возможные вещи, которые
могут получиться из неё после -- результат определенного создавшего её
акта творчества.

Переход потенциальной возможности в существующую действительность --
движение. В процессе движения, простые вещи превращаются во все более и
более сложные. Постепенно, они приближаются к совершенству и к своему
первоисточнику -- Богу. Согласно этой концепции, Бог -- это чистое
мышление, которое не имеет выражения в вещественной форме. В дальнейшем,
мышление просто не может развиваться -- оно достигло совершенства, но
Бог не существует отдельно от материального мира.

\subsection{Аристотель о физике}

По мнению ученого, материя возникает, исчезает и изменяется по законам
движения, которое представляет собой бессмертную жизнь природы во
времени и пространстве. Целью движения является постепенное расширение
границ влияния формы над материей, и совершенствование жизни.

Ученый выделяет 4 основных вещества, из которых состоит Вселенная --
огонь, воздух, вода и земля.

Философия Аристотеля четко разграничивает направления движения: вверх (к
границе мира) и вниз (к центру Вселенной). Обусловлено это тем, что одни
предметы (вода, земля) имеют тяжелый вес, а другие (огонь и воздух)
легкие; из этого следует, что каждая из стихий двигается по-своему:
воздух и огонь стремятся вверх,а вода и земля -- вниз.

Вселенная, согласно философской мысли, имеет форму шара. Внутри нее по
четко обозначенным окружностям движутся небесные тела, которые тоже
имеют шарообразную форму. Граница Вселенной -- это небо, которое
представляет собой живое существо, и состоит из эфира.

\subsection{Что такое душа}

Аристотель считал, что каждый живой организм имеет нечто, руководящее им
-- душу. Они есть не только у людей, но и у растений, животных. Это то,
что отличает живое от мертвого.

Согласно трактатам мыслителя, душа и тело не существуют друг без друга,
поэтому, невозможно изучать одно и другое отдельно.

Мыслитель отличает души растений и животных от человеческой. Последняя
-- частица божественного разума, имеет более возвышенные функции, чем
ответственность за пищеварение, размножение, передвижение и ощущения.

\subsection{Философ о природе}

Аристотель в трудах говорил о том, что материя всегда будет стремиться к
более совершенному состоянию. Так, предметы неорганического мира
постепенно становятся органическими; растения в процессе эволюции
преобразуются в предметы животного царства. Все в природе представляет
собой частицы единого целого.

Постепенно, жизнь организмов становится все ярче и ярче, и достигает
своего пика, воплотившись в человеке.

\subsection{Аристотель об этике}

Древнегреческий философ говорил о том, что суть добродетели состоит не в
знании того, что есть добро и зло, потому что, наличие знания не
способно удержать человека от совершения дурных поступков. Нужно
сознательно тренировать в себе волю к совершению добрых поступков.

Добро -- это преобладание разума над человеческими желаниями и
страстями. Поведение человека можно назвать этическим, лишь тогда, когда
он находит компромисс между своими желаниями и тем, как нужно поступить,
согласно морально-этическим нормам. Не всегда человек хочет поступить
правильно. Но усилием воли он должен контролировать свои действия.
Поступив нравственно и справедливо, мы испытываем чувство довольства
собой.

Нравственность неразрывно должна быть связана с государственностью и
политикой.

\subsection{Аристотель о политике}

Высочайшей целью нравственной деятельности человека является создание
государства. Согласно этой идее, ячейкой общества и государственности
является отдельная семья. Супруги состоят между собой в союзе, который
основан на нравственности. Руководит им мужчина, но женщина в семье тоже
имеет свободу в своих действиях. Мужчина должен в большей степени иметь
власть над детьми, чем над своей супругой.

Согласно Аристотелю, рабство -- это нормальное явление. Каждый грек
может иметь рабов из варварских племен. Ведь они -- существа высшей
природы. Рабы находятся в полном подчинении своего господина.

Несколько семей образуют общину. А когда общины соединяются между собой
--- появляется государство. Оно должно обеспечивать счастливую жизнь для
каждого, стремиться делать граждан добродетельными. Государство должно
стремиться к совершенному устройству жизни.

В своем трактате «Политика» ученый приводит несколько разновидностей
форм государственного правления: монархия (государством правит одно
лицо), аристократия (правят несколько человек) и демократия (источником
власти является народ).

\subsection{«Поэтика» Аристотеля}

Многогранный Аристотель изучал также искусство драмы. Он написал
отдельный трактат, посвященный этой отрасли --- «Поэтика», который не
дошел до нас целиком, но некоторые страницы этого труда, сохранились.
Поэтому мы знаем, что думал великий философ о драматическом искусстве.

Ученый считал, что суть трагедии -- пробуждать в зрителях сострадание и
ужас. Благодаря таким сильным впечатлениям, человек испытывает
«катарсис» -- происходит его духовное очищение.

В пьесах Древней Греции всегда рассматривался определенный период
времени. Философ в трактате «Поэтика» говорил о том, что время, место и
действия в сюжете не должны расходиться с друг другом (теория «трех
единств»).

Многие драматурги в своей работе опирались на учения Аристотеля. Позже,
в «Новое время» в Европе не всегда стали придерживаться теории «трех
единств», но она стала основой классического стиля в искусстве.

\subsection{Природа и душа}

Все живое на земле имеет свою душу, а то, что не имеет, стремится ее
приобрести. Философия Аристотеля кратко и понятно показывает все
разнообразие бытия на нашей планете. Он выделял 3 вида души.
Растительная -- низшая ступень, ее цель только питание. Животная --
чувствующая душа, животные способны чувствовать и отвечать на внешний
мир. Человеческая -- высшая форма души, возможная на земле. Душа не
может существовать без своего материального тела.

Исходя из идеи о развитии, весь природной мир тоже стремится перейти на
новый уровень. Неживая природа стремится перейти в растения, растения в
животных, животные в человека, человек в бога. Это развитие проявляется
в том, что жизнь становится все ярче и разнообразнее. Происходит
своеобразная эволюция души в стремлении к совершенству. Так, душа,
достигшая наивысшей точки, сливается с богом.

\subsection{Учение Аристотеля о бытии}

Основа всякого бытия, по Аристотелю, -- первая материя. Она образует
потенциальную предпосылку существования. Она является основой всякого
бытия, но ее нельзя отождествлять с бытием и даже нельзя считать
составной частью конкретного бытия. Наипростейшей определенностью этой
первой материи являются, по Аристотелю, четыре элемента -- огонь,
воздух, вода и земля -- промежуточная ступень между первой материей,
которая чувственно непостижима, и реально существующим миром, который
чувственно воспринимаем.

Самобытное единичное бытие, то есть такое бытие, которое не способно
пребывать в другом бытии и существует в самом себе, Аристотель называет
субстанцией. Мир -- совокупность субстанций. Каждое единичное бытие --
сочетание материи и формы. Материя -- это первичный материал. «Форма»
есть действительность того, возможностью чего является «материя», и
наоборот, «материя» есть возможность того, действительностью чего будет
форма.

Аристотель придерживается той точки зрения, что душа является формой по
отношению к материи, что она присуща всем объектам, принадлежащим к
живой природе, то есть растениям, животным и человеку. Душа -- это
проявление активной жизненной силы. В ряде своих работ он приходит даже
к таким выводам: «Деятельность души обусловлена состоянием тела», «Душа
не существует без материи».

При изучении конкретных вещей как реального бытия Аристотель говорит о
первых и вторых сущностях. Сущность -- это единое, обладающее
самостоятельностью бытие. Первые сущности состоят из материи и формы.
Они выступают чувственно познаваемым бытием. Вторые сущности --
производные от первых. Они являются видовым определением (стул, стол --
первичные; мебель -- вторичные).

\subsubsection{Четыре вида причин бытия :}

\begin{enumerate}
\def\labelenumi{\arabic{enumi}.}

\item
  Материальные -- то, из чего состоят вещи, их субстрат
\item
  Формальные -- в которых форма проявляет себя, образует сущность,
  субстанцию бытия
\item
  Действующие, или производящие, -- рассматривающие источник движения,
  энергетическая база формирования вещей
\item
  Целевые, или конечные, -- отвечающие на вопросы «Почему?» и «Для
  чего?»
\end{enumerate}

\subsection{Итог}

Аристотель утверждал, что философия возникает на основе эпистемы --
знания, выходящего за рамки чувств, навыков и опыта. Так что
эмпирические знания в области исчисления, здоровья человека,
естественных свойств объектов были не только зачатками наук, но и
теоретическими предпосылками возникновения философии. Аристотель выводит
философию из зачатков наук. Философия Аристотеля -- это не только некое
обобщение, но, можно сказать, логическая переработка и завершение всей
предыдущей греческой философии, это было одно из высших достижений
античной мысли и оказало значительное влияние как на последующую историю
философии. в античности (от эллинизма до неоплатонизма) и по философии
средневековья (аристотелизм в арабской и западной традициях).
Философское знание делит Аристотель на метафизику, логику, этику,
физику, историю, эстетику. Многочисленные работы Аристотеля охватывают
практически всю доступную тогда область знания, которая в его трудах
получила более глубокую философскую основу, была приведена в строгий
систематический порядок, а ее эмпирическая база значительно выросла.
Некоторые из этих работ не были опубликованы им при жизни, а многие
другие были ложно приписаны ему позже. Но даже некоторые отрывки из тех
произведений, которые несомненно принадлежат ему, могут быть подвергнуты
сомнению, и уже древние пытались объяснить себе эту неполноту и
фрагментарность превратностями судьбы рукописей Аристотеля.

Согласно легенде, сохраненной Страбоном и Плутархом, Аристотель завещал
свои сочинения Теофрасту, от которого они перешли Нелию Скепсисскому.
Наследники Нелия спрятали драгоценные рукописи от жадности пергамских
царей в подвале, где они сильно пострадали от сырости и плесени. В 1
веке до нашей эры. е. они были проданы за высокую цену богатому и
любителю книг Апелликону в самом плачевном состоянии, и он пытался
восстановить поврежденные места рукописей собственными дополнениями, но
не всегда успешно. Впоследствии, при Сулле, они оказались среди других
жертв в Риме, где Тиранний и Андроник Родосский опубликовали их в их
нынешнем виде. По мнению некоторых исследователей, эта история может
относиться только к очень небольшому количеству второстепенных работ
Аристотеля. К сожалению, до нас не дошли сочинения Аристотеля,
написанные в общедоступной (экзотерической) форме, например Диалоги,
хотя различие, принятое древними между экзотерическими и эзотерическими
сочинениями, не проводилось так строго самим Аристотелем и, во всяком
случае,, не означает различий по содержанию. Дошедшие до нас сочинения
Аристотеля далеко не идентичны по своим литературным достоинствам: в
одном и том же произведении одни разделы производят впечатление
тщательно обработанных и подготовленных к публикации, а другие
представляют собой более или менее подробные наброски. Наконец, есть те,
которые наводят на мысль, что это были всего лишь заметки учителя к
предстоящим лекциям, а некоторые отрывки, такие как, возможно, его Этика
Евдема, похоже, обязаны своим происхождением заметкам слушателей или, по
крайней мере, были пересмотрены. на этих заметках.

Вся жизнь Аристотеля состояла из бесконечного стремления найти,
проанализировать, постичь истину, докопаться до смысла окружающего его
мира. Аристотель был сильным человеком. Его поиски, вся его жизнь
свидетельствуют о небывалом мужестве великого человека, для которого
даже сама смерть стала актом мудрости и невозмутимого спокойствия.

\chapter{Средневековая философия}

Средневековая Европейская философия охватывает времянной отрезок
примерно в 1000 лет и делится на периоды:

\begin{enumerate}
\def\labelenumi{\arabic{enumi}.}

\item
  \textbf{Партистика} -- 2-8 века нашей эры. Представители: Августин
  Аврелий (354-430 г.)
\item
  \textbf{Схоластика} -- 8-15 века нашей эры. Представитель: Фома
  Аквинский (1226-1274 г.)
\end{enumerate}

Философская мысль эпохи феодализма теоцентрична\footnote{Теоцентризм --
  философская концепция, в основе которой лежит понимание Бога как
  высшего бытия, источника всей жизни и любого блага. При этом основой
  нравственности служит почитание и служение Богу, и подражание и
  уподобление ему считается высшей целью человеческой жизни. Теоцентризм
  связан с теизмом и его принципами. Теоцентризму противопоставляются
  космоцентризм и антропоцентризм.} (от лат. \emph{theos -- Бог}), как
как единственной реальностью, определяющей все сущее провозглашается
Бог, т.е. сверхестественное начало, а не космос и природа.

В борьбе с остатками языческого многобожия христианство нуждалось в
философском способе рассуждения и доказательства, в логике и
аргументации. И оно начинает усваивать элементы культуры, науки и
философии, подчинив их обоснованию догм\footnote{До́гма́т, или до́гма --
  утверждённое церковью положение вероучения, объявленное обязательной и
  неизменяемой истиной, не подлежащей критике (сомнению).} богословия.

Средневековая философия оказалась таким образом тестно связана
христианской теологией и религией: ``Философия --- служанка
богословия''*.

\section{Августин Аврель}

Философ Августин Аврелий (св. Августин Блаженный) -- наиболее известный
представитель патристики, автор многих работ, в том числе таких
известных, как «Исповедь» и «О граде божьем».

\emph{Августин Аврыий (354-430) родился в г. Тагасте в Северной Африке,
на территории Западной Римской империи. В молодости Августин (как он
пишет в своей «Исповеди») вел распутный образ жизни, однако со временем
начал задумываться о смысле своего существования. Некоторое время он
пытался найти ответ на свои духовные запросы в разных сектантских
учениях, но в конце концов пришел к христианству. Достаточно быстро
Августин стал признанным авторитетом в богословии. В концежизни стал
епископом г. Гиппона.}

В своих многочисленных произведениях Августин вьдвинул ряд идей, надолго
определивших пути развития западного богословия и Философии. Бог и
собственная душа -- вот две темы, которые интересовали его прежде всего.

Основные идеи Августина -- духовное развитие личности, цель развития
исторического процесса, проблема теодицеи\footnote{Теодице́я (новолат.
  theodicea «богооправдание» от др.-греч. θεός «бог, божество» + δίκη
  «право, справедливость») -- совокупность религиозно-философских
  доктрин, призванных оправдать управление Вселенной добрым Божеством,
  несмотря на наличие зла в мире: так называемая проблема зла. Термин
  введён Лейбницем в 1710 году.}.

\subsection{Основные идеи Августина}

«Исповедь» -- первое известное произведение, где в основу сюжета
положена не череда внешних событий, а развитие души, более драматичное и
важное, чем все события, проходящие во внешнем мире. В «Исповеди» с ее
тонким психологизмом рождается понятие личности, которого не знала
античность. Для развития личности необходима вера, направляющая духовные
силы человека к пониманию смысла жизни.

В работе «О граде божьем» Августин противопоставляет «градземной» ---
государство и «град божий» --- церковь. Первый воплощает любовь к себе,
второй -- к богу. В борьбе «градов» должна победить церковь и вера,
которая подчинит себе государство: в этом состоит миссия церкви и цель
истории.

Если в мире присутствует зло, а мир сотворен богом, то получается: бог
виноват в том, что зло существует. Однако Библия говорит, что бог
«всеблаг», добр. Оправдание бога за зло называется проблемой теодицеи
(от греч. theos -- бог + dike -- справедливость).

Августин сравнивает добро и зло со светом и тьмой. Свет реально
существует (например, можно измерить его скорость и определить, из чего
он состоит). Тьму нельзя ни измерить, ни обнаружить ее составляющие:
тьмой мы называем просто отсутствие света. Зло есть отсутствие добра,
небытие, значит, оно никем не сотворено. Причина зла -- не в боге, а в
свободе человека, который может идти к богу, а значит -- к добру, или же
не идти к нему, тем самым умножая зло.

\begin{quote}
\emph{Августин полагал, чго знание само по себе неспособно привести к
истине. Ему должна помочь вера. Самое известное высказывание Августина:
«Верую, чтобы знать».}
\end{quote}

\section{Фома Аквинский}

Философ и богослов Фома Аквинский (Аквинат) -- наиболее известный
представитель схоластики. Он использовал труды Аристотеля для того,
чтобы систематизировать богословские идеи. Основные его работы -- «Сумма
теологии» и «Сумма против язычников» (или «Сумма философии»).

\emph{Фома Аквинский (1225/26-1274) родился в Акуино, близ Неаполя в
богатой и влиятельной аристократической семье. Вопреки желанию
родственников Фома вступает в нищенствующий орден доминиканцев. Пытаясь
образумить Фому, родители заточили его в башне фамильного замка, где тот
провел больше года, но не отказался от своего выбора. Фома учится и
работает в Париже, Кёльне и Риме, где пишет ряд трактатов и коммента­
риев к Библии и трудам Аристотеля. В 1274 г. по пути на собор он
умирает. В 1323 г. Фому причислили клику святых.}

В фокусе исследований Аквината -- проблема соотношения веры и разума. По
мнению философа, и вера, и разум ведут к истинному знанию, однако в
случае противоречия между ними следует отдавать предпочтение вере:
\emph{``Философия -- служанка теологии''}. Так, при помощи разума можно
доказать то, что богсуществует и он един. Что касается троичности бога,
первородного греха и других идей, то их нельзя понять без откровения
(Библии) и веры.

\subsection{Пять доказательств бога}

Фоме принадлежат пять доказательств существования бога:

\begin{enumerate}
\def\labelenumi{\arabic{enumi}.}
\item
  
  \textbf{Движение --}
  \emph{все, что движется, имеет источник движения в чем-то другом,
  следовательно, должен быть перводвигатель, т.е. бог.}
  
\end{enumerate}

Не подлежит сомнению и подтверждается чувствами, что в этом мире нечто
движется. Но все, что движется, имеет источник движения. Следовательно,
должеy быть перводвигатель, так как не может быть бесконечной цепи
движущих предметов. А перводвигатель -- это Бог. (Это доказательство
Аквината прямо опирается на философию Аристотеля и его учение о
перводвигателе.)

\begin{enumerate}
\def\labelenumi{\arabic{enumi}.}
\setcounter{enumi}{1}
\item
  
  \textbf{Причина --}
  \emph{все имеет причину, цепь причин не может уходить в бесконечность,
  следовательно, существует первопричина, т.е. бог.}
  
\end{enumerate}

Каждое явление имеет причину. Но у причины тоже есть причина и так
далее. Значит, должна быть верховная причина всех реальных явлений и
процессов, а это Бог. (И опять видна перекличка с аристотелевским
учением о «форме всех форм».)

\begin{enumerate}
\def\labelenumi{\arabic{enumi}.}
\setcounter{enumi}{2}
\item
  
  \textbf{Необходимость --}
  \emph{случайное зависит от необходимого, значит, существует высшая,
  божественная необходимость, т.е. бог.}
  
\end{enumerate}

Люди видят, что вещи возникают и гибнут. Рано или поздно они перестают
существовать. То, что стул, на котором вы сейчас сидите, существует --
случайность с точки зрения мироздания, его могло бы и не быть. Но если
все может быть, а может и не быть, то ко­ гда-нибудь в мире ничего не
будет. Если это так, то уже сейчас ничего не должно быть. Но так как мир
не исчез, значит, существующее случайно «Подпитывается» чем-то
необходимым. Такой абсолютно необходимой сущностью является Бог.

\begin{enumerate}
\def\labelenumi{\arabic{enumi}.}
\setcounter{enumi}{3}
\item
  
  \textbf{Качество --}
  \emph{все имеет разные степени качеств (хуже, лучше), значит, должен
  быть эталон -- высшее совершенство, т.е. бог.}
  
\end{enumerate}

Люди считают одни вещи лучше других. Эта девушка красивее своей соседки,
а этот юноша -- умнее приятеля. Но с чем мы сравниваем? Где «масштаб»
для сравнения? Мы должны чувствовать, что есть «пресовершенная красота,
совершенный ум и т. д. Чем ближе вещь к этому пределу, тем она кажется
нам лучше. Значит, есть то, что обладает этим предельным качеством, --
это Бог. (А это доказательство напоминает по логике рассуждения диалоги
Сократа.)

\begin{enumerate}
\def\labelenumi{\arabic{enumi}.}
\setcounter{enumi}{4}
\item
  
  \textbf{Цель --}
  \emph{все в мире имеет цель, следовательно, существует высшее разумное
  начало, которое и направляет все в мире к цели, т.е. бог.}
  
\end{enumerate}

Все предметы, лишенные разума, устроены целесообразно. Крьmья бабочки
пригодны для того , чтобы летать с цветка на цветок, а орла - чтобы
парить в вышине. Тигры полосатые, и их окраска помогает им скрываться в
джунглях. Почему все так устроено? Поскольку сами предметы лишены
разумения, постольку их должен направлять некто , одаренный разумом.
Значит, есть разумное существо, полагающее цель для всего, что
происходит в природе. Это Бог.

\subsection{Томизм}

Учение Фомы известно как томизм \emph{(по-латыни Фома -- Thomas)}. Идеи
Аквината популярны до сих пор, и современная католическая философия
известна как неотомизм.

Ватикана в Средние века, а сегодня католическая бьm официальной церковь
стоит на позициях неотомизма, то есть видоизмененно­ го, «Продолженного»
учения Фомы Аквинского. За свои заслуги перед церковью Фома Аквинат в
1323 г. бьш причислен к лику Доминиканцы и францисканцысвятых, а в 1567
г. бьm признан пятым великим учителем церк­ви. В 1879 г. его учение бьшо
объявлено «единственно истин­ной» философией католицизма.

\subsection{Итог}

Целью учения Аквината было показать, что разум и философия не
противоречат вере. Двигаясь к истине, разум может вступить в
противоречие с догматом веры. По мнению Фомы, в этом случае ошибается
разум, так как в божественном Оrкровении ошибок нет. Но философия и
религия имеют общие положения, поэтому лучше понимать и верить, чем
просто верить. Есть истины, которые недоступны рациональному познанию, а
есть истины, которые оно может постичь. Например, разум может
свидетельствовать, что Бог есть (и в своих сочинениях Фома Аквинский
пытался именно с помощью разумных, логических доводов доказать Его
существование).

\section{Основные принципы религиозно-философского мышления и
мировоззрения}

\begin{enumerate}
\def\labelenumi{\arabic{enumi}.}
\item
  \textbf{Супронатурализм} \emph{(super -- сверх + nature -- природа)}
  --- главенствующая идея здесь -- идея бога, придание
  сверхестественному роли определяющего начала во всех мировых процессах
  -- в природе, обществе и человеке. Если в античной философии природа
  самодостаточна и божествена, то в средневековой философии бог
  первичен, он вне природы, которая им сотворена. \emph{Истинное бытие
  -- бог, остальное -- его творение.} Такое мировоззрение называется
  креационизмом.
\item
  \textbf{Ориентация всей жизнедеятельности человека на спасение души.}
  При этом спасение понимается, как процесс обожествления или
  соеденинения человека с богом в божьем царстве. Именно в этом
  спасении, а не в познании, общении и деятельности по преобразованию
  мира смысл человеческого бытия.
\item
  \textbf{Принцип богооткровенности} -- познание признается как процесс
  передачи богом людям своей истины через пророков и апостолов
  священного писания. Знания, это не результат интеллектуальных усилий
  человека, а полученное из вне, в готовом виде вечная и неизменная
  истина, которую люди должны принять в силу авторитета того, от кого
  она получена -- от бога или церкви. Это ведет к авторитарному
  догматическому типу мышления, устранению всякой возможности сомнения и
  собственной мыслительной деятельности.
\end{enumerate}

Теоцентирзм пронизывает все части религиозной философии: учение о бытии,
о человеке, о познании, об историческом развитии.

\subsection{Патристика}

\emph{\textbf{Патристика ---} (от слова patr -- отец)}
совокупность учений отцов христианской церкви (2-8 века н.э.)


Если в античности считали, что истиной надо овладевать, то в средние
века считалось, что истина уже отклыта в священном писании. Выявление
этой истины и выработка основ христианского богословиястали возможными
благодаря деятельности отцов церкви. Они создали базу для развития
христианской философии, сформулировали философское обоснование
христианских догматов.

\subsection{Схаластика}

\emph{\textbf{Схаластика ---} (рациональная теология)}
тип религиозной философии, характеризующийся соединением теологичесих
предпосылок с рационалистической методикой.


Метод схаластической философии заключался не в нахождении истины,
которая дана в откровении, а в том, чтобы доказать эту истину
посредством разума.

Из этого вытекает 3 цели:

\begin{enumerate}
\def\labelenumi{\arabic{enumi}.}

\item
  С помощью разума проникнуть в истинную веру.
\item
  Придать религиозной и теологической истине систематическую форму при
  помощи философских методов диалектики и логики.
\item
  Используя философские аргументы исключить критику святых истин.
\end{enumerate}

Схаласты использовали идеи античной философии -- идеи Платона и
Аристотеля.

\chapter{Философия эпохи возрождения}

Следующий исторический тип философии --- философия эпохи Возрождения
\emph{(конец 14 -- начало 17 века)}, существенной чертой которой
становится ориентация не на религию, а главным образом на искусство, на
художественно-эстетический феномен.

Возрождение чего? Возрождение интереса к античности, науке, философии и
культуре, но это не возврат к старому, а поиск нового с опорой на
накопленный опыт. Это другая цивилизация, другая культура, знание.

Центром внимания философской мысли выдвигается человек. Если в центре
внимания античности была природно-космическая жизнь, в древние века --
религиозная, проблема спасения, то в эпоху Возрождения на первый план
выходит светская жизнь, деятельность человека в этом мире на Земле.

\begin{quote}
\emph{Происходит секуляризация\footnote{Секуляризация (позднелат.
  saecularis --- мирской, светский). Исходно --- термин католического
  церковного права, обозначающий выход из монашеского ордена с возвратом
  в миряне или переходом в белое духовенство (лат. clerici saeculares)}
--- освобождение церковного влияния.}
\end{quote}

Философия понимается как наука, обязанная помочь человеку найти свое
место в жизни. Возрождение всего этого характеризуется как
антропоцентрическое.

Центральной фигурой данного типа философии является не Бог, а человек.
Его свобода, судьба, деятельность в этом мире ради достижения счастья в
жизни на Земле.

Человек не только важнейший объект философского рассмотрения, но и
центральное звено всей цепи космического бытия. Мыслители эпохи
Возрождения верили в Бога, признавали его творцом мира и человека.

\begin{quote}
\textbf{Бог} --- начало всех вещей, но центр всего мира --
\textbf{человек.}
\end{quote}

Общество не продукт Божьей воли, а результат деятельности людей. В своих
замыслах и деятельности человек не может быть ни чем ограничен. Другая
черта эпохи возрождения -- \emph{гуманизм}.

\textbf{Гуманизм --}
это образ мышления, который провозглашает идею блага человека главной
ценностью социального и культурного развития и отставивает ценность
человека как личность. \emph{Человек -- существо свободное, он творец
самого себя и окружающего мира.}


Мотивы греховной природы человека ослабляются, основной упор делается не
на помощь Божью, а на собственные силы человека.

Основными чертами философии эпохи Возрождения являются оптимизм, вера в
безграничные возможности человека, гордость и достоинства, осознание
своих сил и таланта, жизнелюбия и свободы мысли, воли и эго,
образованность и высокое отношение к труду.

\subsection{Гносеология -- учение о познании}

\textbf{«Что я могу знать?»} --- первый по порядку из трех наиболее
существенных философских вопросов, поставленных
\href{https://ru.wikipedia.org/wiki/\%D0\%9A\%D0\%B0\%D0\%BD\%D1\%82,_\%D0\%98\%D0\%BC\%D0\%BC\%D0\%B0\%D0\%BD\%D1\%83\%D0\%B8\%D0\%BB}{И.
Кантом} перед философией. Поисками ответов на этот вопрос занимается
гносеология \emph{(от греч. gnosis -- знание + logos -- учение)}.
Традиционно она понимается как раздел философии, изучающий возможности
познания, его формы и методы, условия его истинности. Иногда для
обозначения теории познания (особенно в западной традиции) используется
синонимичный термин эпистемология \emph{(от греч. episteme -- знание,
наука + logos -- учение)}.

Гносеология наряду с онтологией традиционно считается основополагающей
частью философской теории. Исторически философия начинается с
размышлений о бытии и продолжается теорией познания. В 17 веке в
философии произошел «эпистемологический поворот\footnote{Эпистемологический
  поворот --- историко-философское понятие для обозначения становления и
  развития эпистемологии в начале и середине 20 века. Эпистемологический
  поворот связан с возникновением аналитической философии, возрождением
  идей трансцендентальной эстетики И. Канта и теории чувственного опыта
  брит, эмпириков-классиков.}», при котором онтология уступила
центральное место теории познания. Время неспешных размышлений о бытии
прошло - человек хотел не созерцать, а преобразовывать мир. Для этого
необходимо было дать ответ на основные вопросы, касающиеся познаваемости
мира и источников наших знаний:

\begin{itemize}

\item
  Что такое познание?
\item
  Какова структура познания?
\item
  Что мы можем знать о мире?
\item
  Что такое истина, ложь и заблуждение?
\item
  Каковы методы познания мира?
\end{itemize}

\subsection{Познание мира}

\subsubsection{Понятие познания}

Потребность в знаниях --- одна из неотъемлемых характеристик природы
человека. Собственно вся история человечества может быть рассмотрена как
ускоряющийся процесс развития, расширения, уточнения знаний --- от
технологий обработки каменных орудий и добывания огня до способов
получения и использования компьютерных массивов информации.

Ценность знаний и способов их получения увеличивается с каждым годом:
ежедневно в мире появляются тысячи новых книг и компьютерных сайтов, а
доля оцифрованной информации исчисляется терабайтами. В социальной
философии этот процесс обозначается как переход от индустриального
общества (основанного на производстве товаров) к обществу
информационному (основанному, прежде всего на производстве и
распределении знания). В этой ситуации проблемы теории познания в ряду
разделов философии и философских дисциплин приобретают особую
актуальность.

\begin{quote}
В целом под \emph{\textbf{познанием} понимают творческую деятельность
человека, направленную на получение достоверных знаний о мире}. Таким
образом, познание есть активный творческий процесс, целью которого
является достижение истины.
\end{quote}

\subsubsection{Субъект и объект познания}

Субъектом познания (тем, кто познает) является человек или общество в
целом. Объектом познания (тем, что познается) является бытие в целом или
же какая-то его часть.

Категории «субъект» и «объект» --- во многом ключевые понятия для
понимания теории и истории гносеологии. Под субъектом в философии
понимается источник активности, действующее лицо. Поскольку активность
проявляет, как правило, человек, чаще всего именно он именуется
субъектом.

Общество как совокупность людей тоже может выступать в качестве
активного, деятельного начала и соответственно, в качестве полноправного
коллективного субъекта познания.

Объектом в противоположность субъекту называют пассивную, страдательную,
инертную сторону взаимоотношений, над которой производится деятельность.
Для различных наук объектом познания являются определенные фрагменты
мира: природа в физике, язык в языкознании, общество в социологии и т.д.
Для философа таким объектом является бытие в целом или соотношение этого
целого со своими частями (языком, обществом, природой). Если объектом
познания является природа, познание считается естественно-научным, если
человек -- гуманитарным, если общество -- социальным. В двух последних
случаях субъект и объект познания могут совпадать и, следовательно,
можно говорить о самопознании.

Помимо объекта в научном познании часто выделяют \emph{\textbf{предмет}
--- часть объекта, которая специально вычленяется познавательными
средствами}. Например, объектом всех гуманитарных наук является человек,
но познавательные средства психологии направлены на духовный мир
человека, археологии -- на его происхождение, культурологии -- на
культуру, а этнографии -- на нравы и обычаи человечества. Соответственно
в качестве предметов этих наук выступают духовный мир, происхождение,
культура и т.д.

От понятий «субъект» и «объект» также образованы термины «объективный» и
«субъективный». Субъективно все, что связано с субъектом, лицом, т.е.
его воля, желания, стремления, предпочтения, чувства и эмоции и т.д.
Таким образом, субъективность есть характеристика внутреннего мира
человека или то личностное воздействие, которое сознание оказывает на
взаимоотношения человека с миром. Субъективное отношение к чему-либо
есть, как правило, вопрос вкуса и у разных людей оно может различаться.
Субъективность больше относят к мнениям, чем к знаниям, хотя и личное
знание является субъективным уже в силу того, что принадлежит сознанию
человека, а не окружающему миру.

Объективно все то, что не зависит от сознания, воли, желаний. Например,
вращение Земли вокруг Солнца, впадение Волги в Каспийское море,
утверждения «Сократ --- это человек», «Ф.М. Достоевский --- русский
писатель» и т.д. являются объективными фактами или отображениями этих
фактов и никак не зависят от наших личных предпочтений. Вне зависимости
от наших желаний Земля не прекратит своего вращения, Волга не повернет
вспять, а Сократ не станет русским писателем.

\subsection{Развитие гносеологии}

В философии существуют две основные точки зрения на процесс познания:
мир познаваем и агностицизм.

Первые (как правило, материалисты) оптимистично смотрят на настоящее и
будущее познание. По их мнению, мир познаваем, а человек обладает
потенциально безграничными возможностями познания.

Агностики (часто -- идеалисты) не верят либо в возможности человека
познавать мир, либо в познаваемость самого мира или же допускают
ограниченную возможность познания. Среди агностиков наиболее известным
является Иммануил Кант. Им была выдвинута последовательная теория
агностицизма, согласно которой:

\begin{itemize}
\item
  сам человек обладает ограниченными познавательными возможностями
  (благодаря ограниченным познавательным возможностям разума)
\item
  сам окружающий мир непознаваем в принципе --- человек сможет познать
  внешнюю сторону предметов и явлений, но никогда не познает внутреннюю
  сущность данных предметов и явлений --- ``вещей в себе''
\end{itemize}

Идеалисты считают познание самостоятельной деятельностью идеального
разума.

Материалисты считают познание процессом, в результате которого материя
через свою отражательную способность --- сознание -- изучает сама себя.

\subsubsection{Современные гносеологические взгляды}

Современная гносеология в своем большинстве стоит на позициях
возможности познания мира и базируется на следующих принципах:

\begin{enumerate}
\def\labelenumi{\arabic{enumi}.}

\item
  Основной метод познания диалектика. Принцип диалектики --- это
  единство диалектических категорий. Категории -- предельно общие
  понятия.\\
  Часть и целое, причина и следствие, материальное и идеальное.
  Содержание и форма. Конечное и бесконечное, вечное и временное,
  сознание и бытие. Количество и качество. Свобода и необходимость.
\item
  Практика --- признавать главным способом познания практику.
  (деятельность человека по преобразованию окружающего мира и самого
  себя).
\item
  Принцип историзма рассматривать все предметы и явления в контексте их
  исторического возникновения и становления.
\item
  Объективности --- признавать самостоятельное существование предметов и
  явлений независимо от воли и сознания субъекта.
\end{enumerate}

\subsection{Чувственное и логическое познание}

Две основные формы познания: чувственное и логическое познание.

Чувственное познание выступает в форме образов, которые возникают в
результате деятельности органов чувств и центральной нервной системы.

\textbf{Образы ---}
это ощущения, восприятия и представления.


Чувственное познание выступает в трех формах: ощущения, восприятия,
представления.

\textbf{Ощущение ---}
это элементарный чувственный образ. Цвет, звук.
\item[\textbf{Восприятием}
называется целостный чувственный образ предмета, когда одновременно
работают несколько органов чувств.
\item[\textbf{Представлением}
называется чувственный образ предмета возникающий в сознании при
непосредственном отсутствии предмета перед органами чувств.


На этапе перехода от восприятий к представлениям возникает сознание.

\subsection{Общенаучные методы и формы познания}

\textbf{Наука ---}
сфера человеческой деятельности, функция которой -- выработка и
теоретическая систематизация знаний о действительности. Включает как
деятельность по получению нового знания, так и ее результат --- сумму
знаний, лежащих в основе научной картины мира


Непосредственные цели --- описание, объяснение и предсказание процессов
и явлений действительности на основе открываемых наукой законов.

Система науки условно делится на естественные, общественные,
гуманитарные и технические науки.

Научное знание как и сама наука может быть охарактеризована с некоторых
позиций (критерии научности):

\begin{enumerate}
\def\labelenumi{\arabic{enumi}.}
\item
  Системность (структурность: возможность переходить от простого к
  сложному и обратно, взаимосвязь частей и целого)
\item
  Возможность логического доказательства (фундамент доказательства ---
  законы формальной логики)
\item
  Возможность экспериментальной проверки
\item
  Отчетливое языковое выражение мысли. Нет противоречий в определении
  одного и того же понятия. Закон тождества.
\end{enumerate}

\subsubsection{Общенаучные методы (приемы образования понятий)}

\textbf{Анализ ---}
мысленное расчленение содержания предмета.
\item[\textbf{Синтез ---}
мысленное соединение признаков предметов.
\item[\textbf{Сравнение ---}
установление сходства и различия между рассматриваемыми предметами.
\item[\textbf{Абстрагирование ---}
выделение единства признаков.
\item[\textbf{Обобщение ---}
более богатая совокупность признаков.


\section{Космологические взгляды Коперника и их значение}

Николай Коперник (1473--1543). Был астрономом, математиком. Рассчитывая
дни пасхальных праздников, обнаружил, что все планеты движутся в одну
сторону, делают петлю, идут в обратную сторону. Заподозрил ошибку в
астрономических взглядах. Утверждал, что Земля вращается вокруг Солнца,
которое является центром мироздания.

Это было революцией, опровергнувшей существовавшую более тысячи лет
геоцентрическую картину мироздания. Успехи в развитии естествознания в
немалой степени определили и характер философских размышлений. Ведущим
направлением философской мысли в 16 веке, как мы уже сказали, становится
\emph{натурфилософия\footnote{Натурфилософия (от лат. natura --
  «природа») -- исторический термин, приблизительно до 18 века
  обозначавший философию природы, понимаемую как целостную систему самых
  общих законов естествознания.}}.

Это открытие перевернуло мировоззрение человека: научные наблюдения не
согласуются со Священным Писанием. Появилась теория двух истин (истина
откровения и научного открытия). Человек был центром Вселенной, а
Коперник это опроверг. Потребовалось выяснить, что значит человек в этом
мире, каков смысл его жизни. Человек утратил чувство определенности, это
стимулировало развитие науки.

\subsection{Кратко о гелиоцентрической системе}

Согласно его концепции смена дня и ночи, а также движение Солнца по небу
объясняются вращением Земли вокруг своей оси. Точно также, при помощи
движения Земли вокруг Солнца, объясняется движение нашего светила по
небосводу в течение всего года.

\subsection{Теория Коперника}

Собор Успения Богородицы в Фромборке, в котором служил отец Николай,
является одной из главных святынь польского католицизма. Собор был
окружен крепкой стеной с оборонительными башнями и мог при необходимости
служить крепостью. Коперник выбрал не слишком удобное место для жилья
--- северо-западную башню соборной стены. На ее верхнем этаже он открыл
свой офис. Оттуда был доступ к широкому валу с хорошей видимостью. По
нему можно было пройти к соседней башне, на которой была подходящая
платформа для наблюдения за другой частью неба. Коперник лично изготовил
гониометрические астрономические инструменты из дерева, подобные тем,
которые описаны в Альмагесте. Среди них «трикветрум» представляет собой
шарнирный треугольник, одна из досок которого была направлена
\hspace{0pt}\hspace{0pt}на Солнце, а другая была отсчитана вниз,
«гороскопия», или солнечный квадрант, представляет собой вертикальную
плоскость с выступающим стержнем в верхний угол. Прибор был установлен
вдоль линии север-юг и позволял судить о наклоне эклиптики к небесному
экватору в направлении полуденной тени в моменты солнцестояний. Не менее
важным инструментом была армиллярная сфера -- вращающиеся кольца,
вложенные друг в друга, которые служили моделью небесных координат и
позволяли получать показания в нужных направлениях.

Фромбок с точки зрения погодных условий и географического положения не
был благоприятным местом для наблюдений, тем не менее, Коперник наблюдал
много, как можно судить по ссылкам в его основной работе «О вращении
небесных сфер».

Целью наблюдений Коперника было не открытие новых небесных явлений.
Астрономы средневековья занимались измерением положения звезд и
сопоставлением их данных с результатами расчетов по схемам Птолемея.
Многие поколения астрономов настроили эпициклы Птолемея для более
надежного предсказания положения планет. В результате точность
предсказаний оставляла желать лучшего, и Вселенная Птолемея усложнялась,
и стало ясно, что Бог не может создать мир, столь неловкий. В записи
Коперника о его наблюдении за Марсом в противостоянии (по отношению к
Солнцу) 5 июня 1512 года. сказано: «Марс превышает расчет более чем на 2
градуса».

Как и другие астрономы, он думал об улучшении шаблонов проектирования.

Первоначально Коперник стремился сделать модель Птолемея более стройной
и простой. В простоте, он был уверен, правда лежала. Путь к упрощению
был предложен самим Птолемеем на страницах Альмагеста, который отвергал
вращение и вращение Земли вокруг Солнца. Но то, что было абсурдным
полторы тысячи лет назад, стало предметом размышлений Коперника.

Движение Земли просто объясняло многие явления: ежегодное движение
Солнца вдоль эклиптики, прецессия оси Земли (если вы сравниваете Землю с
колеблющейся вершиной), «привязанность» Меркурия и Венеры к Солнцу,
необычайное яркость Марса во время его противостояний и, наконец,
петлеобразное движение планет (мы наблюдаем движущиеся планеты с
движущейся Землей).

Затем Коперник «взял на себя труд читать книги всех философов, которые
он мог получить, желая узнать, выражал ли кто-нибудь когда-либо мнение,
что в мировых сферах существуют движения, отличные от тех, которые
предлагают учителя в математических школах.» И он обнаружил в Цицероне,
что пифагорейцы Экфант и Гикет придерживались мнения о вращении Земли
вокруг оси. Аристотель сообщил о ее орбитальном движении в соответствии
с взглядами пифагорейского Филолая. Коперник, к сожалению, не знал
гелиоцентрическую систему Аристарха Самосского, поскольку история
Архимеда о ней была опубликована в Европе после его смерти. Авторитет
древних ученых укрепил Коперника в стремлении завершить
гелиоцентрическую теорию.

\subsection{Сущность теории Коперника}

Коперник представил первый черновик своей теории в работе, известной под
русским названием: «Небольшой комментарий Николая Коперника к гипотезам,
которые он выдвинул относительно небесных движений». Рукописное эссе
появилось около 1515 года. Оно не было опубликовано при жизни автора. В
малом комментарии, после краткого вступления, заканчивающегося
упоминанием теории концентрических сфер Евдекса и Каллиппа, а также
теории Птолемея, Николай Коперник указывает на недостатки этих теорий,
которые заставляют его предлагать свою теорию.

Эта новая теория исходит из следующих требований:

\begin{itemize}

\item
  Не существует единого центра для всех небесных орбит или сфер.
\item
  Центр Земли - это не центр мира, а только центр тяжести и лунная
  орбита.
\item
  Все сферы движутся вокруг Солнца, как вокруг его центра, в результате
  чего Солнце является центром всего мира.
\item
  Отношение расстояния от Земли до Солнца к высоте небосвода (т. е. К
  расстоянию до сферы неподвижных звезд) меньше, чем отношение радиуса
  Земли к расстоянию от нее до Солнца, а расстояние от Земли до Солнца
  ничтожно мало по сравнению с высотой небосвода.
\item
  Любое движение на небосводе связано не с движением самого небосвода, а
  с движением Земли. Земля вместе с окружающими ее элементами (воздухом
  и водой) в течение дня совершает полный оборот вокруг своих неизменных
  полюсов, в то время как небесный свод и расположенное на ней небо
  остаются неподвижными.
\item
  То, что нам кажется движением Солнца, на самом деле связано с
  движениями Земли и нашей сферы, с которыми мы вращаемся вокруг Солнца,
  как и любая другая планета. Таким образом, Земля имеет более одного
  движения.
\item
  Кажущиеся прямые и обратные движения планет вызваны не их движениями,
  а движением Земли. Следовательно, простого движения самой Земли
  достаточно, чтобы объяснить многие кажущиеся неровности на небе.
\end{itemize}

В этих семи тезисах четко очерчены контуры будущей гелиоцентрической
системы, суть которой заключается в том, что Земля одновременно движется
как вокруг своей оси, так и вокруг Солнца.

Формулируя тезисы своей теории, Николай Коперник использует понятия
астрономии в начале 16 века. Итак, в его тезисах речь идет о движении
сфер, а не о движении планет. Ибо движение планет было тогда объяснено
движением сфер, каждая из которых соответствовала определенной планете.
Пятый тезис следует понимать как то, что сфера неподвижных звезд не
участвует в движении планетных сфер, но остается неподвижной. И в
последнем тезисе речь идет о петлях, описываемых планетами в небе из-за
движения Земли вокруг Солнца. В теории Коперника оказалось достаточно
принять допущение, что мы наблюдаем планеты с движущейся Земли,
орбитальная плоскость которой почти совпадает с орбитальными плоскостями
других планет. Такое предположение значительно упростило объяснение
петлеобразного движения планет по сравнению со сложной системой
эпициклов и трима в теории Птолемея.

Четвертый тезис также был чрезвычайно важен: никто до Коперника, а
большинство астрономов, даже после его смерти, не осмеливался
приписывать такие огромные размеры Вселенной. Сформулировав 7 положений
своей теории, Коперник продолжает описывать последовательность
расположения небесных сфер (планет). Затем Коперник останавливается на
том, почему ежегодное движение Солнца в небе следует объяснять только
движением Земли.

Небольшой комментарий заканчивается следующим утверждением: «Таким
образом, только тридцати четырех кругов достаточно, чтобы объяснить
структуру Вселенной и весь круговой танец планет».

Коперник необычайно гордился своим открытием, поскольку он видел в нем
наиболее гармоничное решение проблемы, сохраняя принцип, в силу которого
все планетарные движения могут интерпретироваться как дополнения
круговых движений.

\subsection{«О вращении небесных сфер»}

В Малом комментарии Коперник не приводит математических доказательств
своей теории, отмечая, что «они предназначены для более обширной
композиции». Эта работа -- «О вращении небесных сфер. Шесть книг »(«De
revolutionibns orbium coelestium»), изданная в Регенсбурге в 1543 году,
разделена на шесть частей и напечатана под руководством лучшего и самого
любимого ученика Коперника Ретика. Автору было приятно видеть и держать
в своих руках это творение даже на смертном одре.

Первая часть рассказывает о сферичности мира и Земли, а также излагает
правила для решения прямоугольных и сферических треугольников; второй
дает основы сферической астрономии и правила расчета видимых положений
звезд и планет в небесном своде. Третий относится к прецессии или
прецессии равноденствий, с его объяснением обратным движением линии
пересечения экватора с эклиптикой. В четвертом -- о Луне, в пятом -- о
планетах вообще, а в шестом -- о причинах изменения широт планет.

Написание «главной книги жизни» заняло более 20 лет кропотливой работы.
Астроном считал, что развитие гипотезы, безусловно, следует сводить к
цифрам, а тем более к таблицам, чтобы данные, полученные с ее помощью,
можно было сравнивать с фактическими движениями тел.

В начале книги Коперник после Птолемея излагает основы действий с углами
на плоскости и, что наиболее важно, на сфере, связанной со сферической
тригонометрией. Здесь ученый внес много нового в эту науку, выступая в
качестве выдающегося математика и компьютера. Среди прочего, Коперник
дает таблицу синусов (хотя это название не применяется) с шагом в десять
угловых минут. Но, оказывается, это всего лишь отрывок из более обширных
и точных таблиц, которые он рассчитал для своих расчетов. Их шаг - одна
угловая минута, а точность -- семь знаков после запятой! Для этих таблиц
Копернику нужно было рассчитать 324 тысячи значений. Эта часть эссе и
подробные таблицы были позже опубликованы в виде отдельной книги.

Книга «О вращениях» содержит описания астрономических инструментов, а
также новый, более точный, чем каталог неподвижных звезд Птолемея. Он
понимает видимое движение солнца, луны и планет. Поскольку Коперник
использовал только круговые равномерные движения, ему приходилось
тратить много энергии на поиски таких пропорций размера системы, которые
описывали бы наблюдаемые движения тел. После всех усилий его
гелиоцентрическая система была не намного точнее, чем система Птолемея.
Только Кеплер и Ньютон смогли сделать это точно. Книга также снабжена
анонимным предисловием, которое, как позже установил И. Кеплер, было
написано лютеранским богословом Осиандером. Последний, стремясь скрыть
прямые противоречия между Библией и учением Коперника, пытался
представить его лишь как «удивительную гипотезу», не связанную с
реальностью, но упрощающую вычисления.

Однако истинное значение системы Коперника не только для астрономии, но
и для науки в целом вскоре было широко понято.

\subsection{Мировоззренческая ценность теории Николая Коперника}

Ценность учения Коперника чрезвычайно велика для развития науки: она
привела к настоящей революции не только в астрономии, но и во всем
человеческом мировоззрении. Действительно, если взглянуть на структуру
Солнечной системы, то вопрос о положении Земли и человека во Вселенной
неразрывно связан. Таким образом, астрономия является важным элементом в
понимании мира, охватывающим как философские, так и религиозные
проблемы.

До Коперника на протяжении почти 15 веков Земля считалась единственным
неподвижным телом Вселенной, центральной и самой важной частью
вселенной. Все религии считали, что небесные тела были созданы для Земли
и человечества.

Согласно учению Коперника, Земля --- \hspace{0pt}\hspace{0pt}это обычная
планета, движущаяся вокруг Солнца вместе с другими подобными ему телами.
Поскольку Земля утратила свое центральное положение и стала такой же,
как и все другие планеты, наблюдаемые на небе, утверждение духовенства о
противопоставлении «земного» и «небесного» утратило свое значение.

Возникла новая идея -- о единстве мира, что «небо» и «земля» подчиняются
одним и тем же законам. Человек перестал быть «венцом творения» и
превратился в обитателя одной из планет солнечной системы. Из учения
Коперника был сделан общий вывод, что видимость является лишь одним из
проявлений многогранной реальности, ее внешней стороны, и истинный
механизм явлений лежит гораздо глубже.

Учение Коперника заставило пересмотреть другие отрасли естествознания, в
частности физику, и освободить науку от устаревших и схоластических
традиций, которые препятствовали ее развитию. После Коперника изучение
природы, по сути, освободилось от религии, и развитие науки сделало
гигантские шаги. Ф. Энгельс писал: «Революционным актом, в котором
изучение природы провозглашало свою независимость и как бы повторялось
сожжение Лютером папского быка, была публикация бессмертного творения, в
которое бросил Коперник -- хотя и робко, и поэтому говорить только на
смертном одре -- церковный вызов авторитету в вопросах природы. Отсюда
начинается отсчет освобождения естествознания от богословия.»

Новое научное мировоззрение обрело свои права в жестокой борьбе со
старым мировоззрением, ярыми сторонниками которого были религиозные
фанатики и реакционные ученые. Вначале все они были терпимы к учению
Коперника, считая его мировую систему простой геометрической схемой,
более удобной, чем система Птолемея, для расчета положения звезд на
небе. Церковь не обращала внимания на философские последствия самой
возможности приведения Земли в один ряд с другими планетами. Но к началу
17 в. Религиозные круги хорошо понимали опасность для них учений
Коперника и подвергались жестоким преследованиям против него. В 1616
году указом инквизиции книга «О вращении небесных сфер» была введена «до
исправления» в «Указатель запрещенных книг» и оставалась запрещенной до
1832 года.

\subsection{Отношение церкви}

Католическая церковь вначале не придала большого значения учению,
предложенному Коперником. Но, когда выяснилось, что оно подрывает основы
религии, его сторонники начали подвергаться преследованию. За
распространение учения Коперника в 1600 году был сожжен на костре
Джордано Бруно, итальянский мыслитель.

Научный спор между сторонниками Птолемея и Коперника превратился в
борьбу между реакционными и прогрессивными силами. В конце концов
победили последние.

\subsection{Заключение}

Утверждение гелиоцентрической системы мира является яркой иллюстрацией
бескомпромиссной борьбы, которую прогрессивные, передовые мыслители вели
на протяжении тысячелетий, стремясь узнать объективную правду и законы
развития мира, с представителями реакционных взглядов, сторонниками
церковных догм. Следует отметить, однако, что гений Коперника не был
немедленно преследован церковью. Основная причина этого заключается,
прежде всего, в том, что трактат «Коперник» могли понять только
высокообразованные люди, которые знали, как понимать математические
вычисления и формулы.

Представители протестантизма гораздо быстрее осознали опасность учения
Коперника религии. Первые очень резкие и оскорбительные нападения на
Коперника со стороны основателей протестантской веры Мартина Лютера
(1483-1546) и Филиппа Меланхтона (1497-1560) датируются 1531 годом. Эти
представители протестантизма сразу же заметили глубокие различия и
непримиримые противоречия между блестящими идеями Коперника и догмами
библейских писаний и возглавил фанатичную борьбу против новой доктрины.

Первым защитником учения был Ретик, который опубликовал «Первую историю»
даже при жизни Коперника. В 60-70-е годы 16 века, благодаря трудам Джона
Филда, Роберта Рекода (1510-1558) и Томаса Диггса, учение Коперника было
несколько распространено в Англии. Однако эти работы еще не сделали
систему Коперника известной массам. Только после того, как в Европе
прозвучала страстная проповедь доминиканского монаха Джордано Бруно,
гелиоцентрическая система мира заняла прочное место в умах людей.

\subsubsection{Итог}

В глобальном смысле теория Коперника определила появление новой методики
познания природы, основанной на научном подходе.

\begin{quote}
Согласно схоластической традиции, которой придерживались его
предшественники, для того чтобы познать сущность того или иного объекта,
не нужно детально изучать его внешнюю сторону, а можно постичь его
непосредственно разумом.
\end{quote}

В отличие от них, Коперник показал, что ее можно понять лишь после
тщательного изучения рассматриваемого явления, его противоречий и
закономерностей. Гелиоцентрическая система мира Н. Коперника стала
мощным толчком в развитии науки.

\section{Галилео Галилей космологические взгляды и открытия}

\subsection{Краткие биографические данные}

Галилео Галилей (1564-1642) --- итальянский физик, механик, астроном,
философ, математик, одним из первых использовал телескоп для наблюдения
небесных тел и сделал ряд выдающихся астрономических открытий. Галилей
--- основатель экспериментальной физики, который своими экспериментами
заложил фундамент классической механики. Приверженец гелиоцентрической
системы мира, согласно которой Земля и остальные планеты совершают
движения вокруг Солнца, что привело его к конфликту с католической
церковью.

Родился в городе Пиза (Италия) в семье обедневшего аристократа видного
теоретика музыки и лютниста Винченцо Галилея и Джулии Амманнати.

О детстве Галилея известно немного. Проявлял способность к изучению
языков, литературе, рисованию. От отца унаследовал талант к композиции и
хороший музыкальный слух. Однако по-настоящему интересна ему была только
наука.

В семнадцать лет Галилей стал студентом медицинского факультета
Пизанского университета. Здесь он впервые познакомился с физикой
Аристотеля. Увлекшись механикой и математикой, оставил медицину,
вернулся во Флоренцию, где в течение нескольких лет продолжал занятия
математикой. Он изучал труды Эвклида и Архимеда, именно они оказали
решающее влияние на формирование Галилея как ученого. К этому же времени
относятся его первые работы по гидростатике, которые привели к
изобретению весов для определения удельного веса сплавов, а также
теоретические исследования о центре тяжести тел.

В 1589 году Галилей возглавил кафедру математики в Пизе, а три года
спустя он переехал в Падую и затем в Венецию. Это период наивысшего
творческого расцвета 30-летнего профессора Галилея. К этому времени
относятся его основополагающие исследования по механике: им был открыт
изохронизм колебаний маятника, изобретен пропорциональный циркуль; в эти
годы Галилей стал сторонником и пропагандистом системы Коперника.

Очень важным для ученого стал 1609 год, когда Галилей впервые направил
на небо построенную им зрительную трубу. Результаты наблюдений были
незамедлительно опубликованы Галилеем в сочинении «Звездный вестник».

Слава Галилея росла. Коперниковские взгляды Галилея в то время не были
запрещены.

Но вскоре все изменилось. В 1616 году учение Коперника было объявлено
нелепым и еретичным. Основной труд астронома «О вращении небесных сфер»
было запрещено, а Галилею указали на недопустимость защиты этого учения.
И все же Галилей выступил с пропагандой коперниковского учения. Несмотря
на наличие всех формальных цензурных разрешений на публикацию и даже
устного согласия Папы, инквизиция потребовала суда над Галилеем.
69-летнего ученого вызвали в Рим. После четырех дней допроса и угрозы
пыткой Галилея заставили произнести публичное отречение от учения
Коперника. «Диалог» стал запрещенной книгой, а ее автор --- пожизненным
«узником инквизиции». Ему были запрещены разговоры и рассуждения о
движении Земли, не разрешались встречи с иностранцами. Тем не менее в
Голландии выходит латинский перевод «Диалога», появляются рассуждения
Галилея об отношении Библии и естествознания. В 1638 году в Голландии
выходит, быть может, самая замечательная, по существу, итоговая книга
Галилея «Беседы и математические доказательства, касающиеся двух новых
отраслей науки».

Галилей умер вблизи Флоренции. И только в 1971 году католическая церковь
отменила решение об осуждении Галилея.

\subsection{Основные научные принципы, сформулированные Галилеем}

Галилей пришел к необходимости сосредоточить основное внимание физики на
таких понятиях, как пространство, время, тяготение, скорость, ускорение,
сила и импульс. В выборе этих понятий проявился гений Галилея, так как
их важность в начале 17 в. не была очевидной, а соответствующие
физические величины не всегда доступны прямому измерению.

Галилей первым поставил задачу получить количественное описание
физических явлений, т. е. облечь физические законы в математические
формулировки. Это коренным образом противоречило подходу Аристотеля,
считавшему, что естественные науки имеют дело с изменяющимися объектами,
в то время как математика --- с неизменными. Именно поэтому «Физика»
Аристотеля изложена без использования математики. В предложенный
Галилеем план изучения природы входило выделение наиболее
фундаментальных характеристик природы, которые, по его мысли, надо
научиться измерять, а затем принять их в качестве переменных в
математических формулах. Галилей считал, что ключом к пониманию языка
Вселенной является математика. Он утверждал, что книга природы написана
математическими символами, без знания которых человек не сможет понять в
ней ни одного слова.

Галилей сыграл решающую роль в происшедшем в дальнейшем развитии науки
переломе в пользу экспериментального подхода, который окончательно
утвердился только в 19 в. Он подчеркивал, что если мы хотим установить
правильные основополагающие принципы, то необходимо «прислушиваться к
голосу природы» (а не следовать тому, что кажется предпочтительным
нашему разуму). Критикуя средневековых схоластов, занимавшихся в
основном изощренными логическими построениями и спорами, Галилей
неустанно повторял, что знания берутся из наблюдений, а не из книг.
«Природа создает свои творения как ей заблагорассудится, и человеческому
разуму надо напрягать все силы, чтобы понять ее». Надо сказать, что
понятие эксперимента Галилей трактовал весьма широко. Будучи блестящим
экспериментатором, он часто проводил так называемый умозрительный
эксперимент (опыт в уме). Например, размышляя о движении тел под
действием силы тяжести, Галилей вначале придерживался позиции
Аристотеля, согласно которой тяжелые тела падают на землю быстрее, чем
легкие. Но затем он провел следующий умозрительный эксперимент. Если к
тяжелому камню добавить легкий, то он должен падать быстрее, так как его
масса при этом возрастет. С другой стороны, добавление к тяжелому телу
части, падающей медленнее, должно его тормозить. Возникает противоречие,
разрешить которое можно единственным способом --- считать, что и
тяжелый, и легкий камень падают с одинаковой скоростью.

Именно Галилей впервые высказал и неоднократно использовал важнейший
научный принцип --- принцип идеализации (игнорирования второстепенных
деталей). Например, всякий реальный предмет обладает определенными
размерами и геометрической формой, однако при проведении, скажем,
физических экспериментов ни размеры, ни форма предмета не играют
существенной роли, поэтому допустимо рассматривать его как материальную
точку, в которой сосредоточена вся его масса (точечная масса). Принцип
идеализации играет важнейшую роль в современной математике и физике.

Галилей внес решающий вклад в развитие представлений о движении. Один из
основных принципов средневековой натурфилософии, восходящий к
Аристотелю, гласит: «Все, что движется, движется посредством чего-то»,
т. е. посредством действующей силы. Галилей первым высказал
предположение, что если бы не было трения и сопротивления воздуха, то
всякое тело, будучи раз приведенным в движение, двигалось бы (в
отсутствие внешних воздействий) неограниченно долго и равномерно. Под
действием силы происходит лишь изменение скорости, т. е. создается
ускорение. Таким образом, постоянно действующая сила есть причина не
скорости, а ускорения.

Всякое тело сопротивляется изменению своей скорости --- как по величине,
так и по направлению. Мера сопротивления изменению скорости тела
называется его массой (точнее, инертной массой).

Важнейшей заслугой Галилея было установление того факта, что все
механические явления протекают одинаково во всех системах отсчета,
которые движутся равномерно и прямолинейно (инерциальных системах
отсчета), --- это положение называется принципом относительности
Галилея. Опыт, который Галилей предложил для подтверждения этого
принципа, состоит в следующем. Если наблюдать за протеканием различных
механических процессов на движущемся (равномерно и прямолинейно) корабле
и на стоящем неподвижно корабле --- никакой разницы заметить невозможно.
Именно этим объясняется тот факт, что никто из нас не замечает никаких
последствий орбитального движения Земли (которое происходит с огромной
скоростью --- 30 км/с). Осознание принципа относительности Галилея
сыграло важнейшую роль в признании гелиоцентрической системы Коперника.

\subsection{Космологические труды Галилея}

У Галилео Галилея впервые связь космологии с наукой о движении приобрела
осознанный характер, что и стало основой создания научной механики.
Первоначально Галилеем были открыты законы механики, но первые
публикации и трагические моменты его жизни были связаны с менее
оригинальными работами по космологии.

Изобретение голландцем Хансом Линнерсхеем, телескопа дало возможность
Галилею «открыть новую астрономическую эру». Оказалось, что Луна покрыта
горами, Млечный Путь состоит из звезд, Юпитер окружен четырьмя
спутниками и т.д. Иначе говоря, другие планеты оказались похожи на
Землю. «Аристотелевский мир» с его геоцентризмом рухнул окончательно.

Галилей поспешил опубликовать увиденное в своем «Звездном вестнике»,
который вышел в марте 1610 г. Книга была написана на латыни и
предназначена для ученых.

В 1632 г. во Флоренции была напечатана наиболее известная работа
Галилея, послужившая поводом для судебного процесса над ученым. Ее
полное название --- «Диалог Галилео Галилея Линчео, Экстраординарного
Математика Пизанского университета и Главного Философа и Математика
Светлейшего Великого Герцога Тосканского, где в четырех дневных беседах
ведется обсуждение двух Основных Систем Мира, Птолемеевой и
Коперниковой; и предполагаются неокончательные философские и физические
аргументы как с одной, так и с другой стороны».

Эта книга была написана на итальянском языке и предназначалась для
«широкой публики». В книге много необычного. Несмотря на легкость и
изящество литературной формы, книга полна тонких научных наблюдений и
обоснований (в частности, по вопросам инерции, гравитации и т.д.).

Вместе с тем Галилей не создал цельной космологической системы.

\subsection{Новая механика Галилея}

В 1638 г. вышла последняя книга Г. Галилея «Беседы и математические
доказательства, касающиеся двух новых отраслей науки, относящихся к
механике и местному движению\ldots», в которой он касается проблем
механики, решенных им около 30 лет назад.

Механика Галилея дает идеализированное описание движения тел вблизи
поверхности Земли, пренебрегая сопротивлением воздуха, кривизной земной
поверхности и зависимостью ускорения свободного падения от высоты.

В основе «теории» Галилея лежат простые аксиомы, но сам он в явном виде
их не сформулировал. Суть этих идей следующая:

\begin{itemize}

\item
  свободное движение по горизонтальной плоскости происходит с постоянной
  по величине и направлению скоростью (сегодня так понимается закон
  инерции, или первый закон Ньютона)
\item
  свободно падающее тело движется с постоянным ускорением
\item
  тело, скользящее без трения по наклонной плоскости, движется с
  постоянным ускорением, зависящим от угла наклона плоскости. Эта
  зависимость позволяла связать скорость движения с высотой «горки».
  Галилей чрезвычайно гордился этой формулой, поскольку она позволяла
  определять скорость с помощью геометрии
\item
  скорость тел относительна и зависит от выбора системы отсчета.
  «Преобразования Галилея» (среди которых формула, связывающая скорости
  тела относительно неподвижной и движущейся систем отсчета) позволяли
  переходить от значений физических величин в одной системе к значениям
  в другой
\item
  траектория снаряда описывается уравнением параболы, которое связывает
  начальные условия, высоту и дальность полета
\end{itemize}

\subsection{Заключение}

Основа мировоззрения Галилея --- признание объективного существования
мира, то есть его существования вне и независимо от человеческого
сознания. Мир бесконечен, считал он, материя вечна. Во всех процессах,
происходящих в природе, ничто не уничтожается и не порождается ---
происходит лишь изменение взаимного расположения тел или их частей.
Материя состоит из абсолютно неделимых атомов, её движение ---
единственное, универсальное механическое перемещение. Небесные светила
подобны Земле и подчиняются единым законам механики. Всё в природе
подчинено строгой механической причинности. Подлинную цель науки Галилей
видел в отыскании причин явлений. Он считал, что познание внутренней
необходимости явлений есть высшая ступень знания. Исходным пунктом
познания природы Галилей считал наблюдение, основой науки --- опыт.

Галилео Галилей утверждал, что задача ученых не добывать истину из
сопоставления текстов признанных авторитетов и путем абстрактных,
отвлеченных умствований, а «\ldots изучать великую книгу природы,
которая и является настоящим предметом философии». Развивая в
гносеологии идею безграничности «экстенсивного» познания природы,
философ-естествоиспытатель допускал и возможность достижения абсолютной
истины, т.е. «интенсивного» познания.

\subsubsection{Итог}

В изучении природы Галилей выделял два основных метода познания:

\begin{itemize}

\item
  аналитический
\item
  синтетический
\end{itemize}

Сущность первого заключалось в том, что понятие опыта, в отличие от
своих предшественников, Галилей не сводил к простому наблюдению, а
предпочитал планомерно поставленный эксперимент, посредством которого
исследователь как бы ставит природе интересующие его вопросы и ищет на
них ответы. Метод этот ученый назвал резолютивным, который, в сущности,
есть метод анализа, расчленения природы, т.е. аналитический.

Другим важнейшим методом познания Галилей признавал композитивный, т.е.
синтетический, который посредством цепи дедукции проверяет истинность
выдвинутых при анализе гипотетических предположений. При этом Галилей
считает, что, хотя опыт и является исходным пунктом познания, но сам по
себе он не дает еще достоверного знания. Последнее достигается
планомерным реальным или мысленным экспериментированием, которое
опирается на строгое количественно-математическое описание. В итоге
достоверное знание мы получаем при сочетании синтетического и
аналитического, чувственного и абстрактного.

Таким образом, гениальным ученым была разработана методика научного
исследования, которая состояла из следующих этапов: наблюдение,
выдвижение гипотезы, расчеты по воплощению гипотезы на практике и
экспериментальная проверка выдвинутой гипотезы на практике.

В итоге можно сказать, что Галилей заложил основы современной физики и
создал прообраз современной научной мысли. Как говорят А. Эйнштейн и Л.
Инфельд, «переход от аристотелева образа мышления к галилееву положил
самый важный краеугольный камень в обоснование науки».

\section{Натурфилософия Николая Кузанского}

Николай Кузанский (1401 - 1464) - родоначальник ренессансной
натурфилософии. Он не отрицал то мировосприятие, которое было присуще
средневековой философии. Мыслитель представлял окружающий мир как
сотворенное божественное бытие и при этом бесконечное. Кузанский
соглашался с тем, что идеи природы в средневековой философии является
идеей фундаментальной. Главный его тезис, что бог, это бесконечное бытие
и бесконечное единство. На основе идеи единства Кузанский считает
необходимым развить концепцию бесконечности.

\subsection{Биография}

Николай Кузанский родился в Кузе на реке Мозель. По месту своего
рождения он и получил прозвание --- Кузанский или Кузанец. Достоверных
сведений о детских годах жизни будущего мыслителя нет. Известно лишь,
что отец его был рыбаком и виноградарем, а сам Николай подростком бежал
из родного дома. Его приютил граф Теодорик фон Мандершайд. Возможно,
Николай учился в школе «братьев общей жизни» в Девентере (Голландия).
Затем он продолжил обучение в Гейдельбергском университете (Германия) и
в школе церковного права в Падуе (Италия). В 1423 г. Николай получил
звание доктора канонического права. Вернувшись в Германию, он занимался
богословием в Кёльне. В 1426 г., вскоре после того как он получил сан
священника, Николай становится секретарем папского легата в Германии
кардинала Орсини. Через некоторое время он стал настоятелем Церкви св.
Флорина в Кобленце.

В эти годы Николай Кузанский впервые знакомится с идеями гуманистов,
которые оказывают на него определенное влияние. Недаром он оказался
среди тех римско-католических священников, которые выступали за
ограничение власти римского папы и усиление значения церковных соборов.
В своем первом сочинении «О согласии католиков» он, кроме того,
высказывал сомнение в истинности «Константинова дара», а также
провозгласил идею народной воли, имеющей равное значение для Церкви и
государства. В 1433 г. эти идеи он высказывал на Базельском соборе. Но
уже к концу собора Николай перешел на сторону папы, видимо усомнившись в
возможности осуществления реформы.

Вскоре Николай Кузанский поступил на службу в папскую курию. В 1437 году
он входил в состав папской делегации в Константинополь, которая должна
была встретиться с императором, патриархами и возможными делегатами от
восточных церквей для объединительного собора между западной и
восточными церквами. Глядя с опаской на османскую угрозу, греки всё
больше стремились к унии. Однако, собор, открытый в Ферраре и
продолженный во Флоренции, не дал желаемых результатов. По дороге из
Константинополя на Кузанца, по его словам, сошло Божественное
откровение, которое вскоре станет основой знаменитого трактата «De docta
ignorantia» («Об учёном незнании»).

В 1448 г. Николай был возведен в сан кардинала, а уже в 1450 г. --
епископом Бриксена и папским легатом в Германии. В 50-е гг. XV века
Кузанец много путешествует, стремится примирить различные христианские
течения Европы, в частности, гуситов с католической Церковью.

В 1458 г. Николай вернулся в Рим и в качестве генерального викария
пытался проводить реформы Церкви. Он рассчитывал на успех, ибо новым
папой Пием 2 стал друг его юности Пикколомини. Но смерть помешала
Николаю Кузанскому завершить задуманное.

\subsection{Философия и теология}

Николай Кузанский внёс вклад в развитие представлений, прокладывавших
дорогу натурфилософии и пантеистическим тенденциям 16 веке. В отличие от
современных ему итальянских гуманистов, он обращался в разработке
философских вопросов не столько к этике, сколько, подобно схоластам, к
проблемам мироустройства. Традиционно понимая Бога как творца, «форму
всех форм», немецкий мыслитель широко использовал математические
уподобления и диалектическое учение о совпадении противоположностей,
чтобы по-новому осветить соотнесение Бога и природы. Николай Кузанский
их сближает. Подчёркивая бесконечность Бога, он характеризует его как
«абсолютный максимум», в то же время отмечая, что любые определения его
ограничены. Мир трактуется, как некое «развёртывание» Бога. Суть своих
взглядов, пантеистическая тенденция которых опирается на широчайшие
философские основы от Платона и неоплатонизма до мистики средневековья,
Николай Кузанский выразил в формуле «Бог во всём и всё в Боге». Много
внимания он уделяет и проблеме места человека в мире. Изображая все
явления природы взаимосвязанными, он видит в человеке «малый космос»,
намечает его особую центральную роль в сотворённом мире и способность
охватывать его силой мысли.

Занятия философией вызвали в нём отвращение к аристотелизму: он резко
осуждает Аристотеля, смешивая перипатетизм схоластический с подлинной
аристотелевской философией и распространяя свою антипатию к последней на
самую личность Аристотеля. Он основал в Неаполе естественно-историческое
общество Academia Telesiana. Последние годы его учёно-философской
деятельности протекли в тяжёлой борьбе против врагов свободного и
непосредственного исследования законов природы, которое настойчиво
провозглашал Телезий\footnote{Бернардино Телезио (Телезий, Телесий)
  (1509-1588 г.) -- итальянский учёный и философ. Окончил Падуанский
  университет в 1535. Основное сочинение --- «О природе вещей согласно
  её собственным началам» (1565).}. Его девизом служили слова «Realia
entia, non abstracta» \emph{(на лат. -- «Реальный, а не абстрактный»)}.
В предисловии к своему важнейшему философскому труду «De natura rerum
juxta propria principia» \emph{(на лат. -- «У природы свои правила»)}
(1565) он говорит, что берёт в руководители свои чувства, а предметом
своего исследования -- природу, которая всегда остаётся неизменною в
своей сущности, следует тем же законам, производит те же явления.
Провозглашая верховное значение опыта как главного источника познания,
Телезий не применяет в достаточной мере этот принцип при исследовании
явлений внешней природы.

Его натурфилософия напоминает досократовские наивные умозрения ионийцев.
Из противоположности между небом с его светилами, приносящими тепло, и
землёй, из которой после заката солнца появляется холод, он выводит, что
эти два начала -- первоосновные в природе. Кроме того, по словам
Телезия, есть нечто телесное (corporea moles), оно расширяется и
утончается под влиянием теплового начала и сжимается и сплачивается под
влиянием холодного начала. Тепло -- источник движения и жизни, холод --
смерти и покоя: борьба этих двух начал -- источник всего мирового
развития.

\subsection{Астрономия}

С именем Николая Кузанского связаны также важные натурфилософские
представления о движении Земли, которые не привлекли внимания его
современников, но были оценены позже. Заметно опередив своё время, он
высказал мнение, что Вселенная бесконечна, и у неё вообще нет центра: ни
Земля, ни Солнце, ни что-либо иное не занимают особого положения. Все
небесные тела состоят из той же материи, что и Земля, и, вполне
возможно, обитаемы. Почти за два века до Галилея он утверждал: все
светила, включая Землю, движутся в пространстве, и каждый наблюдатель
вправе считать себя неподвижным. У него встречается одно из первых
упоминаний о солнечных пятнах. Николай Кузанский отметил плохую точность
юлианского календаря и призвал к календарной реформе (эта реформа долго
обсуждалась и была реализована только в 1582 году).

Астрономические труды Николая Кузанского, по мнению историков науки,
оказали (прямое или косвенное) влияние на взгляды Коперника, Джордано
Бруно и Галилея.

\subsection{Залкючение}

Согласно этой концепции бесконечное следует понимать как абсолютное
совершенство, в которой нет каких либо ступеней и иерархий. Кузанский
использует понятие максимума, разработанное им с помощью математического
метода . Максимум не имеет никаких количественных градаций. Максимум --
абсолютное единство, абсолютная полнота окружающего нас мира. Максимум
-- абсолютная основа познавательной деятельности.

Чувственное познание не является истинным, достоверным познанием.
Поэтому чувственное познание никогда не может достичь совершенства,
достичь бесконечности, которое в своей определенности и полноте и есть
бесконечный максимум. Таким образом, понятие бесконечного становится
ключевым понятием в учении о боге, мире и человеке.

Нам важно отметить, что в средневековой мысли понятие бесконечности
имело вторичный характер. Человек в силу своей ограниченности мог
постигать только конечный мир природных предметов. В своей концепции
позитивной бесконечности Николай Кузанский приходит к осознанию
необходимости новых условий познания.

Новое познание окружающего мира и Бога должно начинаться с бесконечности
как максимума, чтобы затем достигать достоверного знания о конечных
предмета природы.

Идея окружающего мира как бесконечности представляет собой новую
концепцию природы, которая порождала революционные последствия для
развития математики, философии и теологии. Человек не способен получит
истинное знание о конечном как таковом без учета бесконечного. С
конечного следует начинать познание истины в изменчивом мире, с
бесконечного же надлежит постигать истину всего конечного.

Знание которое достигается через постижение конечного, -- не является
подлинным знанием. Это лишь «ученое незнание». Вне позитивного понимания
бесконечного мы не можем достичь истинного знания окружающего нас мира.
Истинное знание возможно только в сфере понимания бесконечного бытия
бога. Только на таком основании мы способны обрести знание о реальном
мире, о природе как таковой.

Таким образом, Николай Кузанский, отвергая терминологию Священного
писания, ставит проблему бога не как теологическую, а как собственно
философскую проблему. Речь идет о соотношении конечного мира, мира
конченых вещей с их бесконечной сущностью. Концепцию Кузанского об
отношении бога и мира необходимо определять как пантеизм\footnote{Пантеизм
  -- это философское учение, в котором бог объединяется или
  отождествляется с мирозданием. Сторонники данной концепции исходят из
  того, что Бога как отдельной личности или некоего высшего разума не
  существует. Вместо этого они считают, что само мироздание имеет
  божественную сущность.}. Пантеизм - философское учение, согласно
которому бог и мир являются отождествленными.

\subsubsection{Итог}

В эпоху Возрождения поэзия, риторика, художественное творчество
выполняли функцию средств формирования нового мироотношения. Архитектор
и живописец понимали свои творения как проект новой науки, отличной от
средневековой схоластики.

С помощью инженерно-художественного творчества, в котором важным
компонентом являлась математика, осуществлялось утверждение
антропоцентрического мировоззрение, в котором человек подобно богу
способен представить высшую гармонию мироздания и математически ее
продемонстрировать.

В эпоху Возрождения формируется антропоцентрический уровень сознания, в
котором человек воссоздает свое бытие как художественное творение. Он
утверждает значение собственной личности в противовес религиозной
нравственности.

Натурфилософия Возрождения выступает против догматического мышления,
ограничивающего научные и философские исследования системой заданных
постулатов. Натурфилософия в лице Кузанского, Монтеня, Бруно выдвигает
требование свободы мысли.

\section{Философия Джордано Бруно}

Джордано Бруно (1548 - 1600) --- представитель позднего Ренессанса.
Права человека для Джордано Бруно -- вне всяких ограничений. Он
развивает философские представления Кузанского и астрономические
воззрения Коперника.

\subsection{Биография}

Место, где родился философ, называется Нола. Ныне это коммуна в Италии с
населением почти 35 тыс. человек. Поселение известно с античных времен:
вблизи населенного пункта произошли три сражения Второй Пунической
войны, здесь умер император~Октавиан Август.

Бруно родился в городе, который считался объектом христианского
паломничества. Нола входила в состав Неаполитанского королевства,
которое подчинялось Испании. Глава семейства Джованни Бруно имел
дворянские корни. Но род постепенно обеднел и потерял влияние. Джованни
состоял на военной службе с жалованьем 60 дукатов в год, что считалось
небольшим доходом. Например, итальянские чиновники средней руки получали
200--300 дукатов.

В свободное время отец ухаживал за посадками в огороде и саду, которые
обеспечивали семью продуктами. Джованни поддерживал Испанию и уважал
власть, поэтому назвал сына Филиппо в честь короля Филиппа 2. Мать
мальчика, Фраулисса Саволина, происходила из крестьянской семьи. Женщина
помогала мужу вести домашнее хозяйство.

Точная дата рождения Джордано Бруно неизвестна. Но история сохранила
данные, что философ родился в январе 1548 года, когда в Русском
государстве царь~Иван Грозный~готовился к первому Казанскому походу.

Бруно обладал уникальной памятью, поэтому рассказывал эпизод из
младенчества. Джордано вспоминал, что лежал в колыбели, когда через щель
в доме в комнату заползла змея. Ученый рассказывал, что на помощь пришел
отец, который отогнал пресмыкающееся от единственного сына. История
ассоциируется с мифологическим детством~Геракла, которому была
предначертана великая судьба.

Дом, в котором прошло детство Бруно, был неказистый, но вид из окна
восхищал воображение мальчика. Юный Филиппо любовался могучим Везувием,
который вызывал у ребенка ассоциации с чем-то мифическим и загадочным.

Отец много времени проводил с сыном. Вероятно, Джованни привил Джордано
любовь к литературе и сложению стихотворений. Отец рассказывал о
древнеримских поэтах и показывал, где похоронен~Вергилий.

Юного Филиппо не заинтересовали ни военное дело, ни торговля, ни
государственная служба. Мальчик чувствовал тягу к знаниям, стремился
познать мир, в котором живет, объяснить сущность бытия.

В 11 лет Бруно привезли в Неаполь, где мальчик изучал литературу, логику
и диалектику. Через четыре года юноша перешел под опеку монахов и сменил
имя Филиппо на Джордано. Поначалу подросток был прилежным учеником. В
1568 году Бруно написал первое произведение «Ноев ковчег», которое
подарил папе Пию 5 во время поездки в Рим.

В 24 года Бруно стал священником, но взгляды служителя церкви
изменились. Джордано много читал, рассматривал различные теории. Догмы
католической веры не вдохновляли молодого человека. Все началось с того,
что Джордано убрал из своей кельи иконы, оставив только распятие.

Руководство начало расследование против молодого священника, который не
только оскорбил католицизм, но и читал запрещенную литературу. Бруно не
стал дожидаться, пока его обвинят в ереси, покинул Неаполь и поехал в
Рим. Вечный город оказался небезопасным, поэтому философ продолжил путь.

Сначала Джордано скитался по северу Италии, который изъездил с запада на
восток: побывал в Генуе, Турине и Венеции. Подрабатывал преподавателем,
чтобы свести концы с концами. Чтобы окончательно скрыться от католиков,
Бруно перебрался в Швейцарию, где заинтересовался идеями кальвинизма. Но
имеющий на все собственное мнение Джордано и там нажил себе врагов и был
обвинен в ереси, но уже со стороны последователей~Жана Кальвина.

Из Швейцарии Бруно бежал во Францию. Итальянец выступал с лекциями и
обратил на себя внимание короля~Генриха 3 Валуа. Правителя поразили
обширные знания и отличная память философа.

Генрих сам увлекался идеями, которые осуждала церковь. Однажды король
даже посылал отряд в Испанию на поиски трактата «Пикатрикс», который
содержал сведения о симпатической и астральной магии и делал акцент на
талисманах.

Таким образом, Генрих не был профаном в запрещаемых священнослужителями
темах, и речь Бруно заинтересовала монарха, который пригласил ученого ко
двору, дав несколько лет отдыха после скитаний, а позднее вручил
рекомендательное письмо для поездки в Англию.

Философ жил в Лондоне, а позже в Оксфорде. Английский период считается
одним из самых ярких в научной деятельности Бруно. В Туманном Альбионе
ученый написал труд «О бесконечном, Вселенной и мирах», который
обеспечил его имени увековечивание в истории.

\subsection{Натурфилософия Джордано Бруно}

Идеи Н. Кузанского и Н. Коперника получили свое развитие в
натурфилософии итальянского мыслителя Джордано Бруно, воплотившего в
своем творчестве наиболее полно и глубоко такие важные черты
гуманистической философии как пантеизм, диалектичность, острое чувство
гармонии природы, ее бесконечности.

Радикальный пантеизм мыслителя, т.е. абсолютное отождествление Бога и
природы, что отрицало постулат вероучения о сотворенности мира --
причина его непримиримого конфликта с церковью.

Джордано Бруно пошел дальше Коперника, отрицая не только
геоцентрические, но и гелиоцентрические представления об устройстве
мира. Он обосновал идею бесконечности Вселенной, идею о существовании во
Вселенной бесчисленного множества миров, подобных миру Солнечной
системы. «Вселенная -- писал он, --- есть бесконечная субстанция,
бесконечное тело в бесконечном пространстве. Вселенная одна --- миры же
бесчисленны. У каждого мира -- своя звезда. Эта Вселенная не сотворена,
она существует вечно и не может исчезнуть. В ней происходит непрерывное
изменение и движение».

Основополагающим в его учении является понятие Единого, являющегося и
причиной бытия, и самим бытием вещей. Во Вселенной, которую представлял
Джордано Бруно, не оставалось особого, а тем более центрального места,
для личности Бога, но самого факта его существования он не отрицал. Он
представлял Бога по-своему, как нечто, погруженное в природу,
растворенное в бесконечности. Бог для Джордано Бруно -- «душа мира»,
существующая «внутри материи». Исходя из неразрывности бога и природы,
Джордано Бруно придавал последней активную роль, утверждал, что материя
«творит все из своего лона». Так как Бог отождествляется с природой, то
он немыслим вне материального мира: «Искусство имеет дело с чужой
материей, природа -- со своей собственной. Искусство находится вне
материи, природа -- внутри материи, более того: она сама есть материя.
Итак, материя все производит из собственного лона, так как природа сама
есть внутренний мастер, живое искусство, она есть двигатель, действующий
изнутри». В этом и состоит натуралистический пантеизм Бруно.

Джордано Бруно придавал физическую однородность всем бесконечным мира,
придерживался гилозоизма (всеобщая одушевленность природы), объясняя тем
самым причину движения космических тел: закон всемирного тяготения еще
не был открыт.

Активно использует положения диалектики Кузанского, освобождает ее от
теологического содержания и формулирует как учение о природе. Так,
например, Джордано Бруно отказывается от признания абсолютного центра
Вселенной: бесконечность Единого исключает саму возможность подобного
центра. Тем самым снимаются различные теологические ограничения
бесконечности Вселенной, окружающего мира. «Вселенная никоим образом не
может быть охвачена и поэтому неисчислима и беспредельна, а тем самым
бесконечна и безгранична». Эта Вселенная не сотворена, она существует
вечно и не может исчезнуть. Она неподвижна, «ибо ничего не имеет вне
себя, куда бы могла переместиться, ввиду того, что она является всем». В
самой же Вселенной происходит непрерывное изменение и движение.

Обращаясь к характеристике этого движения, Бруно указывает на его
естественный характер. Он отказывается от идеи внешнего
перводвигателя, т.е. Бога, а опирается на принцип самодвижения материи.
А материя, говорит Бруно, «столь совершенна, что она, как это ясно при
правильном созерцании, является божественным бытием в вещах».

Разорвав границы мира и утвердив бесконечность Вселенной, Бруно
оказывается перед необходимостью выработать новое представление о Боге и
его отношении к миру. Решение этой проблемы свидетельствует о
пантеистической позиции мыслителя. Бруно утверждает, что «Природа есть
или ничто, пли божественное могущество, воздействующее изнутри на
материю, и запечатленный во всем вечный порядок\ldots».

Для католической инквизиции Джордано Бруно был перерожденцем. Выдающийся
мыслитель закончил свою жизнь трагически -- 17 февраля 1600 года он был
заживо сожжён на площади Цветов в Риме. Его труды тоже сожгли. Его имя
было запрещено упоминать публично.

\subsection{Следствие и казнь}

Ученый томился в венецианской темнице до февраля 1593-го, когда пришло
требование доставить его в Рим для инквизиторского суда. Семь лет
мыслителя держали в тюрьме и требовали отречься от взглядов, которые
признавались церковью еретическими. В ходе следствия Бруно отказался от
выдвигаемых условий -- священники не спешили казнить Джордано, считая
его ценным философом и ученым, который «сбился с пути». Но итальянец
продолжал упорствовать и высказывать новые идеи, идущие вразрез с
официальной позицией христианского учения.

Бруно в процессе следствия заявил, что душа человека после смерти
переселяется в другое тело и продолжает жить на Земле. Терпение
инквизиторов лопнуло, поэтому Джордано решили предать светскому суду.
Священники поручили казнить ученого «без пролития крови».

Историографы отмечают, что Бруно стойко выслушал заключение суда и
твердо произнес: «Сжечь не значит опровергнуть». Казнили философа в Риме
17 февраля 1600 года.

Кстати,~Галилео Галилей~тоже поддерживал учение Коперника, но на суде
признал ошибочными свои суждения, за что получил мягкий приговор. Однако
позже ученый сказал знаменитую фразу, ставшую цитатой: «И все-таки она
вертится!».

\subsection{Заключение}

Философия Бруно в целом материалистична, но по форме это пантеизм. У
него Бог окончательно растворяется в природе, которая, по его словам,
есть «Бог в вещах». Природа бесконечна, поэтому Бруно отказывается от
воззрения польского астронома Н. Коперника, согласно которому Солнце
представляет собой абсолютный центр Вселенной. Такого центра во
Вселенной, по мнению Бруно, нет. Солнце -- лишь относительный центр
нашей планетной системы. Во Вселенной есть бесконечное множество
обитаемых планет: «другие миры так же обитаемы, как этот».

Атомистическую теорию он считал ценной для физики, в связи именно с этим
признавал атом физическим минимумом. Что же касается философии, то
признание атомов и пустоты он считал недостаточным для решения ее задач.
Философу нужна «материя», которая бы «склеивала» атомы и пустоту.
Поэтому философским минимумом он признавал не атом, а монаду\footnote{Монада
  (греч. μονάδα, от др.-греч. μονάς, μονάδος -- единица, простая
  сущность, от μόνος -- один) -- согласно пифагорейцам, обозначала
  «божество», или «первое существо», «единицу» или «единое, как
  неделимое». Позднее --- многозначный термин в различных философских
  системах Нового времени и современности, в психологии и эзотерике.},
которой соответствует философский максимум -- бесконечная природа в
единстве всех ее форм.

Материя есть начало, которое все производит. Она есть первооснова,
субстанциальная причина всего существующего, Единое. В Едином заключена
внутренняя активность, которую Бруно называет «душой мира», всеобщим
разумом.

\subsubsection{Итог}

Большое значение для развития философии имела теория познания Бруно.
Познание должно начинаться с сомнения во всем. Лишь рассмотрев два
противоположных суждения, лишь путем столкновения двух противоположных
взглядов можно обнаружить истину. Существует три ступени постижения
истины. Первая ступень познания -- это чувство, ощущение, вторая ступень
-- разум и высшая ступень познания -- ум. Полная истина познается лишь
при помощи интеллекта, который познает внутренние связи вещей.

Философия Бруно в целом оптимистична. Мир гармоничен и совершенен,
несовершенство и смерть присущи лишь единичным явлениям. Уничтожение
кладет начало возникновению и наоборот. Источник этой связи
противоположностей, их внутреннего родства -- бесконечная субстанция,
частицей которой является и человек.

\section{Философия Френсиса Бекана}

Фрэнсис Бэкон (1561-1626) --- английский философ и политический деятель,
в 1620-1621 гг.-- лорд-канцлер Великобритании, второе должностное лицо в
стране после короля), явился основателем эмпирического направления в
философии.

\section{Биография}

Фрэнсис родился в семье политического деятеля и ученого Николаса, и его
жены Анны, которая происходила из известной в те времена семьи -- ее
отцом был воспитан наследник английского и ирландского престолов Эдвард
VI. Роды случились 22 января 1561 года в Лондоне.

Мальчика с детства приучали быть прилежным и поддерживали его тягу к
знаниям. Подростком он посещал колледж при Кембриджском университете,
потом отправился учиться во Францию, но смерть отца привела к тому, что
у юного Бэкона не осталось денег, что сказалось на его биографии. Тогда
он начал изучать право и с 1582 года зарабатывал себе на жизнь
адвокатской деятельностью. Двумя годами позже он вошел в парламент, где
сразу стал заметной и значимой фигурой. Это привело к тому, что семь лет
спустя его назначили советником графа Эссекса, который в ту пору был
фаворитом королевы. После попытки государственного переворота, затеянной
Эссексом в 1601 году, Бэкон принимал участие в судебных заседаниях как
обвинитель.

Критикуя политику королевской семьи, Фрэнсис потерял покровительство
королевы и смог возобновить карьеру в полной мере только в 1603 году,
когда на троне оказался новый монарх. В том же году он стал рыцарем, а
через пятнадцать лет -- бароном. Еще через три года ему жаловали титул
виконта, но в тот же год ему предъявили обвинение во взяточничестве и
лишили поста, закрыв двери в королевский двор.

Несмотря на то, что он многие годы жизни посвятил юриспруденции и
адвокатуре, его сердце было отдано философии. Он разработал новые
инструменты мышления, раскритиковав дедукцию Аристотеля.

Мыслитель умер из-за одного из своих экспериментов. Он изучал, как холод
влияет на начавшийся гнилостный процесс и простудился. В возрасте
шестидесяти пяти лет он умер. Уже после его смерти было опубликовано
-незавершенным -- одно из главных произведений, написанных им: «Новая
Атлантида». В нем он предвидел многие открытия последующих веков,
основываясь на опытном знании.

\subsection{Общая характеристика философии Фрэнсиса Бэкона}

Фрэнсис Бэкон стал первым крупным философом своего времени и открыл
Эпоху Разума. Несмотря на то, что он был хорошо знаком с учениями
мыслителей, живших во времена древности и средневековья, он был убежден,
что путь, который они указывали -- ложный. Философы прошлых веков были
сосредоточены на нравственных и метафизических истинах, забывая о том,
что знания должны приносить практическую выгоду людям. Он
противопоставляет праздное любопытство, которому до сих пор служило
философствование, производству материальных благ.

Будучи носителем практичного англосаксонского духа, Бэкон не искал
знаний ради стремления к истине. Он не признавал подход к философии
через религиозную схоластику. Он считал, что человеку предначертано
господствовать над животным миром, и он должен исследовать мир
рационально-потребительски.

Силу он видел в знаниях, которые можно применить на практике. Эволюция
человечества возможна только через господство над природой. Эти тезисы
стали ключевыми в мировоззрении и философских учениях эпохи Возрождения.

\subsubsection{«Новая Атлантида»}

Одним из важнейших произведений Бэкона принято считать «Новую
Атлантиду», названную по аналогии с работой Платона. Написанию
утопического романа мыслитель посвятил время с 1623 по 1624 г. Несмотря
на то, что книга увидела свет незаконченной, она быстро приобрела
популярность в массах.

Фрэнсис Бэкон рассказал об обществе, которое управлялось одними учеными.
Это общество было найдено английскими мореплавателями, высадившимися на
острове посреди Тихого океана. Они обнаружили, что жизнь на острове
подчинена Дому Соломона -- организации, в которую входят не политики, а
ученые. Дом имеет своей целью расширить власть людей над миром живой
природы, чтобы она работала на них. В специальных помещениях проводились
эксперименты по вызову грома и молний, получения из ничего лягушек и
других живых существ.

Позднее, взяв за основу роман, создали реальные научные академии,
занимающиеся анализом и верификацией явлений. Примером такой организации
является Королевское общество поощрения науки и искусств.

Сейчас, некоторые рассуждения в романе могут показаться наивными, но в
эпоху, когда он был опубликован, изложенные в нем взгляды на научное
знание были популярны. Могущество человека казалось огромным, основанным
на божественных силах, и знания должны были помочь ему реализовать
власть над миром природы. Бэкон считал, что ведущими науками должны быть
магия и алхимия, которые могли бы помочь достичь этой власти.

Чтобы работать на человека, у экспериментальной науки должны быть
большие комплексы сооружений, двигатели, работающие с помощью воды и
воздуха, электростанции, сады, заповедники и водоемы, где можно было бы
проводить эксперименты. В результате их необходимо научиться работать
как с живой, так и неорганической природой. Большое внимание уделено
конструированию различных механизмов и машин, которые могут
передвигаться быстрее, чем пуля. Военные машины, орудия для сражений --
все это подробно описано в книге.

Только эпохе Возрождения свойственна такая сильная ориентация на
изменение мира природы. Как сторонник алхимии, Бэкон пытается
представить в «Новой Атлантиде», как можно вырастить растение без
использования семян, создать животных из воздуха, используя знания о
веществах и соединениях. Его поддержали такие видные деятели медицины,
биологии и философии, как Бюффон, Перро и Мариотт. В этом теория
Фрэнсиса Бэкона кардинально отличается от представлений Аристотеля о
неизменности и постоянстве видов животных и растений, имевших влияние на
зоологию нового времени.

Королевское общество поощрения науки и искусств, созданное на основе
описанных в «Новой Атлантиде» сообществ, много внимания уделяло световым
экспериментам -- как и ученые в романе Бэкона.

\subsubsection{«Великое восстановление наук»}

Фрэнсис Бэкон считает, что алхимия и магия могли бы послужить человеку.
Чтобы знание было общественно контролируемым, он отказывается от
магического. В «Великом восстановлении наук» он делает упор на то, что
настоящие знания не могут принадлежать частным лицам -- группе
«посвященных». Оно -- общедоступно и может быть понятно любому.

Бэкон также говорит о необходимости сведения философии к делам, а не
словам, как это было прежде. Традиционно, философия служила душе, а
Бэкон считает правильным покончить с этой традицией. Он отвергает
древнегреческую философию, диалектику Аристотеля, труды Платона.
Продолжая принятую в философии традицию, человечество не продвинется в
научном познании и лишь умножит ошибки прошлых мыслителей. Бэкон
отмечает, что в традиционной философии господствуют алогичность и
нечеткие понятия, которые кажутся выдуманными и не имеющими под собой
никакого реального основания.

В противовес описанному, Фрэнсис Бэкон предлагает истинную индукцию,
когда наука движется вперед постепенно, опираясь на промежуточные
аксиомы, контролируя достигнутые знания и проверяя их опытом.

\begin{quote}
Он выделяет два способа поиска истины:

\begin{enumerate}
\def\labelenumi{\arabic{enumi}.}
\item
  Через чувства и частные случаи -- к достижению самых общих аксиом,
  которые необходимо сужать и конкретизировать, соизмерять с уже
  доподлинно известными фактами.
\item
  Через чувства и частное -- к общим аксиомам, смысл которых не
  сужается, а расширяется до наиболее общих законов.
\end{enumerate}
\end{quote}

В результате такого деятельного познания, человечество придет к
научно-технической цивилизации, оставив в прошлом историко-литературный
тип культуры. Мыслитель считал необходимым привести в гармонию общение
ума и вещей. Для этого необходимо избавиться от бесплотных и смутных
понятий, которые употребляются в науках и философии. Затем, нужно заново
посмотреть на вещи и исследовать их, пользуясь современными, точными
средствами.

В «Великом восстановлении наук» Бэкон призывает современников сделать
упор на науки, применимые на практике и улучшающие жизнь человечества.
Это положило начало резкой смене ориентации в культуре Европы, когда
наука, видевшаяся многим праздной и подозрительной, стала важной и
престижной частью культуры. Большинство философов того времени
последовало примеру Бэкона и занялось наукой вместо схоластического
многознания, которое было оторвано от реальных законов природы.

\subsubsection{«Новый органон»}

Бэкон -- философ нового времени не только потому, что родился в эпоху
Возрождения, но и по своим взглядам на прогрессивную роль науки в
общественной жизни. В своем труде «Новый органон» он проводит сравнение
науки с водой, которая может падать с неба или происходить из недр
земли. Как вода имеет божественное происхождение и чувственную суть, так
и наука подразделяется на философию и теологию.

Он высказывает аргументы в пользу концепции двойственности истинного
знания, настаивая на четком разделении областей теологии и философии.
Теология изучает божественное, и Бэкон не отрицает, что все сущее --
творение Бога. Как предметы искусства говорят о таланте и силе искусства
своего творца, так и сотворенное Богом мало говорит о последнем. Фрэнсис
Бэкон заключает, что Бог не может быть объектом науки, а должен
оставаться только объектом веры. Это означает, что философия должна
прекратить попытки проникнуть в божественное и сконцентрироваться на
природе, познавая ее методом опытов и наблюдений.

Он критикует научные открытия, говоря, что они не соответствуют научному
прогрессу и отстают от жизненных потребностей общества. Это означает,
что вся наука как коллективное знание должна быть усовершенствована так,
чтобы она опережала практику, делая возможными новые открытия и
изобретения. Приведение в действие человеческого разума и управление
явлениями природы -- главная цель возрождения науки.

«Органом» содержит логические подсказки, говорящие, каким методом можно
соединить мышление и практику, чтобы они позволили овладеть силами
природы. Бэкон отвергает старый метод силлогизма как абсолютно
беспомощный и бесполезный.

\subsection{Теория борьбы с предрассудками -- идолами}

Фрэнсис Бэкон разработал собственную теорию о предрассудках,
господствующих над умом людей. Она говорит об «идолах», которых
мыслитель нового времени называет также «призраками» за их свойство
искажать действительность. Прежде, чем учиться познавать вещи и явления,
важно избавиться от этих идолов.

\subsubsection{Идолы «рода»}

К первой категории относятся идолы-призраки, присущие каждому человеку,
поскольку его ум и органы чувств несовершенны. Эти идолы заставляют его
сравнивать природу с самим собой и наделять ее теми же качествами. Бэкон
восстает против тезиса Протагора, говорящего, что человек представляет
собой меру всех вещей. Фрэнсис Бэкон заявляет, что ум человека, как
плохое зеркало, отражает мир в неправильном виде. В результате рождаются
теологическое миропонимание и антропоморфизм.

\subsubsection{Идолы «пещеры»}

Идолы-призраки «пещеры» порождаются самим человеком под влиянием условий
его жизни, особенностей воспитания и образования. Человек смотрит на мир
из покрова собственной «пещеры», то есть с точки зрения личного опыта.
Преодоление таких идолов заключается в использовании опыта, накопленного
совокупностью индивидов -- обществом, и постоянном наблюдении.

\subsubsection{Идолы «рынка»}

Поскольку люди постоянно контактируют друг с другом и живут плечо к
плечу, рождаются идолы «рынка». Их поддерживает использование речи,
старых понятий, обращение к словам, которые искажают суть вещей и
мышление. Чтобы избежать этого, Бэкон рекомендует отказаться от
словесной учености, которая оставалась в те времена от Средневековья.
Главная идея -- в изменении категорий мышления.

\subsubsection{Идолы «театра»}

Признаком идолов «театра» является слепая вера авторитетам. К таким
авторитетам философ относит старую философскую систему. Если верить
древним, то восприятие вещей исказится, возникнут предубеждения и
предвзятость. Чтобы победить таких призраков, следует обращаться к
современному опыту и изучать природу.

\subsubsection{Об основных идолах}

Все описанные «призраки» -- это препятствия к научному познанию,
поскольку благодаря им рождаются ложные представления, которые не дают в
полной мере понять мир. Преобразование наук по Бэкону невозможно без
отказа от перечисленного и опоры на опыт и эксперимент как часть знания,
а не на мысли древних.

Суеверия -- мыслитель нового времени тоже относит к причинам, которые
задерживают развитие научного знания. Теория двойственной истины,
описанная выше и разграничивающая изучение Бога и реального мира,
призвана оградить философов от суеверий.

Слабые продвижения в науке Бэкон объяснял отсутствием правильных
представлений об объекте познания и самой цели изучения. Правильным
объектом должна выступать материя. Философы и ученые должны выявлять ее
свойства и изучать схемы превращения ее из одного предмета в другой.
Человеческая жизнь должна обогащаться наукой за счет действительных
открытий, внедряемых в жизнь.

\subsection{Эмпирический метод научного познания Бэкона}

После того, как метод познания -- индукция --- определен, Фрэнсис Бэкон
предлагает несколько основных путей, по которым может идти
познавательная деятельность:

\begin{enumerate}
\def\labelenumi{\arabic{enumi}.}

\item
  «путь паука»
\item
  «путь муравья»
\item
  «путь пчелы»
\end{enumerate}

Под первым путем понимается получение знаний рационалистическим
способом, но это подразумевает оторванность от реальности, потому что
рационалисты опираются на собственные рассуждения, а не на опыт и факты.
Их паутина мыслей выткана из их собственных мыслей.

\begin{quote}
Империзм -- тот метод и способ познания, которому должен следовать
субъект чтобы достичь истинного знания.
\end{quote}

\subsubsection{«Путь паука»}

Под первым путем понимается получение знаний рационалистическим
способом, но это подразумевает оторванность от реальности, потому что
рационалисты опираются на собственные рассуждения, а не на опыт и факты.
Их паутина мыслей выткана из их собственных мыслей.

\subsubsection{«Путь муравья»}

По «пути муравья» идут те, кто принимает во внимание только опыт. Этот
метод получил название «догматический эмпиризм» и он основан на
информации, получаемой из фактов и практики. Эмпирикам доступная внешняя
картина знания, но не сущность проблемы

\subsubsection{«Путь пчелы»}

Идеальным методом познания является последний путь -- эмпирический.
Говоря кратко, идея мыслителя такова: для применения метода нужно
соединить воедино два других пути и убрать их недостатки и противоречия.
Знания выводятся из совокупности обобщенных фактов с использованием
доводов разума. Этот метод можно назвать эмпиризмом, в основе которого
лежит дедукция.

\subsection{Заключение}

Идеальным методом познания является последний путь -- эмпирический.
Говоря кратко, идея мыслителя такова: для применения метода нужно
соединить воедино два других пути и убрать их недостатки и противоречия.
Знания выводятся из совокупности обобщенных фактов с использованием
доводов разума. Этот метод можно назвать эмпиризмом, в основе которого
лежит дедукция.

Бэкон остался в истории философии не только как человек, положивший
начало развитию отдельных наук, но и как мыслитель, обозначивший
необходимость в изменении движения познания. Он был у истоков опытной
науки, задающей правильное направление теоретической и практической
деятельности людей.

\subsubsection{Итог}

Суть философии Фрэнсиса Бэкона -- эмпиризма --- заключается в том, что в
основе познания лежит исключительно опыт. Чем больше опыта (как
теоретического, так и практического) накопило человечество (и отдельный
человек), тем ближе оно к истинному знанию. Истинное знание, по Бэкону,
не может быть самоцелью. Главные задачи знания и опыта -- помочь
человеку добиваться практических результатов в его деятельности,
способствовать новым изобретениям, развитию экономики, господству
человека на природе. В связи с этим Бэконом был выдвинут афоризм,
который сжато выразил все его философское кредо: «Знание -- сила».

Бэкон выдвинул новаторскую идею, в соответствии с которой главным
методом познания должна стать индукция. Индукция -- логическое
умозаключение, идущее от частного положения к общему.

Под индукцией Бэкон понимал обобщение множества частных явлений и
получение на основе обобщения общих выводов. Метод индукции Бэкон
противопоставил методу дедукции, предложенному Декартом, согласно
которому истинное знание можно получить, опираясь на достоверную
информацию с помощью четких логических приемов. Достоинство индукции
Бэкона перед дедукцией Декарта -- в расширении возможностей,
интенсификации процесса познания.

Недостаток индукции -- ее недостоверность, вероятностный характер, в
каждом отдельном случае возникает необходимость в экспериментальной
проверке, подтверждении индукции. Путь преодоления главного недостатка
индукции (ее неполноты, вероятностного характера), по Бэкону, -- в
накоплении человечеством как можно большего опыта во всех областях
знания.

Определив главный метод познания -- индукцию, философ выделяет
конкретные пути, с помощью которых может проходить познавательная
деятельность. Это:

«Путь паука» -- получение знания из «чистого разума», то есть
рационалистическим путем. Данный путь игнорирует либо значительно
принижает роль конкретных фактов, практического опыта. Рационалисты
оторваны от реальной действительности, догматичны и, по Бэкону, «ткут
паутину мыслей из своего ума».

«Путь муравья» -- такой способ получения знаний, когда во внимание
принимается исключительно опыт, то есть догматический эмпиризм (полная
противоположность оторванного от жизни рационализма). Данный метод также
несовершенен. «Чистые эмпирики» концентрируют внимание на практическом
опыте, сборе разрозненных фактов, доказательств. Таким образом, они
получают внешнюю картину знания, видят проблемы «снаружи», «со стороны»,
но не могут понять внутреннюю сущность изучаемых вещей и явлений,
увидеть проблему изнутри.

«Путь пчелы» -- наиболее совершенный способ познания. Используя его,
философ-исследователь берет все достоинства «пути паука» и «пути
муравья» и в то же время освобождается от их недостатков. Следуя по
«пути пчелы», необходимо собрать всю совокупность фактов, обобщить их
(взглянуть на проблему «снаружи») и, используя возможности разума,
заглянуть «вовнутрь» проблемы, понять ее сущность.

Но Фрэнсис Бэкон не только показывает, какими путями должен происходить
процесс познания, но и выделяет причины, которые препятствуют человеку и
человечеству получить истинное знание. Данные причины философ
иносказательно называет «призраками» (или «идолами») и определяет четыре
их разновидности: идолы рода, пещеры, рынки и таетра.

Идолы рода и призраки пещеры -- врожденные заблуждения людей, которые
заключаются в смешивании природы познания с собственной природой. В
первом случае (идолы рода) речь идет о преломлении познания через
культуру человека (рода) в целом -- то есть человек осуществляет
познания, находясь в рамках общечеловеческой культуры, и это откладывает
отпечаток на итоговый результат, снижает истинность знания. Во втором
случае (идолы пещеры) речь идет о влиянии личности конкретного человека
(познающего субъекта) на процесс познания. В итоге личность человека
(его предрассудки, заблуждения -- «пещера») отражается в конечном
результате познания.

Идолы рынка и идолы театра -- приобретенные заблуждения.

Идолы рынка возникают из-за неправильного, неточного употребления
речевого, понятийного аппарата: слов, дефиниций, выражений.

Идолы театра возникают из-за влияния существующей философии на процесс
познания. Зачастую при познании старая философия мешает проявлять
новаторский подход, направляет познание не всегда в нужное русло. Исходя
из наличия четырех основных препятствий познания, Бэкон советует
максимально абстрагироваться от существующих «идолов» и получать
свободное от их влияния «чистое знание».

\section{Философия Джона Локка}

Джон Локк -- выдающийся философ 17 века, оказавший существенное влияние
на становление западной философии. До Локка западные философы основывали
свои взгляды на учении Платона и других идеалистов, согласно которым
бессмертная душа человека -- средство получения информации прямо из
Космоса. Ее наличие позволяет человеку родиться с готовым багажом
знаний, и ему уже не нужно было учиться.

\subsection{Биография}

Джон Локк родился в Англии, в 1632 г. Его родители придерживались
пуританских взглядов, которые будущий философ не разделял. Окончив с
отличием Вестминстеровскую школу, Локк стал преподавателем. Обучая
студентов греческому языку и риторике, он и сам продолжал учиться,
особое внимание уделяя естественным наукам: биологии, химии и медицине.

Локка интересовали и политико-правовые вопросы. Социально-экономическая
обстановка в стане подтолкнула его примкнуть к оппозиционному движению.
Локк становится близким другом лорда Эшли Купера -- родственником короля
и главой оппозиционного движения.

Стремясь принять участие в реформации общества, он бросает
преподавательскую карьеру. Локк переезжает в имение Купера и вместе с
ним и несколькими дворянами, разделявшими их революционные взгляды,
готовит дворцовый переворот.

Попытка переворота становится переломным моментом в биографии Локка. Она
оборачивается неудачей, и Локк вместе с Купером вынужден бежать в
Голландию. Здесь, в течение следующих нескольких лет, он посвящает все
свое время изучению философии и пишет свои лучшие труды.

\subsection{Познание как результат наличия сознания}

Локк полагал, что сознание -- это уникальная способность человеческого
мозга воспринимать, запоминать и отображать действительность. Родившийся
младенец -- чистый лист бумаги, у которого еще нет впечатлений и
сознания. Оно будет формироваться в течение жизни, базируясь на
чувственных образах -- впечатлениях, полученных посредством органов
чувств.

\begin{quote}
Согласно представлениям Локка, каждая идея -- продукт человеческой
мысли, появившейся благодаря уже существующим вещам.
\end{quote}

\subsection{Основные качества вещей}

К созданию каждой теории Локк подходил с позиции оценивания качеств
вещей и явлений. У каждой вещи есть первичные и вторичные качества.

К первичным качествам относятся объективные данные о вещи:

\begin{itemize}

\item
  форма
\item
  плотность
\item
  размер
\item
  количество
\item
  способность к движению
\end{itemize}

Эти качества присущи каждому объекту, и ориентируясь на них, человек
составляет свое впечатление о каждой вещи.

Ко вторичным качествам относятся впечатления, порождаемые органами
чувств:

\begin{itemize}

\item
  зрением
\item
  слухом
\item
  ощущениями
\end{itemize}

\begin{quote}
Взаимодействуя с предметами, люди получают информацию о них, благодаря
образам, возникающим на основе чувственных впечатлений.
\end{quote}

\subsection{Что такое собственность}

Локк придерживался концепции, согласно которой собственность --
результат труда. И она принадлежит человеку, который этот труд
вкладывал. Так, если человек посадил сад на земле дворянина, то
собранные плоды принадлежат ему, а не хозяину земли. Человек должен
владеть только той собственностью, которую он получил своим трудом.
Поэтому имущественное неравенство -- естественное явление и искоренить
его не получиться.

\subsection{Основные принципы познания}

Теория познания Локка базируется на постулате: «Нет ничего в уме, чего
раньше не было в ощущении». Он означает, что любое знание -- результат
восприятия, личного субъективного опыта.

По степени очевидности философ разделил знания на три вида:

\begin{itemize}

\item
  исходное -- дает знание об одной вещи
\item
  демонстративное -- позволяет строить умозаключения, сравнивая понятия
\item
  высшее (интуитивное) -- оценивает соответствие и несоответствие
  понятий непосредственно разумом
\end{itemize}

Согласно представлениям Джона Локка, философия дает человеку возможность
определить назначение всех вещей и явлений, развивать науку и общество.

\subsection{Педагогические принципы воспитания джентльменов}

Локк выделял три категории наук:

\begin{enumerate}
\def\labelenumi{\arabic{enumi}.}

\item
  Натуральная философия -- в нее входили точные и естественные науки.
\item
  Практическое искусство -- включает философию, логику, риторику,
  политические и социальные науки.
\item
  Учение о знаках -- объединяет все лингвистические науки, новые понятия
  и идеи.
\end{enumerate}

Согласно теории Локка о невозможности естественного получения знаний
через Космос и силы природы, человек осваивает точные науки только через
учение. Большинство людей не знакомы с основами математики. Им
приходится прибегать к напряженному умственному труду в течение долгого
времени, чтобы усвоить математические постулаты. Этот подход верен и для
освоения естественных наук.

Также мыслитель полагал, что понятия нравственности и морали передаются
по наследству. Поэтому люди не могут обучиться нормам поведения и стать
полноценными членами общества вне семьи.

Воспитательный процесс должен учитывать индивидуальные особенности
ребенка. Задача воспитателя -- постепенное обучение будущего джентльмена
всем необходимым навыкам, в которые входит освоение всего спектра наук и
норм поведения в обществе. Локк выступал за раздельное обучение детей из
благородных семей и детей простолюдинов. Последним следовало обучаться в
специально созданных рабочих школах.

\subsection{Политические взгляды}

Политические взгляды Джона Локка были антиабсолютическими: он выступал
за смену действующего режима и утверждение конституционной монархии. По
его мнению, свобода -- естественное и нормальное состояние индивида.

Локк отвергал представления Гоббса о «войне всех против всех» и полагал,
что изначальное понятие частной собственности сформировалось у людей
гораздо раньше установления государственной власти.

Торгово-экономические отношения должны быть построены на простой схеме
обмена и равенства: каждый человек ищет свою выгоду, производит товар и
обменивает его на другой. Насильственный отъем товара -- нарушение
закона.

Локк стал первым мыслителем, который принял участие в создании
первоучредительного государственного акта. Он разработал текст
конституции для Северной Каролины, который в 1669 г. одобрили и
утвердили члены народного собрания. Идеи Локка были новаторскими и
перспективными: плоть до сегодняшнего дня вся североамериканская
конституционная практика опирается на его учение.

\subsection{Права личности в государстве}

Основной правового государства Локк считал три неотъемлемых права
личности, которые есть у каждого гражданина независимо от его
социального положения:

\begin{itemize}

\item
  на жизнь
\item
  на свободу
\item
  на собственность
\end{itemize}

Конституция государства должна создаваться с оглядкой на эти права и
быть гарантом сохранения и расширения свободы человека. Нарушение права
на жизнь -- это любая попытка порабощения: насильственное принуждение
человека к какой-либо деятельности, присвоение его собственности.

\subsection{Религиозные взгляды}

Локк был стойким приверженцем идеи о разделении церкви и государства. В
своей работе «Разумность христианства» он описывает необходимость
веротерпимости. Каждому гражданину (за исключением атеистов и католиков)
гарантируется свобода вероисповедания.

Джон Локк считает религию не основой морали, а средством ее укрепления.
В идеале, человек должен руководствоваться не церковными догмами, а
самостоятельно прийти к широкой веротерпимости.

\subsection{Итог}

Джон Локк (1632--1704) -- английский философ --- был противником
подчинения знания откровению и утверждал, что вера не может иметь силу
авторитета перед лицом ясных и очевидных опытных данных.

Отвергая точку зрения о врожденных идеях, Локк полагал, что все наши
знания мы черпаем из опыта, ощущений. Люди не рождаются с готовыми
идеями. Голова новорожденного -- «чистая доска», на которой жизнь рисует
свои узоры -- знание. Локк утверждал: если бы идеи были врожденными, они
были бы известны одинаково как ребенку, так и взрослому, как идиоту, так
и нормальному человеку. «Нет ничего в уме, чего раньше не было в
ощущении», -- таков основной тезис Локка. Ощущения получаются в
результате действия внешних вещей на наши органы чувств. В этом состоит
внешний опыт. Внутренний опыт (рефлексия) есть наблюдение ума за своей
деятельностью и способами ее проявления. Однако Локк все же допускает,
что уму присуща некая спонтанная сила, не зависящая от опыта, что
рефлексия помимо внешнего опыта порождает идеи существования, времени,
числа. Отрицая врожденные идеи как внеопытное и доопытное знание, Локк
признавал наличие в разуме определенных задатков, или
предрасположенности к той или иной деятельности.

Он выделил три вида знания: исходное (чувственное, непосредственное),
дающее знание единичных вещей; демонстративное знание через
умозаключение, например, через сравнение и отношение понятий; высший вид
-- интуитивное знание, то есть непосредственная оценка разумом
соответствия и несоответствия идей друг другу.

Локк оказал огромное влияние не только на последующее развитие
философии, но и, наметив диалектику врожденного и социального, во многом
определил дальнейшее развитие педагогики и психологии.

\section{Филосфия Джорджа Беркла}

Джордж Беркли (1685--1753) -- наиболее значительный представитель
английского эмпиризма. C помощью эмпиризма и сенсуализма критиковал
материализм и защищал религию.

Беркли обратился к философии Локка и сделал из нее неожиданные выводы.
Концепция первичных и вторичных качеств дала ему для этого все
основания. Попробуем рассуждать вслед за Беркли. Вторичные качества --
субъективны, они существуют не в самом предмете, а в человеческом
сознании. Локк, правда, говорил, что эти вторичные качества зависят от
объективных, первичных, -- тех, что присутствуют в самих вещах. Но
Беркли в этом усомнился. С его точки зрения, все наоборот! Возьмем,
например, такое первичное качество, как форма. Мы воспринимаем формы
предметов только благодаря контрасту цветов, то есть «первичное»
качество зависит от «вторичного». Более того, подразделения на первичные
и вторичные качества вообще не существует -- заявил Беркли. Пользуясь
терминологией Локка, все качества -- вторичны, потому что все они
зависят от воспринимающего субъекта. Скажем, протяженность (величина).
Лодка, которая стоит рядом с нами на берегу, кажется вместительной и
большой. Но на горизонте ее величина сравнима с величиной мухи. Любая
информация о внешнем мире преломляется через наше восприятие, становится
субъективной. Ощущения всегда существуют только в сознании субъекта, они
«не похожи» на те предметы, которые эти ощущения вызывают. «На что может
быть похоже ощущение, кроме ощущения?» -- вполне обоснованно спрашивал
Беркли. Значит, делал он вывод, все ощущения -- субъективны, зависят от
нас.

То, что мы называем партой, машиной, мухой -- любой вещью, является лишь
совокупностью наших ошущений. Отсюда прямо следовал знаменитый тезис
Беркли: \emph{«Быть -- значит быть воспринимаемым».} Для человека
предмет и ощущение -- одно и то же, они не могут быть оторваны друг от
друга. Если для Локка мы знаем столько, сколько ощущаем, то для Беркли
существует столько, сколько мы ощущаем. По сути, эта позиция прямо вела
к субъективному идеализму в его самой крайней форме: есть я, мыслящий
субъект, а все остальное -- лишь мои ощущения, и я с уверенностью не
могу сказать, не снится ли мне этот мир, учителя, соседи за партой и
оценки в зачетке. Но такой крайний вывод (очень логичный, если исходить
из суждения «быть -- значит быть воспринимаемым») Беркли все-таки не
сделал. Он был епископом, глубоко верующим человеком, верил в
существование Бога и в созданный им мир. Поэтому от субъективного
идеализма он перекинул мостик к объективному идеализму, к тому, что
существую не только я со своими ощущениями, но и Бог, который создал
реальный мир вокруг меня.

То, что Беркли ввел в свою философскую систему Бога, спасло его от
многих трудностей и неудобств. Например, такие трудности возникали с
доказательством непрерывности существования вещей. Вот стол, за которым
я пишу -- он есть, потому что я его воспринимаю. Но если я отвернусь,
уйду в другую комнату и перестану его воспринимать? Он исчезнет? Нет,
говорит Беркли, он не исчезнет, потому что его могут воспринимать другие
люди. Но если я живу один, и некому больше созерцать мой заваленный
бумагами стол? Он исчезнет? Нет, отвечает Беркли, он будет существовать,
потому что ты будешь воспринимать его в своей памяти. Но если я поставлю
стол в кладовку, забуду о его существовании, он покроется толстым слоем
пыли и никто про него и не вспомнит? Как быть тогда? Вот тут-то и
пригодился Бог: если никто из людей не будет воспринимать стол, он все
равно будет существовать непрерывно, потому что есть Бог, всегда
воспринимающий все предметы в мире.

Введя в свою философию Бога, Беркли тем самым ввел в нее и весь
окружающий человека реальный мир. Вещи вокруг нас нам не снятся, они
действительно существуют благодаря божественному восприятию. Поэтому
наши ощущения дают нам какую-то информацию о мире, хотя и субъективную.
Но как узнать, насколько мое ощущение истинно, то есть насколько оно
соответствует реальности? Беркли предложил несколько критериев
истинности ощущений, но все они могут быть опровергнуты в рамках его же
системы, кроме последнего. Посмотрим, какие критерии истинности он
рассматривал:

\begin{enumerate}
\def\labelenumi{\arabic{enumi}.}

\item
  Истинные ощущения ярки и отчетливы. Но ведь и в кошмарном сне наши
  ощущения бывают чрезвычайно отчетливы и ярки!
\item
  Одновременность одинаковых восприятий у нескольких людей: не только я
  воспринимаю предмет, но и другие тоже. Но ведь бывают массовые
  психозы, да и заблуждения подчас разделяются большинством, а истина
  становится уделом одиночек. Вопрос об истине никогда не решался
  голосованием, в противном случае мы бы с вами до сих пор считали, что
  Земля -- в центре Вселенной, ведь большинство в эпоху Возрождения
  думало именно так, а Коперник, Галилей, Бруно были в явном
  меньшинстве.
\item
  Согласованность ощущений друг с другом: надо, чтобы ощущения не
  противоречили уже имеющимся. Но ведь согласованность представлений
  может быть и ложной. Теория Птолемея была согласованной, ее положения
  не противоречили друг другу, но, тем не менее, она не была истинной.
\item
  Принцип экономии мышления: надо отдавать предпочтение более простому
  объяснению. А каков критерий этой «экономности»? Для одного сложным
  является даже таблица умножения, а для другого и теория
  дифференциальных уравнений проста. Что «экономнее» -- мыслить атом
  неделимым или состоящим из электронов и протонов?
\item
  Соответствие нашего восприятия восприятию в божественном сознании. Тут
  уж и сказать нечего -- в рамках мировоззрения Беркли этот подход
  является окончательным, решающим аргументом: если мое представление
  совпадает с представлением Бога, оно истинно. Только вот как об этом
  узнать?
\end{enumerate}

Получается, что человек не знает ничего, кроме своих ощущений. Судить же
об их истинности он не может. Ему остается надеяться, что его
представления о том, что происходит в мире, совпадают с реальным,
заданным Богом ходом вещей. Поэтому наука -- лишь удобная для людей
иллюзия, но никакого решающего подтверждения или опровержения истинности
тех или иных теорий мы никогда не сможем получить: любое подтверждение
все равно будет базироваться на наших субъективных ощущениях.

Вот таким образом Беркли показал, что сенсуализм, доведенный до его
логического предела, приводит к субъективному идеализму. Сам Беркли
субъективного идеализма постарался избежать, чего не сделал его
последователь -- Давид Юм.

\subsection{Итог}

Беркли полагал, что существование вторичных и первичных качеств
предметов обусловлено нашим восприятием. Он считал, что все качества
предметов являются вторичными, полагая, что и первичные качества имеют
тот же характер, что и вторичные, ибо такие качества, как протяжение, не
являются объективными, а зависят от нашего восприятия, сознания. Так,
величина предметов -- это не нечто объективное, а определяется тем, что
предмет нам кажется то большим, то маленьким. Иными словами, величина
предметов -- это результат нашего опытного заключения, которое опирается
на органы чувств.

Так же Беркли рассуждал и при рассмотрении понятия материи. Он полагал,
что существование абстрактно-общих идей невозможно, так как при
восприятии в нашем уме возникает конкретное впечатление, конкретный
образ, но не может быть никакой общей идеи. Если мы воспринимаем
треугольник, то это конкретный треугольник, а не какой-то абстрактный,
не обладающий специфическими чертами. Таким же образом, согласно Беркли,
невозможно образовать абстрактные общие идеи человека, движения и т. д.

Тем самым он и не признавал существования понятия материи как
абстрактной идеи, материи как таковой.

Из этих рассуждений он переходил к отрицанию объективного существования
вещей. Так как существование качеств вещей обусловлено нашим
восприятием, а субстанция -- это носитель свойств, качеств, то, значит,
все вещи и предметы окружающего мира, которые образуются из свойств,
являются лишь восприятиями наших органов чувств. Для Беркли «быть --
значит быть воспринимаемым» (\emph{«esse est percipi»}).

Беркли также утверждает, что вещи продолжают существовать в силу того,
что в тот момент, когда мы их не воспринимаем, их воспринимает другой
человек. Так, Беркли, с одной стороны, утверждает, что вещи, или идеи,
по его терминологии, не существуют, с другой -- что они продолжают
существовать в нашей мысли.

\section{Филосолия Рене Декарта}

Рене Декарт (1596--1650) -- французский математик и философ. Сводил роль
опыта к простой практической проверке данных интеллекта. Источником
познания и критерием его истинности признавал разум (мышление).

Другое направление философии Нового времени -- рационализм -- получило
свое обоснование в трудах французского философа и математика Рене
Декарта. Он тоже поставил вопрос о необходимости создания совершенно
нового метода научного познания, способного дать достоверное,
доказательное знание о мире. Два основоположника Новой философии -- Р.
Декарт и Ф. Бэкон -- были равно уверены в необходимости полного
пересмотра накопленных ранее знаний о мире и построения науки заново,
уже на основе нового научного метода. Но метод этот они понимали
совершенно по-разному.

\subsection{Биография}

Рене Декарт родился в семье небогатого дворянина на юге Франции. Спустя
несколько дней после появления Рене на свет его мать умерла, а мальчик,
несмотря на пессимистические прогнозы врачей, выжил. В возрасте 8 лет он
был отправлен отцом в иезуитский колледж, где пристрастился к чтению
книг. Там, писал Декарт позднее, он убедился, сколь мало мы знаем, хотя
в математике дела в этом смысле обстоят лучше, чем в любой другой
области; он понял также, что для обнаружения истины необходимо
отказаться от опоры на авторитет, принадлежащий традиции или
сегодняшнему дню, и не принимать ничего на веру, пока оно не будет
окончательно доказано. После колледжа Рене записался добровольцем в
армию сначала французского, затем баварского короля. В качестве
вольнонаемного офицера он кочевал по Европе -- побывал в Германии,
Австрии, Италии и, по-видимому, в Дании, Польше и Венгрии. Однажды, как
вспоминал потом сам Декарт, сидя в одиночестве в сельском доме в
Баварии, он сформулировал все основные положения своего философского
учения, а также идею сведения физики к геометрии, а геометрии -- к
алгебре (благодаря этой гениальной догадке Декарта физика и математика
приобрели современный вид). После «армейского периода» жизни Декарт
возвратился в Париж и приступил к написанию своих трудов.

Приступил к наш Философские идеи Декарта сразу нашли много поклонников,
но и вызвали неудовольствие церкви. Декарт вынужден был издавать свои
произведения анонимно, а в 1629 г. переехать в более либеральную
Голландию. Но и здесь он натолкнулся на непонимание, хотя открыто и не
высказывал своих взглядов на сотворение и развитие Вселенной, которые
совпадали со взглядами осужденного Галилея. Во избежание неприятностей
Декарт принял приглашение шведской королевы Христины и переехал в
Стокгольм, где он рано по утрам давал уроки философии королеве, работал
над уставом Шведской Академии наук. Суровый климат Швеции не подошел
родившемуся на юге Декарту, он заболел воспалением легких и умер зимой
1650 г. Через 16 лет после его смерти по требованию французского
правительства останки ученого были перевезены в Париж и торжественно
погребены в Пантеоне, однако произнесение речей над могилой было
запрещено, а последователи философии Декарта подвергались преследованиям
до конца 17 века. Более того, в 1663 году сочинения Декарта были внесены
Ватиканом в «Индекс запрещенных книг».

\subsection{Мыслю, следовательно, существую}

Декарт тоже задался вопросом о методе: нельзя ли создать общий метод
исследования, позволяющий решать все проблемы, возникающие перед
человеческим разумом? Возможность нахождения такого общего метода Декарт
выводил из признания родства всех наук. Он рассматривал все научные
дисциплины как ветви одного дерева, вырастающего из одного корня --
метафизики (или, как мы сегодня бы сказали, -- философии). Стволом этого
«дерева» является физика, а ветвями -- конкретные науки.

В отличие от Бэкона, Декарт скептически относился к возможности
построения новой науки на основе опыта. Чувственный опыт не способен
дать достоверное знание, ибо мы часто сталкиваемся с иллюзиями и
галлюцинациями, а мир, воспринимаемый нами с помощью чувств, может
оказаться сном. Но даже если мы поверим в истинность собственных
ощущений, очевидно, что опытная наука строится на основе индукции.
Индукция же, как мы с вами уже выяснили, в большинстве случаев не дает
100 \% гарантии истинности выводов. Например, для того, чтобы проверить
истинность предложения «Все тела расширяются при нагревании», необходимо
подвергнуть нагреванию все тела, что невозможно сделать. Значит,
индукция в такого рода общих суждениях всегда неполна, и есть
вероятность того, что мы встретим тело, которое при тепловом воздействии
на него не расширится. Индуктивное знание -- вероятностное, а Декарт
мечтал о науке, дающей абсолютную истину.

Поэтому Декарт считал дедукцию более подходящим методом для научного
познания. Но и здесь возникали свои трудности: дедуктивное рассуждение
есть выведение заключений из посылок, и до тех пор, пока у нас нет
достоверных посылок, мы не можем рассчитывать на достоверность
заключений. А где же взять посылки, из которых мы можем дедуктивно
вывести следствия? Декарт полагал, что если бы мы знали абсолютно
истинные первые принципы, то могли бы вывести из них все остальное
знание. Поэтому поиск достоверного знания составлял важную часть его
философии науки.

Первой ступенью в поисках абсолютно достоверных посылок у Декарта
является сомнение. Прежде чем начинать исследование чего бы то ни было,
надо усомниться во всем, что вы знаете. Такое фундаментальное сомнение
помогает очистить науку от веры в ложные авторитеты, критически
отнестись к накопленному знанию. Но что является для человека
несомненным? Мы можем сомневаться в истинности изложенных в учебниках
теорий, в словах других людей, в правильности результатов эксперимента.
Да что там эксперимент, мы можем сомневаться даже в том, что существует
мир вокруг нас: может, это иллюзия, окружающий мир нам снится, а на
самом деле его нет. В чем же нельзя сомневаться? Только в самом факте
нашего сомнения: мы не можем сомневаться, что мы сомневаемся. Но
сомнение -- это акт мышления. А может ли мыслить то, что не существует?
Он утверждал что нет. Значит, \textbf{я мыслю, следовательно, существую}
(на латыни \emph{cogito ergo sum}). Абсолютную достоверность Декарт
обнаружил только в знании о своем собственном существовании. Декарт
рассуждал так: у меня нет достоверного знания о существовании моего
тела, ибо я мог бы быть животным или покинувшим тело духом, которому
снится, что он человек; однако мой разум, мой опыт существуют несомненно
и достоверно. Содержание мыслей или убеждений может быть ложным и даже
абсурдным; однако сам факт мышления абсолютно достоверен.

Получается, что мышление -- несомненная реальность, достоверность
которой нам дана интуитивно. Но сознание не бывает «пустым», это всегда
сознание о чем-то. Среди многочисленных мыслей, представлений, суждений,
составляющих содержание сознания, некоторые тоже обладают интуитивной
самоочевидностью. Именно такие самоочевидные, ясные, отчетливые идеи и
должны быть положены в основу научных рассуждений. Декарт называл такие
самоочевидные суждения врожденными идеями.

Его знаменитое «мыслю, следовательно, существую (cogito ergo sum)» --
одна из таких врожденных идей, сомневаться в которых невозможно. В число
врожденных идей Декарт включил существование мира, Бога, чисел,
длительности и др. Он приводил примеры врожденного знания: «Целое больше
своей части», «Линия состоит из точек», «Все тела протяженны» и т. д.
Наличие таких врожденных идей было обязательным для построения Декартом
науки на основе дедукции, ведь дедукция -- выведение из чего-то, значит,
необходимо знание, из которого можно делать выводы. Тогда строение науки
будет похоже на строение геометрии Евклида: из самоочевидных аксиом
выводятся теоремы. Кстати, Декарт был уверен, что евклидовы аксиомы тоже
являются врожденными идеями -- они просты, самоочевидны, ясны нам
интуитивно.

Таким образом, Декарт сформулировал правила своего метода так:

\begin{enumerate}
\def\labelenumi{\arabic{enumi}.}

\item
  Включать в свои суждения только то, что представляеется моему уму
  столь ясно и отчетливо, что никоим образом не сможет дать повод к
  сомнению.
\item
  Разделять всякую проблему на столько частей, на сколько возможно.
\item
  Располагать мысли в определенном порядке, начиная с простого и
  переходя к сложному.
\item
  Делать всюду перечни настолько полные и обзоры столь всеохватывающие,
  чтобы быть уверенным, что ничего не пропущено.
\end{enumerate}

Итак, по мнению Декарта, начав с сомнения и руководствуясь затем этими
довольно простыми правилами, можно построить истинные научные теории.
Конечно, правила Декарта кажутся слишком простыми и очевидными для того,
чтобы произвести настоящую революцию в познании. Но поразительный факт!
Сам Декарт добился столь впечатляющих успехов, применяя свой метод в
различных областях науки, что вот уже несколько столетий исследователи
гадают: так ли все просто и понятно с методом великого ученого? Может
быть, за внешней простой (и даже примитивностью) скрывается какая-то
загадка, «волшебный ключик», с помощью которого Декарт сформулировал
законы инерции, преломления и отражения лучей, развил теорию оптических
поверхностей; отверг представление о теплоте как жидкости (теорию
«теплорода») и создал, по-сути, кинетическую теорию теплоты; разработал
законы сохранения и измерения движения; дал жизнь новой области
математики -- аналитической геометрии; чрезвычайно много сделал для
алгебры переменных величин; заложил основы для изучения условных
рефлексов у животных; высказал идею об относительности движения; ввел
использование в математике и физике оси координат X, Y, Z; создал
теорию, объясняющую образование и движение небесных тел вихревым
движением частиц материи («вихри Декарта») и т. д.

\subsubsection{Мыслящая и протяженная субстанции}

Из положения «мыслю, следовательно, существую» Декарт вывел
существование духовной, или, как он ее назвал, «мыслящей» субстанции. Но
Декарт отнюдь не считал, что человек -- это «мозги», которые сами
придумали окружающий нас мир. Значит, надо было ввести в философскую
систему не только мыслящую, но и материальную субстанцию, природу.
Декарт это сделал благодаря ссылке на Бога. Раз наше сознание обладает
содержанием и в нем содержатся представления о журчащем ручье, голубом
небе, других людях и животных, то все это должно существовать, это не
иллюзия: ведь Бог не может быть обманщиком. Он наделил людей разумом не
для того, чтобы водить их за нос, а для того, чтобы дать инструмент
познания мира. Поэтому содержание нашего познания свидетельствует о том,
что не только я, мыслящий субъект, существую, но существует и мир вокруг
меня. Этот телесный мир Декарт описывает как протяженный -- то есть,
имеющий объем, величину. Протяженность не только Декартом, но и
большинством мыслителей того времени воспринималась как неотъемлемое
качество материального, телесного мира. Поэтому в философии Декарта
указаны две независимые друг от друга субстанции -- мыслящая и
протяженная (телесная и материальная).

\subsection{Дуализм Декарта}

Неразрешимая задача: его философская система не о материализме и
идеализме. Поэтому Декарта считают одним из наиболее известных
представителей дуализма\footnote{Дуализм (от лат. dualis --
  двойственный) --- философская позиция, признающая наличие двух
  противоположных начал бытия, субстанций (материальной и духовной).} в
философии. В философии Декарта мы видим дуализм души и тела, «мыслящей»
и «протяженной» субстанции. Как же они взаимодействуют? Эту сложную
проблему разрешить Декарт так и не смог. С одной стороны, субстанции эти
-- разные, обладающие столь различными характеристиками, что
взаимодействие между ними невозможно. С другой стороны, как быть с
человеком? Ведь в нем-то есть и телесная, и мыслящая составляющая,
причем они взаимодействуют! Декарт не смог дать удовлетворительного
объяснения, как это происходит. Впрочем, и до сих пор так называемую
психофизиологическую проблему ученые до конца не решили: как мысль
возникает из химических реакций в мозгу? Каким образом материальное
влияет на идеальное и наоборот? Декарт попытался дать описание
механизмов такого взаимодействия, но это было ответом на частные вопросы
без решения общего, главного.

Взаимодействие тела и души в человеке объяснялось Декартом
механистически. Душа, по его мнению, расположена в шишковидной железе
мозга (физиологи тогда не знали ее назначения в человеческом организме,
вот Декарт и приписал ей функцию «седалища» души), которая воздействует
на нервы, т.е. -- на мышцы и другие органы тела, в результате чего люди
ходят, двигают руками, поворачивают голову и т. д. в соответствии со
своим желанием. Есть и обратное воздействие: окружающий телесный мир
действует на нас, воздействие передается через нервы в шишковидную
железу и вызывает какую-то нашу реакцию. По сути, человек действует у
Декарта как сложный механизм. Декартовская схема воспринимается сегодня
как сильно упрощенное объяснение человеческой активности, но для того
времени такой подход был вполне передовым.

\subsection{Деизм в объяснении происхождения Вселенной}

Когда мы рассказывали вам о мировоззрении Средневековья, то
характеризовали его как теизм. Для философии эпохи Возрождения был
характерен пантеизм. А вот в Новое время появляется новый подход в
понимании роли бога в мироздании -- деизм\footnote{Деизм (от лат. deus
  -- бог) --- концепция, сторонники которой признают, что Бог сотворил
  мир, но затем мир развивается без участия и вмешательства Бога.}. Бог
-- как часовщик, который завел часы, но дальше ему совсем не нужно
подталкивать их стрелки. У Декарта Бог дал природе законы движения,
после чего мир стал развиваться по этим законам совершенно естественно.
Поэтому вера в чудеса, действенность молитв, жертвоприношений и прочее
подвергалась Декартом сомнению.

Деизм Декарта особенно ярко проявился в его теории происхождения
Вселенной. В первоначальном хаосе частиц возникли круговые (вихревые)
движения (движение, видимо, было привнесено Богом, он дал своеобразный
«первотолчок» миру). В процессе такого вихревого движения образовались
корпускулы (соединения частиц), а затем и три вида этих корпускул --
твердое вещество, жидкое вещество, газообразное вещество. Газообразное
(«огненное») вещество сконцентрировалось в центре вихревых потоков, из
него возникли звезды. Из более твердых корпускул возникли планеты. Как
вы видите, Богу у Декарта отведено довольно скромное место в истории
возникновения Вселенной: он не строит мир, как строитель дом, а лишь
дает миру движение и вещество. Кстати, гипотеза Декарта о роли вихревых
потоков в космических процессах оказалась очень плодотворной для
последующего развития науки. \#\# Вывод Подводя итог нашему разговору о
Декарте, надо сказать, что Декарт создал философию, которая, несмотря на
запреты, довольно быстро стала доминировать в Европе. Строгость и
ясность мысли, тесная связь с естествознанием, разработка идеала
научного знания способствовали популярности философских идей
французского мыслителя. Если же оценивать вклад Декарта с точки зрения
теоретического диалога между эмпиризмом и рационализмом в европейской
мысли, то здесь он является признанным основателем рационализма Нового
времени, продолжавшего линию Сократа и Платона.

\subsection{Итог}

Он стремился разработать универсальный дедуктивный метод для всех наук,
исходя из наличия в человеческом уме врожденных идей, которые во многом
определяют результаты познания. Декарт отталкивался от «принципа
очевидности», при котором всякое знание должно быть проверено, а все
суждения, принятые на веру (например, обычаи, примеры как традиционные
формы передачи знаний) не принимаются в расчет.

По Декарту, научное знание должно было быть построено как единая
система, в то время как до него оно было лишь собранием случайных истин.
Самое первое достоверное суждение, по Декарту, -- мыслящая субстанция.
Она открыта нам непосредственно. Декарт определяет эту первоначальную
субстанцию как вещь, которая для своего существования не нуждается ни в
чем, кроме самой себя. В строгом смысле подобной субстанцией может быть
только Бог, который «вечен, вездесущ, всемогущ, источник всякого блага и
истины, творец всех вещей\ldots» Мыслящая и телесная субстанции
сотворены Богом и им поддерживаются. Разум Декарт рассматривает как
конечную субстанцию, «вещь несовершенную, неполную, зависящую от чего-то
другого и\ldots{} стремящуюся к чему-то лучшему и большему, чем Я
сам\ldots»

По Декарту, мир -- механизм, наука о нем -- механика, а процесс познания
есть конструирование определенного варианта машины мира из простейших
начал, которые находятся в человеческом разуме. В качестве инструмента
Декарт предложил свой метод, в основу которого легли следующие правила:

\begin{itemize}
\item
  начинать с простого и очевидного
\item
  путем дедукции получать более сложные высказывания
\item
  действовать таким образом, чтобы не упустить ни одного звена
  (непрерывность цепи умозаключений)
\end{itemize}

Для этого нужны интуиция, которая усматривает первые начала, и дедукция,
которая дает следствия из них.


\end{document}
