\documentclass[a5paper, 12dd, twoside]{article}
% !TeX spellcheck = ru_RU
%Настройка страниц

\usepackage[left=2.5cm, right=1.5cm, top=2.5cm, bottom=2.5cm]{geometry}
%twoside, openany,%

%Настройка языка и отображения
\usepackage{cmap}				%Ссылки в PDF
\usepackage[T2A]{fontenc}		%Шрифты
\usepackage[utf8]{inputenc}		%Кодировка
%\usepackage{lmodern}			%Хз, не трогать
\usepackage[russian]{babel}		%Грамматика
\usepackage{graphicx}			%Пикчи
\DeclareGraphicsExtensions{.pdf,.png,.jpg}
\frenchspacing					%Отключает большой пробел между предложениями

\usepackage{indentfirst}		%Красная строка
\setlength{\parindent}{1.25cm}	%Настройка отступа красной строки для шрифта 14pt, по умолчанию 15pt
\linespread{1.25} 				%Межстрочный интервал

\usepackage{parskip} 			%Интервал между абзацами
\setlength{\parindent}{1cm} 	%Настройка интервала между абзацами, по умолчанию будет 0

\usepackage{enumitem}
\setlist{noitemsep}


\pagestyle{plain} 				%Использование стандартных колонитулов

%\usepackage{fancyhdr}			%Колонтитулы
%\pagestyle{fancy} 				%Использование кастомных колонитулов
%\fancyhf{} 					%Отчистить все колонтитулы
%\lhead{} 						% левый верхний колонтитул
%\chead{} 						% центральный верхний
%\rhead{} 						% правый верхний
%\lfoot{} 						% левый нижний
%\cfoot{\thepage} 				% центральный нижний
%\rfoot{} 						% правый нижний


\usepackage{listings}
% настройка подсветки кода и окружения для листингов
%\usemintedstyle{colorful}
%\newenvironment{code}{\captionsetup{type=listing}}{}
\hyphenation{
    ко-леб-лю-ще-го-ся 
    им-пульс-но-го 
    ус-та-нав-ли-ва-ют 
    не-пос-то-ян-но-го 
    ха-рак-те-рис-ти-ки
    про-из-вод-ствен-но-го
    пос-ле-ду-ю-щей}

\title{Практическое занятие №3\\<<Оценка воздействия вредных веществ, содержащихся в воздухе>>}
\author{Краснов Александр МР--19}

\begin{document}
\maketitle
\tableofcontents
\clearpage

\subsubsection*{Цель работы}
Ознакомиться с общими сведениями о вредных газах и парах экспресс~-- методом определения их содержания в воздухе рабочей зоны, конструкциями и правилами пользования приборами, используемыми при этом методе, научиться производить оценку загазованности и упрощенные расчеты проветривания производственных помещений.

\section{Теоретические сведения}
\subsection{Термины и определения}
Вредное вещество, которое при контакте с организмом человека в случае нарушения требований безопасности может вызывать производственные травмы, профессиональные заболевания или отклонения в состоянии здоровья, обнаруживаемые современными методами, как в процессе работы, так и отдаленные сроки жизни настоящего и последующего поколений (ГОСТ 12.1007-76).

Вредные вещества, которые, проникая в организм человека через органы дыхания, желудочно--кишечный тракт и кожные покровы, вызывают нарушение его жизнедеятельности, называются ядовитыми или токсичными веществами. Рабочая зона - пространство высотой до 2 м над уровнем пола или площадки, на которых находятся места постоянного или временного пребывания работающих.

Предельно-допустимая концентрация (ПДК) вредных веществ в воздухе рабочей зоны --- концентрация, которая при ежедневной работе в течение 8 часов или при другой продолжительности, но не более 41 часов в неделю, в течение всего рабочего стажа не может вызывать заболеваний или отклонений в состоянии здоровья, обнаруживаемых современными методами исследований в процессе работы или в отдаленные сроки жизни настоящего и последующих поколений (ГОСТ 12.1.005-88).

\subsection{Действие ядовитых веществ на организм}
Во многих отраслях промышленности при ведении технологических процессов в воздухе рабочей зоны выделяются вредные различные газы и пары. Например, в горной -- окись углерода, оксиды азота, метан, альдегиды и др; в металлургической -- сернистый газ, окись углерода, оксиды азота, аэрозольные оксиды токсичных металлов и пр.; в нефтегазовой -- сероводород, сернистый газ, окись углерода, углеводороды, оксиды азота, пары сырой нефти и её фракций; в машиностроительной -- туманы масел и кислот, пары растворителей, аммиак, оксиды азота; в радиоэлектронной и приборостроительной -- пары токсичных металлов, кислот растворителей и т. д.

При несовершенной организации труда и отсутствии соответствующих профилактических мер, все эти вредные газы и пары могут вызывать профессиональные отравления, которые подразделяются на острые и хронические. Первые из них возникают за короткое время под воздействием ядов большой дозы, вторые -- в результате систематического отравления ядами малой дозы за длительное время.

Исход отравления зависит от многих таких факторов, как; токсичность (вид и физико-химические свойства), концентрация, длительность воздействия на организм и путь проникновения в него промышленных ядов; состояние и особенность организма человека; метеорологические условия окружающей среды.

Повышенная чувствительность наблюдается у детей и подростков, а также у людей после перенесенных болезней. Чем выше температуры тела человека, тем он восприимчивее к действию ядов. Люди, страдающие ожирением и отеками, также более подвержены воздействию токсичных веществ.

Температура, влажность и барометрическое давление воздуха могут усиливать или ослаблять эффект воздействия вредных газов и паров. При высокой температуре воздуха расширяются кожные сосуды, увеличивается потовыделение, учащается дыхание и повышается кроваток. В результате ускоряется проникновение ядов в организм. Она также усиливает скорость испарения и летучесть токсичных веществ, что способствует росту загрязненности ими воздуха. Опасность отравления при работе со многими вредными веществами возрастает в жаркое время года, а со свинцом -- в холодные месяцы. Влажность воздуха повышает токсичность некоторых веществ (соляной кислоты, фтористого водорода и др).

Промышленные яды проникают в организм человека тремя путями: через органы дыхания, желудочно--пищеварительный тракт и кожный покров. Попавшие внутрь организма с вдыхаемым воздухом токсичные вещества быстро всасываются слизистой оболочкой дыхательных путей и огромной поверхностью легочных альвеол (около \(130 m^2\)); оттуда усваиваются потоками крови и разносятся ими по всему организму. Большинство отравлений (до 95\%) происходит этим наиболее опасным путем. Через пищеварительный тракт вредные вещества могут попасть в организм вместе с загрязненной пищей и водой. Здесь опасны лишь те яды, которые растворяются в желудке (в воде, жирах и желудочном соке), всасываются стенками желудка и кишечника и попадают в кровь. Токсичный эффект этого пути отравления существенно ниже, чем через органы дыхания, т. к. вредные вещества попадают в кровь через печень, где подвергаются частичному обезвреживанию. Через кожный покров попадают внутрь организма только некоторые, растворимые в жидкостях и жирах органов, яды. Тем не менее, опасность отравления здесь выше, чем при пищеварительном отравлении, поскольку токсичные вещества попадают прямо в большой круг кровообращения, минуя печень.



\section{Практическая часть}
\subsubsection*{Входные данные}
\(N = 10\) -- число учтенных несчастных случаев за анализируемый период

\(R = 4000\) -- среднесписочное число работающих за этот же период

\(D = 7\) -- общее число дней нетрудоспособности (кроме несчастных случаев с летальным исходом)

\subsubsection*{Коэффициент частоты травматизма}
\[K = \frac{1000N}{R} = \frac{1000 \times 10}{4000} = 2.5\]
\subsubsection*{Коэффициент тяжести травматизма}
\[KT = \frac{D}{N} = \frac{7}{10} = 0.7\]
\subsubsection*{Коэффициент потерь производства}
\[KP = K \times KT = \frac{1000 D}{R} = \frac{1000 \times 7}{4000} = 1.75\]

%\section{Ответы на контрольные вопросы}
%\begin{enumerate}
%    \item {\bfseries Какие нарушения, приведшие к несчастным случаям, объединены в технические причины?}
%    \item []
%    \item {\bfseries Какие нарушения, приведшие к нечастным случаям, объединены в организационные причины?}
%    \item []
%    \item {\bfseries Какие нарушения, приведшие к несчастным случаям, объединены в санитарно-гигиенические мероприятия?}
%    \item []
%    \item {\bfseries По каким признакам несчастные случаи распределяются по группам, от чего это зависит?}
%    \item []
%    \item {\bfseries Что выявляет топографический метод при исследовании несчастных случаев?}
%    \item []
%    \item {\bfseries ри анализе причин травматизма монографическим методом, что выявляется дополнительно в процессе исследования?}
%    \item []
%    \item {\bfseries Что можно выявить при исследовании несчастных случаев, используя коэффициент частоты и коэффициент тяжести, при статистическом методе анализа несчастных случаев?}
%    \item []
%\end{enumerate}


\subsection*{Вывод}
В ходе выполнения данной практической работы я ознакомился с краткими теоретическими сведениями о методах оценки состояния техники безопасности на производственном объекте, а также выполнил расчет и выполнение анализа причин и уровня травматизма статистическим методом.
\end{document}