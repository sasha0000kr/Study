\documentclass[a5paper, 12dd, twoside]{article}
% !TeX spellcheck = ru_RU
%Настройка страниц

\usepackage[left=2.5cm, right=1.5cm, top=2.5cm, bottom=2.5cm]{geometry}
%twoside, openany,%

%Настройка языка и отображения
\usepackage{cmap}				%Ссылки в PDF
\usepackage[T2A]{fontenc}		%Шрифты
\usepackage[utf8]{inputenc}		%Кодировка
%\usepackage{lmodern}			%Хз, не трогать
\usepackage[russian]{babel}		%Грамматика
\usepackage{graphicx}			%Пикчи
\DeclareGraphicsExtensions{.pdf,.png,.jpg}
\frenchspacing					%Отключает большой пробел между предложениями

\usepackage{indentfirst}		%Красная строка
\setlength{\parindent}{1.25cm}	%Настройка отступа красной строки для шрифта 14pt, по умолчанию 15pt
\linespread{1.25} 				%Межстрочный интервал

\usepackage{parskip} 			%Интервал между абзацами
\setlength{\parindent}{1cm} 	%Настройка интервала между абзацами, по умолчанию будет 0

\usepackage{enumitem}
\setlist{noitemsep}


\pagestyle{plain} 				%Использование стандартных колонитулов

%\usepackage{fancyhdr}			%Колонтитулы
%\pagestyle{fancy} 				%Использование кастомных колонитулов
%\fancyhf{} 					%Отчистить все колонтитулы
%\lhead{} 						% левый верхний колонтитул
%\chead{} 						% центральный верхний
%\rhead{} 						% правый верхний
%\lfoot{} 						% левый нижний
%\cfoot{\thepage} 				% центральный нижний
%\rfoot{} 						% правый нижний


\usepackage{listings}
% настройка подсветки кода и окружения для листингов
%\usemintedstyle{colorful}
%\newenvironment{code}{\captionsetup{type=listing}}{}
\hyphenation{
    ко-леб-лю-ще-го-ся 
    им-пульс-но-го 
    ус-та-нав-ли-ва-ют 
    не-пос-то-ян-но-го 
    ха-рак-те-рис-ти-ки
    про-из-вод-ствен-но-го
    пос-ле-ду-ю-щей}

\title{Практическое занятие №2\\<<Использование средств индивидуальной и групповой защиты>>}
\author{Краснов Александр МР--19}

\begin{document}
\maketitle
\tableofcontents
\clearpage

\subsubsection*{Цель работы}
Ознакомиться со средствами индивидуальной и групповой защиты.

\section{Теоретические сведения}
Выполнение работ в сопряжено с неблагоприятным влиянием окружающей среды на организм работающих: запыленности воздуха, шума и вибрации.

Для предохранения и защиты организма человека от вредного воздействия шахтной окружающей среды применяют различные средства.

В качестве индивидуальных средств защиты органов дыхания применяют противопылевые респираторы. В соответствии с требованиями Правил безопасности их следует рассматривать как вспомогательное средство профилактики пневмокониоза при обязательном осуществлении комплекса основных противопылевых мероприятий на всех технологических процессах добычи полезных ископаемых. Использование респираторов как основного средства борьбы с пылью допускается в исключительных случаях с разрешения органов госгортехнадзора, госсанинспекции при невозможности применения других средств, обеспечивающих снижение запыленности воздуха до предельно допустимых концентраций.

По конструктивному исполнению различают две группы респираторов: респираторы многоразового использования со сменными фильтрами и респираторы кратковременного (одно-, двукратного) пользования, в которых фильтрующим элементом является сама маска. К первой группе относятся: ``Астра-2'', Ф62Ш, ко второй клапан-ный У-2Ки бесклапанный респиратор ШБ-1, ``Лепесток-200'', ``Лепесток-40'' и ``Лепесток-5'' (цифры обозначают область применения при концентрации запыленности, превышающей предельно допустимую в 200, 40 и 5 раз соответственно при размере частиц пыли до 1 мкм).

Общий вид респираторов показан на рисунке \ref{fig}.

\begin{figure}[h]
    \centering
    \includegraphics[]{1.jpg}
    \caption{Противопылевые респираторы: а--Астра-2; б--Ф62Ш;\\ в--УК-2М; г--ШБ-1}
    \label{fig}
\end{figure}


Респиратор ``Астра-2'' состоит из резиновой полумаски, снабженной клапаном выдоха, и двух полиэтиленовых коробок с клапанами вдоха. В коробки заложен гофрированный фильтр из материала фГТП-15. Его можно применять при температуре до -- 25° С.

Респиратор Ф62Ш состоит из резиновой полумаски с закрепленной на ней пластмассовой коробкой, в которой помещается сменный противопылевой фильтр. Коробка соединена с полумаской клапаном вдоха, в нижней части которой находится клапан выдоха.

Респиратор РП-КМ состоит из резиновой полумаски, снабженной клапанами вдоха и выдоха. По внешнему периметру маска имеет эластичную манжету, под которую вставляются и пристегиваются две фильтрующие оболочки внутренняя из материала Ф1111-15 и наружная из поролона. Благодаря клапану выдоха фильтрующие оболочки не увлажняются, меньше забиваются пылью и не затрудняют дыхание. Внутреннюю оболочку респиратора можно заменить в течение 1 мин, внешнюю промывают в воде и высушивают.

Для работающих в выработке при повышенной запыленности воздуха с большой физической нагрузкой рекомендованы респираторы Астра-2М, Ф62Ш, РП-КМ, а при выполнении легкой и средней тяжести работ — респираторы ТТТБ-1, "Леписток-5", УК-2М. В настоящее время освоены новые респираторы ПРШ 741 и РПМ-73 повышенной пылеёмкости. Эффективность пылеулавлавливания составляет 99,9\%.


\section{Ответы на контрольные вопросы}
\begin{enumerate}
    \item Какие средства защиты органов дыхания вы знаете?
    \item [Ответ:] Противогазы, респираторы, простейшие средства защиты (противопыльные тканевые маски ПТМ-1, ватно-марлевые повязки).
    \item []
    \item Как различаются респираторы по конструктивному испытанию?
    \item [Ответ:] Респираторы многоразового использования со сменными фильтрами и респираторы кратковременного (одно-, двукратного) пользования, в которых фильтрующим элементом является сама маска.
    \item []
    \item Предохраняют ли респираторы от газов?
    \item [Ответ:] Нет.
    \item []
    \item По разрешению, каких органов надзора респираторы используются как основное средство защиты?
    \item [Ответ:] Госгортехнадзор, госсанинспекция.
    \item []
    \item Какова эффективность пылеулавливания у самых современных респираторов?
    \item [Ответ:] Эффективность пылеулавлавливания составляет 99,9\%.
\end{enumerate}


\subsection*{Вывод}
В ходе выполнения данной практической работы я самостоятельно ознакомился со средствами индивидуальной и групповой защиты и ответил на контрольные вопросы.
\end{document}