\title{Расследование, оформление и учет несчастных случаев на производстве}
\author{Краснов А.С.\\МР--19}
\date{19 января 2023 г.}
\maketitlepage

\section{Введение}
Проблема производственного травматизма во всем мире и в России стоит очень остро. В нашей стране ежегодно в результате несчастных случаев на производстве гибнут тысячи людей, сотни тысяч получают производственные травмы. Поэтому необходимо проводить соответствующую государственную политику в области охраны труда, так как бездействие влечет за собой огромные человеческие, а также экономические потери.


Каждый несчастный случай на производстве подлежит расследованию, учету и документальной фиксации. Порядок расследования несчастных случаев на производстве регламентирован законодательно и требует от администрации компании соблюдения всех предписанных процедур в течение определенных временных рамок.


\section{Ведение расследования}
Администрация фирмы, на территории которой произошел несчастный случай, должна скрупулезно подойти к документальному сопровождению процедуры расследования. Оформлением всех бумаг в такой ситуации занимается назначенная руководителем фирмы комиссия.


Составленные комиссией и полученные в ходе расследования от экспертов, медиков, пострадавшего и иных причастных лиц бумаги объединяются в единое дело.

Данный комплект документов должен раскрывать все нюансы и обстоятельства несчастного случая на производстве, а также содержать ответы на ключевые вопросы:
\begin{itemize}
    \item о характере трудовых взаимоотношений пострадавшего и работодателя
    \item причинах наступления несчастного случая
    \item взаимосвязи поступков работника с исполнением им своих трудовых функций в момент повреждения его здоровья и (или) работоспособности
    \item степени вины работников фирмы, ответственных за недопущение производственного травматизма
\end{itemize}

Перечень основных документов для оформления несчастного случая на производстве установлен в ст. 229.2 ТК РФ.

\subsection{Основные документы}
Чтобы не нарушить законодательно установленную процедуру по документальному сопровождению расследования несчастного случая на производстве, в комплект документов необходимо включить бумаги трех видов:
\begin{itemize}
    \item уведомительные (сообщение о несчастном случае, принятых мерах и др.)
    \item организационные (приказ о создании комиссии, изменении сроков расследования и т. д.)
    \item описательно-фиксирующие (акты, заключения, экспертизы)
\end{itemize}


Каждый несчастный случай на производстве специфичен и индивидуален, поэтому комплект документов может быть разным по объему и включать как обязательные, так и дополнительные (носящие вспомогательный характер) бумаги.

Существует основополагающий локальный норматив на предприятии, составляемый в целях обеспечения безопасности работы людей -- положение об охране труда.

Первая группа документов, о которых идет речь, носит информативно-уведомительный характер. Избежать их оформления не получится -- такая необходимость закреплена законодательно.

Организационные документы -- это внутрифирменные распоряжения, требующиеся для организации процесса расследования производственного несчастного случая.

Третья группа документов наиболее разнообразна по видам и значительно превышает первые две по объему. Центральное место в данной группе занимает акт о несчастном случае на производстве (форма Н-1, используется на основании п. 26 Положения по постановлению Минтруда России от 24.10.2002 № 73, письма Минтруда России от 27.10.2017 № 15-3/В-2862).


Акт по форме Н-1 составляется в том случае, если работник утратил работоспособность, переведен на другую работу по медпоказаниям или погиб в результате несчастного случая на производстве (ст. 230 ТК РФ).

\subsection{Дополнительные документы}
Помимо акта по форме Н-1, комиссии по расследованию НС потребуется оформить протокол осмотра места происшествия, приложив к нему раскрывающие обстоятельства несчастного случая фото и видеоматериалы, всевозможные планы и схемы.


В материалах расследования, помимо объяснений пострадавших, должны содержаться результаты опросов очевидцев несчастного случая на производстве, итоги проведенных экспертиз и заключения медиков.


Дополнительно придется собрать и внутренние доказательства того, что работодатель предпринимал все предусмотренные законодательством меры по соблюдению работником требований охраны труда (о проведенных с пострадавшим инструктажах, результатах проверки его знаний по вопросам охраны труда, выданных ему средствах индивидуальной защиты и др.).


Также стоит отметить, что ознакомление работника с порядком обеспечения охраны труда на предприятии начинается с первичного инструктажа на рабочем месте.


Указанный список источников в рамках расследования не является закрытым и может быть дополнен, чтобы картина происшествия была максимально раскрыта. Состав дополнительных документов определяется комиссией по расследованию НС.

\subsection{Сроки оформления документов}
Сроки оформления документов напрямую связаны со сроками расследования несчастного случая на производстве. Несчастный случай на производстве расследуется в сроки, установленные ст. 229.1 ТК РФ и п. 19 Положения о расследовании несчастных случаев, утвержденного постановлением Минтруда России от 24.10.2002 № 73.


Предусматриваются два вида сроков: основные и дополнительные. Основной срок зависит от степени нанесенного в результате несчастного случая на производстве вреда здоровью и работоспособности работника. Он составляет:
\begin{itemize}
    \item 3 календарных дня (при незначительном вреде здоровью)
    \item 15 календарных дней (при тяжелой степени повреждений или смерти)
    \item 30 календарных дней (если расследование проводится по заявлению пострадавшего (или его доверенного лица) спустя какое-то время после несчастного случая)
\end{itemize}


Отсчет длительности промежутка, в течение которого расследуется несчастный случай на производстве (в том числе документально оформляется), ведется в календарных днях от даты составления приказа о создании комиссии по расследованию (п. 19 Положения № 73).


Дополнительные сроки предусмотрены для ситуаций, когда расследование не укладывается в регламентированное время по объективным причинам. Такое возможно, если требуется уточнение обстоятельств НС или необходимо получить экспертные заключения. Председателю комиссии в этом случае законодательно позволено продлить срок на календарных 15 дней. Если к расследованию подключены органы дознания, следствия или суд, продление срока согласовывается с указанными инстанциями.


Также председатель комиссии обязан сообщить об изменении сроков пострадавшему или его доверенному лицу (п. 20 Положения № 73).


Продление сроков придется оформить приказом, в котором указать причины и дату окончания расследования.


Такое пристальное внимание к соблюдению сроков поможет работодателю избежать наказания по ст.  5.27 КоАП РФ.

\section{Практическая часть}
\subsection{Описание происшествия}
3 апреля 2003 года в 9 часов 40 минут в Томском государственном промышленно--гуманитарном колледже, работник административно-хозяйственного отдела, слесарь-ремонтник 8еле Наил,родившийся 12 ноября 1936 года, при переходе из одного корпуса в другой поскользнулся на асфальтированной дорожке, покрытой льдом и припорошенной снегом. В результате падения сломал наружную лодыжку и правой голени. Стаж работы: 36 лети 6 месяцев, в данной организации проработал 15 лет и 7 месяцев.


\subsection{Сведения о проведенных инструктажах и стажировке}
Сведения о проведении вводного инструктажа не сохранились. Повторный инструктаж был проведен 12 сентября 2002 года.
Сведения о стажировке отсутствуют.

\subsection{Краткая характеристика места происшествия}
Покрытие дороги -- ровное, асфальт; освещенность удовлетворительная легкий снег, гололедица.

\subsection{Очевидцы}
Очевидцы несчастного случая: Морозов С.Ю., проживающий по адресу: Г. Томск, ул. Мичурина, 6а к. 311. Домашний телефон: 72-63-95.

\subsection{Расследование несчастного случая}
\subsubsection{Причины несчастного случая}
Неудовлетворительное состояние территории (асфальтированная дорожка покрыта льдом, покрытым свежевыпавшим снегом).
\subsubsection{Лица, допустившие нарушение требований охраны труда}
\begin{itemize}
    \item Уборщик территории
    \item Директора ТГПГК по АХР, нарушен пункт 2 подпункт 2.2 Должностной инструкции: «Обеспечивать создание комфортных производственных условий во всех помещениях и на территории учебного заведения». Не обеспечил контроль за качеством работы уборщика.
\end{itemize}
\subsubsection{Мероприятия по устранению причин несчастного случая, сроки}
\begin{enumerate}
    \item Организовать уборку территории
    \item Провести инструктаж с уборщиком
    \item Провести внеплановый инструктаж с работниками колледжа
\end{enumerate}
Срок: 6 апреля 2003 г. Ответственный: Морозов С.Ю.



\section{Заключение}
Документальное оформление несчастного случая на производстве сопровождает все этапы его расследования.


Потребуется сформировать полный комплект документов: составить приказ о создании комиссии по расследованию, зафиксировать все обстоятельства НС, собрать результаты опроса свидетелей и экспертиз, а также письменно информировать госорганы и заинтересованных лиц.

\begin{thebibliography}{4}
    \bibitem{1}
    Конституция Российской Федерации от 12.12.1993 г. № б/н // Российская газета. – 1993. – 25 декабря.
    \bibitem{2}
    Трудовой кодекс Российской Федерации от 30 декабря 2001 г. N 197-ФЗ // Российская газета. – 2001. – 31 декабря.
    \bibitem{3}
    Федеральный закон от 24 июля 1998 г. N 125-ФЗ "Об обязательном социальном страховании от несчастных случаев на производстве и профессиональных заболеваний" // Российская газета. – 1998. – 12 августа.
    \bibitem{4}
    Постановление Минтруда РФ от 24.10.2002 N 73 "Об утверждении форм документов, необходимых для расследования и учета несчастных случаев на производстве, и Положения об особенностях расследования несчастных случаев на производстве в отдельных отраслях и организациях" // Российская газета. – 2002. – 18 декабря
%    \bibitem{5}
%    Об определении степени тяжести повреждения здоровья при несчастных случаях на производстве: Приказ Минздравсоцразвития РФ от 24 февраля 2005 г. N 160 // БНА. - 2005. - N 16.
%    \bibitem{6}
%    Крапивин О.М., Власов В.И. Комментарий к законодательству об охране труда. - Система ГАРАНТ, 2007 г.
%    \bibitem{7}
%    Рыженков А.Я., Мелихов В.М., Шаронов С.А. Трудовое право России. Курс лекций. - Система ГАРАНТ, 2007 г.
\end{thebibliography}
