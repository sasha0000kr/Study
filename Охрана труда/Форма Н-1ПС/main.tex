% Расследование, оформление и учет несчастных случаев на производстве
% Краснов Александр

\documentclass[a4paper, 12pt, draft]{article}
\usepackage[utf8]{inputenc}%Кодировка
\usepackage[left=30mm, right=15mm,
           top=20mm, bottom=20mm]{geometry}%Параметры страницы, поля, отступы, переплет
\usepackage{cmap}%Ссылки в PDF
\usepackage[T2A]{fontenc}%Стандартные шрифты
\usepackage[english, russian]{babel}%Грамматика
\usepackage{microtype}%Микротипографические эффекты
%\usepackage{extsizes}%Добавляет поддержу дополнительных размеров текста 8pt, 9pt, 10pt, 11pt, 12pt, 14pt, 17pt, and 20pt
%\usepackage{graphicx,svg}\DeclareGraphicsExtensions{.pdf,.png,.jpg}%Поддержка изображений
\usepackage[nodisplayskipstretch]{setspace}\setstretch{1.3}%Настройка межстрочного интервала
%\usepackage{indentfirst}\setlength{\parindent}{1.25cm}%Красная строка
%\frenchspacing%Отключает большой пробел между предложениями
\usepackage{enumitem}\setlist{noitemsep}%Убираем расстояние между элементами списка
%\usepackage{parskip}\setlength{\parindent}{1.5pt}%Интервал между абзацами
\usepackage{pscyr}%Собственные шрифты, может не работать
%\renewcommand{\rmdefault}{fjn}
%\renewcommand{\ttdefault}{flz}
\renewcommand{\bfseries}{\relax}%Отключение жирного шрифта
\clubpenalty=5000\widowpenalty=5000%Висячие строки
\sloppy%Принудительное включение автоматического переноса

\newcommand{\code}[1]{\fbox{\ttfamily Код #1.}}
\newcommand{\key}[1]{\underline{#1}\hspace{1cm}}

\begin{document}
    \newcommand{\director}{ФИО}
\newcommand{\actnumber}{1}
\newcommand{\actdate}{\today}
\newcommand{\maincode}{3.01}%Код несчастного случая

\newcommand{\getdate}{11.11.2011}
\newcommand{\gettime}{13:00}
\newcommand{\timecode}{3.02}
\newcommand{\worktime}{5}
\newcommand{\worktimecode}{3.1}
        \begin{titlepage}
            \begin{center}
                {\large УТВЕРЖДАЮ\\[\bigskipamount]}
                \director\\
                \hrulefill\\
                {\small подпись, фамилия, инициалы работодателя (его представителя)\\}
                \actdate\\[3cm]
                {\Large АКТ №\actnumber}\\
                о несчастном случае на производстве\\
                \code{\maincode}\\
            \end{center}
            \tableofcontents
            \newpage
        \end{titlepage}
    {
    \renewcommand{\large}{\normalsize}
    \renewcommand{\Large}{\normalsize}
    \section{Дата несчастного случая: \key{\getdate}}
    \subsection{Время происшествия несчастного случая: \key{\gettime}\code{\timecode}}
    \subsection{Количество полных часов от начала работы: \key{\worktime}\code{\worktimecode}}
    
    
    }
\end{document}