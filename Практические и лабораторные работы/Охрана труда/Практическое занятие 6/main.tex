\documentclass[a5paper, 12dd, twoside]{article}
% !TeX spellcheck = ru_RU
%Настройка страниц

\usepackage[left=2.5cm, right=1.5cm, top=2.5cm, bottom=2.5cm]{geometry}
%twoside, openany,%

%Настройка языка и отображения
\usepackage{cmap}				%Ссылки в PDF
\usepackage[T2A]{fontenc}		%Шрифты
\usepackage[utf8]{inputenc}		%Кодировка
%\usepackage{lmodern}			%Хз, не трогать
\usepackage[russian]{babel}		%Грамматика
\usepackage{graphicx}			%Пикчи
\DeclareGraphicsExtensions{.pdf,.png,.jpg}
\frenchspacing					%Отключает большой пробел между предложениями

\usepackage{indentfirst}		%Красная строка
\setlength{\parindent}{1.25cm}	%Настройка отступа красной строки для шрифта 14pt, по умолчанию 15pt
\linespread{1.25} 				%Межстрочный интервал

\usepackage{parskip} 			%Интервал между абзацами
\setlength{\parindent}{1cm} 	%Настройка интервала между абзацами, по умолчанию будет 0

\usepackage{enumitem}
\setlist{noitemsep}


\pagestyle{plain} 				%Использование стандартных колонитулов

%\usepackage{fancyhdr}			%Колонтитулы
%\pagestyle{fancy} 				%Использование кастомных колонитулов
%\fancyhf{} 					%Отчистить все колонтитулы
%\lhead{} 						% левый верхний колонтитул
%\chead{} 						% центральный верхний
%\rhead{} 						% правый верхний
%\lfoot{} 						% левый нижний
%\cfoot{\thepage} 				% центральный нижний
%\rfoot{} 						% правый нижний


\usepackage{listings}
% настройка подсветки кода и окружения для листингов
%\usemintedstyle{colorful}
%\newenvironment{code}{\captionsetup{type=listing}}{}
\hyphenation{
    ко-леб-лю-ще-го-ся 
    им-пульс-но-го 
    ус-та-нав-ли-ва-ют 
    не-пос-то-ян-но-го 
    ха-рак-те-рис-ти-ки
    про-из-вод-ствен-но-го
    пос-ле-ду-ю-щей}

\title{Практическое занятие №6\\<<Оценка состояния производственной санитарии и гигиены на рабочем месте>>}
\author{Краснов Александр МР--19}

\begin{document}
\maketitle
\tableofcontents
\clearpage

\subsubsection*{Цель работы}
Научиться оценивать состояние производственной санитарии и гигиены на рабочем месте.
\section{Теоретические сведения}
\subsection*{Статистический метод}
{\bfseries Производственная санитария}~--- система организационных мероприятий и технических средств, предотвращающих или уменьшающих воздействие на работников вредных производственных факторов.

{\bfseries Гигиена труда}~--- область медицинской науки, которая освещает основные вопросы.

{\bfseries Производственные помещения}~--- замкнутые пространства в специально предназначенных зданиях и сооружениях, в которых постоянно (по сменам) или периодически (в течение рабочего дня) осуществляется трудовая деятельность людей.

{\bfseries Рабочее место}~--- участок помещения на котором в течение рабочей смены или части её осуществляется трудовая деятельность. Рабочим местом может являться несколько участков производственного помещения.

{\bfseries Холодный период года}~--- период года, характеризуемый среднесуточной температурой наружного воздуха равной +10оС и ниже.

{\bfseries Теплый период года}~--- период года, характеризуемый среднесуточной температурой наружного воздуха выше +10оС.

{\bfseries Среднесуточная температура наружного воздуха}~--- средняя величина температуры наружного воздуха, измеренная в определенные часы суток через одинаковые интервалы времени. Она принимается по данным метеорологической службы.

Разграничение работ по категориям осуществляется на основе интенсивности общих энерготрат организма в ккал/ч (Вт).

{\bfseries Тепловая нагрузка среды (ТНС)}~--- сочетанное действие на организм человека параметров микроклимата (температура, влажность, скорость движения воздуха, тепловое облучение), выраженное одночисловым показателем в \(^\circ C\).


\section{Ответы на контрольные вопросы}
\begin{enumerate}
    \item {\bfseries Какие критерии устанавливают санитарные правила для граждан России?}
    \item [Ответ:]
    \item []
    \item {\bfseries Какое деяние считается санитарным правонарушением?}
    \item [Ответ:]
    \item []
    \item {\bfseries Какие виды ответственности предусматриваются Законом о санитарно-эпидемиологическом благополучии РФ для лиц, допустивших санитарное правонарушение?}
    \item [Ответ:]
    \item []
    \item {\bfseries Что такое производственное помещение?}
    \item [Ответ:]
    \item []
    \item {\bfseries Что такое рабочее место?}
    \item [Ответ:]
    \item []
    \item {\bfseries Что такое холодный период года?}
    \item [Ответ:]
    \item []
    \item {\bfseries Что такое теплый период года?}
    \item [Ответ:]
    \item []
    \item {\bfseries Что такое среднесуточная температура наружного воздуха?}
    \item [Ответ:]
    \item []
    \item {\bfseries Какие категории работ выделяются по общим энерготратам организма?}
    \item [Ответ:]
    \item []
    \item {\bfseries Что такое тепловая нагрузка среды?}
    \item [Ответ:]
    \item []
    \item [Ответ:]
    \item []
    \item {\bfseries Что такое микроклимат в производственных помещениях?}
    \item [Ответ:]
    \item []
    \item {\bfseries Какие параметры составляют микроклимат рабочих помещений?}
    \item [Ответ:]
    \item []
    \item {\bfseries Каково главное требование к параметрам микроклимата в производственных помещениях?}
    \item [Ответ:]
    \item []
    \item {\bfseries Какие условия влияют на величину параметров микроклимата?}
    \item [Ответ:]
    \item []
    \item {\bfseries Какие виды микроклиматов (классификацию) различают?}
    \item [Ответ:]
    \item []
    \item {\bfseries Что такое температура воздуха?}
    \item [Ответ:]
    \item []
    \item {\bfseries Что такое влажность воздуха?}
    \item [Ответ:]
    \item []
    \item {\bfseries Что такое абсолютная влажность и в каких единицах она измеряется?}
    \item [Ответ:]
    \item []
    \item {\bfseries Что такое максимальная влажность и в каких единицах она измеряется?}
    \item [Ответ:]
    \item []
    \item {\bfseries Что такое относительная влажность и в каких единицах она измеряется?}
    \item [Ответ:]
    \item []
    \item {\bfseries Что такое движение воздуха в рабочих помещениях и почему оно возникает?}
    \item [Ответ:]
    \item []
    \item {\bfseries Что такое тепловое излучение и в каких единицах оно измеряется?}
    \item [Ответ:]
    \item []
    \item {\bfseries Как действуют на человека избыточные величины параметров микроклимата?}
    \item [Ответ:]
    \item []
    \item {\bfseries Что такое терморегуляция?}
    \item [Ответ:]
    \item []
    \item {\bfseries За счет каких механизмов осуществляется теплоотдача организмом?}
    \item [Ответ:]
    \item []
    \item {\bfseries По какому интегральному показателю оценивают тепловое состояние организма?}
    \item [Ответ:]
    \item []
    \item {\bfseries Какие осложнения возникают при нарушениях теплоотдачи организмом?}
    \item [Ответ:]
    \item []
    \item {\bfseries В чем заключается различие между тепловым и солнечным ударами?}
    \item [Ответ:]
    \item []
    \item {\bfseries В каких пределах могут находится величины параметров микроклимата?}
    \item [Ответ:]
    \item []
    \item {\bfseries Что такое оптимальная величина параметра микроклимата?}
    \item [Ответ:]
    \item []
    \item {\bfseries Какой может быть перепад температуры при обеспечении ее оптимального уровня?}
    \item [Ответ:]
    \item []
    \item {\bfseries Что такое допустимая величина параметра микроклимата?}
    \item [Ответ:]
    \item []
    \item {\bfseries При какой величине параметр микроклимата становится вредным или опасным?}
    \item [Ответ:]
    \item []
    \item {\bfseries Какой может быть перепад температуры при обеспечении ее допустимого уровня на рабочем месте?}
    \item [Ответ:]
    \item []
    \item {\bfseries Какова допустимая величина относительной влажности на рабочем месте?}
    \item [Ответ:]
    \item []
    \item {\bfseries Какова допустимая величина скорости движения воздуха на рабочем месте?}
    \item [Ответ:]
    \item []
    \item {\bfseries Какова допустимая интенсивность теплового излучения на рабочем месте?}
    \item [Ответ:]
    \item []
    \item {\bfseries Каковы главные требования к методам измерения и контроля параметров микроклимата?}
    \item [Ответ:]
    \item []
    \item {\bfseries Какими приборами измеряются параметры микроклимата на рабочем месте?}
    \item [Ответ:]
    \item []
    \item {\bfseries Каким образом оценивается истинная температура на рабочем месте?}
    \item [Ответ:]
    \item []
    \item {\bfseries Какой параметр микроклимата измеряется стационарным психрометром и как устроен этот прибор?}
    \item [Ответ:]
    \item []
    \item {\bfseries Каким образом повышается точность показаний стационарного психрометра?}
    \item [Ответ:]
    \item []
    \item {\bfseries По какой формуле определяется абсолютная влажность воздуха при использовании стационарного психрометра?}
    \item [Ответ:]
    \item []
    \item {\bfseries По какой формуле определяется относительная влажность воздуха?}
    \item [Ответ:]
    \item []
    \item {\bfseries По какой формуле определяется относительная влажность при использовании аспирационного психрометра?}
    \item [Ответ:]
    \item []
    \item {\bfseries Что такое производственная санитария?}
    \item [Ответ:]
    \item []
    \item {\bfseries Что такое гигиена труда?}
    \item [Ответ:]
    \item []
    \item {\bfseries Задачи производственной санитария?}
    \item [Ответ:]
    \item []
    \item [Ответ:]
    \item []
    \item {\bfseries Задачи гигиены труда?}
    \item [Ответ:]
    \item []
    \item {\bfseries Классификация вредных веществ по признакам?}
    \item [Ответ:]
    \item []
\end{enumerate}


\subsection*{Вывод}
В ходе выполнения данной практической работы я ознакомился с краткими теоретическими сведениями о методах оценки состояния техники безопасности на производственном объекте, а также выполнил расчет и выполнение анализа причин и уровня травматизма статистическим методом.
\end{document}