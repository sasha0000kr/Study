\documentclass[a5paper, 12pt, twoside]{article}
% !TeX spellcheck = ru_RU
%Автор - Краснов Александр 2022

%Настройка языка и отображения
\usepackage[russian]{babel}
%Шрифты
\usepackage[T2A]{fontenc}		%Русские шрифты
%\usefont{T2A}{Tempora-TLF}{m}{n}
%\usepackage{lmodern}			%Шрифт Latin Modern, нужен если нет cm-super

\usepackage[utf8]{inputenc}		%Кодировка
\usepackage{amssymb,amsmath}	%Математические символы
\usepackage{multicol}			%Несколько колонок
%\usepackage{color}				%Использовать цветной текст

\usepackage{graphicx}			%Пикчи
\DeclareGraphicsExtensions{.pdf,.png,.jpg}
\frenchspacing					%Отключает большой пробел между предложениями

\usepackage{indentfirst}		%Красная строка
\setlength{\parindent}{1.25cm}	%Настройка отступа красной строки для шрифта 14pt, по умолчанию 15pt
\linespread{1.25} 				%Межстрочный интервал

\usepackage{parskip} 			%Интервал между абзацами
\setlength{\parindent}{1cm} 	%Настройка интервала между абзацами, по умолчанию будет 0

\usepackage{enumitem}			%Настройка списков
\setlist{noitemsep}				%Убирает лишнюю строку в списках

\usepackage{textcomp}			%Улучшает знак номера

\usepackage{cmap}				%Ссылки в PDF
\usepackage[					%Гипертекстовое оглавление в PDF
bookmarks=true, colorlinks=true, unicode=true,
urlcolor=black,linkcolor=black, anchorcolor=black,
citecolor=black, menucolor=black, filecolor=black,
]{hyperref}

%\sloppy						%Автоматическое разряжение строк (только для черновиков)
\emergencystretch=20pt			%Аварийное разряжение строк (подбор опытным путем)
%\hfuzz=0.5pt					%Разрешить переполнение абзаца на 0.5dd (выглядит приемлемо)
\tolerance=300					%Настройка максимальной разряженности строки
\hyphenpenalty=100				%Настройка частоты переносов

\clubpenalty=5000				%Настройка висячих строк в начале абзаца от 0 до 10000
\widowpenalty=5000				%настройка висячих строк в конце абзаца от 0 до 10000


%Колонтитулы и номера страниц
%\pagestyle{empty} 				%Нет ни колонтитулов, ни номеров страниц
\pagestyle{plain} 				%Номера страниц ставятся внизу в середине строки, колонтитулов нет
%\pagestyle{headings} 			%Присутствуют колонтитулы (включающие в себя и номера страниц)
%\pagestyle{myheadings}			%То же, что и headings, но делается вручную

\pagenumbering{arabic}			%Нумерация страниц арабскими цифрами
%\pagenumbering{roman}			%Нумерация страниц римскими строчными цифрами
%\pagenumbering{Roman}			%Нумерация страниц римскими заглавными цифрами
%\pagenumbering{alph}			%Нумерация страниц строчными английскими буквами
%\pagenumbering{Alph}			%Нумерация страниц заглавными английскими буквами
%\pagenumbering{asbuk}			%Нумерация страниц строчными русскими буквами
%\pagenumbering{Asbuk}			%Нумерация страниц заглавными русскими буквами


%\usepackage{fancyhdr}			%Колонтитулы
%\pagestyle{fancy} 				%Использование кастомных колонтитулов
%\fancyhf{} 					%Отчистить все колонтитулы
%\lhead{} 						% левый верхний колонтитул
%\chead{} 						% центральный верхний
%\rhead{} 						% правый верхний
%\lfoot{} 						% левый нижний
%\cfoot{\thepage} 				% центральный нижний
%\rfoot{} 						% правый нижний


\usepackage{listings}
% настройка подсветки кода и окружения для листингов
%\usemintedstyle{colorful}
%\newenvironment{code}{\captionsetup{type=listing}}{}


%Спецсимволы
\usepackage{wasysym}			%Специальные символы, в том числе и гачи
\newcommand{\gachi}{\male}		%Гачи значок


%реализовать модификаторы шрифтов горячими клавишами



%Настройки верстки
%\openany						%Глава может начинаться с любой страницы
%\openright						%Глава только с правой страницы

%\fleqn							%Формулы слева
%\leqno							%Номера формул слева

%\raggedbottom					%Страницы разной высоты
\flushbottom					%Страницы одинаковой высоты

%\columnseprule=0.4pt			%Ширина линейки при верстке в колонки
%\columnsep=0mm					%Расстояние между колонками при верстке в колонки


%Настройка страниц

\usepackage[left=1.5cm, right=1.5cm, top=1.5cm, bottom=1.5cm]{geometry}
%twoside, openany,%

\title{Практическое занятие №1\\<<Выполнение расчета уровня шума на рабочем месте>>}
\author{Краснов Александр МР--19}


\begin{document}
\maketitle
\tableofcontents

\subsubsection*{Цель работы}
Овладение практическими навыками измерений шума с пос\-ле\-ду\-ю\-щей оценкой условий труда на рабочем месте.

\subsubsection*{Задачи}
Самостоятельно изучить основные физические ха\-рак\-те\-рис\-ти\-ки звука, классификацию производственного шума, его вредное действие на организм человека, нормирование; получить практические навыки измерений приборами уровней шума от различных источников; произвести расчѐты эквивалентного уровня звука на рабочем месте; сравнить эффективность различных методов защиты от про\-из\-вод\-ствен\-но\-го шума.

\section{Шум}
{\itshape Шум} --- это звук, оцениваемый негативно и наносящий\- вред здоровью. 
В качестве звука человек воспринимает упругие колебания, распространяющиеся в среде, которая может быть твердой, жидкой или газообразной. В зависимости от источника генерирующего колебания, различают шумы механического, аэродинамического и электромагнитного происхождения. 

\subsection{Механический шум}
На ряде производств преобладает механический шум, основными источниками которого являются зубчатые передачи, механизмы ударного типа, цепные передачи, подшипники качения и т.п. Он вызывается силовыми воздействиями неуравновешенных вращающихся масс, ударами в сочленениях деталей, стуками в зазорах, движением материалов в трубопроводах и т.п. Спектр механического шума занимает широкую область частот. Определяющими факторами механического шума являются форма, размеры и тип конструкции, число оборотов, механические свойства материала, состояние поверхностей взаимодействующих тел и их смазывание. Машины ударного действия, к которым относится, например, кузнечно-прессовое оборудование, являются источником импульсного шума, причем его уровень на рабочих местах, как правило, превышает допустимый. На машиностроительных предприятиях наибольший уровень шума создается при работе металло -- и деревообрабатывающих станков. 

\subsection{Аэродинамические и гидродинамические шумы}
\begin{enumerate}
    \item шумы, обусловленные периодическим выбросом газа в атмосферу, работой винтовых насосов и компрессоров, пневматических двигателей, двигателей внутреннего сгорания
    \item шумы, возникающие из-за образования вихрей потока у твердых границ. Эти шумы наиболее характерны для вентиляторов, турбовоздуходувок, насосов, турбокомпрессоров, воздуховодов
    \item кавитационный шум, возникающий в жидкостях из-за потери жидкостью прочности на разрыв при уменьшении давления ниже определенного предела и возникновения полостей и пузырьков, заполненных парами жидкости и растворенными в ней газами.
\end{enumerate}

\section{Шумы электромагнитного происхождения}
Шумы электромагнитного происхождения возникают в различных электротехнических изделиях (например, при работе электрических машин). Их причиной является взаимодействие ферримагнитных масс под влиянием переменных во времени и пространстве магнитных полей. Электрические машины создают шумы с различными уровнями звука от \(\frac{20}{30}\) дБ (микромашины) до \(\frac{20}{30}\) дБ (крупные быстроходные машины).

\section{Классификация шумов, воздействующих на человек}
\subsection{По характеру спектра шум делится}
\begin{enumerate}
    \item на широкополосный шум, с непрерывным спектром шириной более 1 октавы
    \item на тональный шум, в спектре которого имеются выраженные тоны
\end{enumerate}
Тональный характер шума для практических целей устанавливается измерением в \(\frac{1}{3}\) октавных полосах частот по превышению уровня в одной полосе над соседними не менее чем на 10 дБ.
\subsection{По временным характеристикам шум делится}
\begin{enumerate}
    \item на постоянный шум, уровень звука которого за 8-часовой рабочий день или за время измерения в помещениях жилых и общественных зданий, на территории жилой застройки изменяется во времени не более чем на 5 дБА
    \item непостоянный шум, уровень которого за 8-часовой рабочий день, рабочую смену или во время измерения в помещениях жилых и общественных зданий, на территории жилой застройки изменяется во времени более чем на 5 дБА
\end{enumerate}
\subsection{Непостоянные шумы подразделяют}
\begin{enumerate}
    \item на колеблющийся во времени шум, уровень звука которого непрерывно изменяется во времени
    \item прерывистый шум, уровень звука которого ступенчато изменяется на 5дБА и более, причем длительность интервалов, в течение которых уровень остается постоянным, составляет 1 с и более
    \item импульсный шум, состоящий из одного или нескольких звуковых сигналов, каждый длительностью менее 1 с, при этом уровни звука в дБАI и дБА отличаются не менее чем на 7 дБ.
\end{enumerate}

\section{Методы измерения шума на рабочих местах (в соответствии с ГОСТ 12.1.050-86 «Методы измерения шума на рабочих местах»)}
\begin{enumerate}
    \item Микрофон следует располагать на высоте 1,5 м над уровнем пола или рабочей площадки (если работа выполняется стоя) или на высоте уха человека, подвергающегося воздействию шума (если работа выполняется сидя). Микрофон должен быть ориентирован в направлении максимального уровня шума и удален не менее чем на 0,5 м от оператора, проводящего измерения.
    \item Для оценки шума на постоянных рабочих местах измерения следует проводить в точках, соответствующих установленным постоянным местам.
    \item Для оценки шума на непостоянных рабочих местах измерения следует проводить в рабочей зоне в точке наиболее частого пребывания работающего.
    \item При проведении измерений октавных уровней звукового давления переключатель частотной характеристики прибора устанавливают в положение "фильтр". Октавные уровни звукового давления измеряют в полосах со среднегеометрическими частотами 63-8000 Гц. При проведении измерений уровней звука и эквивалентных уровней звука, дБА, переключатель частотной характеристики прибора устанавливают в положение ``А''.
    \item При проведении измерений уровней звука и октавных уровней звукового давления постоянного шума переключатель временной характеристики прибора устанавливают в положение ``медленно''. Значения уровней принимают по средним показателям при колебании стрелки прибора.
    \item Значения уровней звука и октавных уровней звукового давления считывают со шкалы прибора с точностью до 1 дБА, дБ.
    \item Измерения уровней звука и октавных уровней звукового давления постоянного шума должны быть проведены в каждой точке не менее трех раз.
    \item При проведении измерений эквивалентных уровней звука колеблющегося во времени шума для определения эквивалентного (по энергии) уровня звука переключатель временной характеристики прибора устанавливают в положение ``медленно''. Значения уровней звука принимают по показаниям стрелки прибора в момент отсчета.
    \item При проведении измерений максимальных уровней звука колеблющегося во времени шума переключатель временной характеристики прибора устанавливают в положение ``медленно''. Значения уровней звука снимают в момент максимального показания прибора.
    \item При проведении измерений максимальных уровней звука импульсного шума переключатель временной характеристики прибора устанавливают в положение ``импульс''. Значения уровней принимают по максимальному показанию прибора.
    \item Интервалы отсчета уровней звука колеблющегося во времени шума при измерениях эквивалентного уровня продолжительностью 30 мин составляют 5–6 с при общем числе отсчетов 360.
    \item При проведении измерений эквивалентных уровней звука непостоянного шума переключатель временной характеристики прибора устанавливают в положение ``медленно'', измеряют уровни звука и продолжительность каждой ступени.
\end{enumerate}
\end{document}