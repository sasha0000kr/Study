\documentclass[a5paper, 12dd, twoside]{article}
% !TeX spellcheck = ru_RU
%Настройка страниц

\usepackage[left=2.5cm, right=1.5cm, top=2.5cm, bottom=2.5cm]{geometry}
%twoside, openany,%

%Настройка языка и отображения
\usepackage{cmap}				%Ссылки в PDF
\usepackage[T2A]{fontenc}		%Шрифты
\usepackage[utf8]{inputenc}		%Кодировка
%\usepackage{lmodern}			%Хз, не трогать
\usepackage[russian]{babel}		%Грамматика
\usepackage{graphicx}			%Пикчи
\DeclareGraphicsExtensions{.pdf,.png,.jpg}
\frenchspacing					%Отключает большой пробел между предложениями

\usepackage{indentfirst}		%Красная строка
\setlength{\parindent}{1.25cm}	%Настройка отступа красной строки для шрифта 14pt, по умолчанию 15pt
\linespread{1.25} 				%Межстрочный интервал

\usepackage{parskip} 			%Интервал между абзацами
\setlength{\parindent}{1cm} 	%Настройка интервала между абзацами, по умолчанию будет 0

\usepackage{enumitem}
\setlist{noitemsep}


\pagestyle{plain} 				%Использование стандартных колонитулов

%\usepackage{fancyhdr}			%Колонтитулы
%\pagestyle{fancy} 				%Использование кастомных колонитулов
%\fancyhf{} 					%Отчистить все колонтитулы
%\lhead{} 						% левый верхний колонтитул
%\chead{} 						% центральный верхний
%\rhead{} 						% правый верхний
%\lfoot{} 						% левый нижний
%\cfoot{\thepage} 				% центральный нижний
%\rfoot{} 						% правый нижний


\usepackage{listings}
% настройка подсветки кода и окружения для листингов
%\usemintedstyle{colorful}
%\newenvironment{code}{\captionsetup{type=listing}}{}
\hyphenation{
    ко-леб-лю-ще-го-ся 
    им-пульс-но-го 
    ус-та-нав-ли-ва-ют 
    не-пос-то-ян-но-го 
    ха-рак-те-рис-ти-ки
    про-из-вод-ствен-но-го
    пос-ле-ду-ю-щей}

\title{Практическое занятие №5\\<<Оценка состояния техники безопасности на производственном объекте>>}
\author{Краснов Александр МР--19}

\begin{document}
\maketitle
\tableofcontents
\clearpage

\subsubsection*{Цель работы}
Научиться оценивать состояние техники безопасности на производстве по результатам расследования несчастного случая. Краткие теоретические сведения Учет несчастных случаев на производстве позволяет изучить причины и обстоятельства возникновения несчастных случаев, и на их основе разработать и выполнить мероприятия по предупреждению травматизма и профессиональных заболеваний.

\section{Теоретические сведения}
\subsection*{Статистический метод}
Статистический метод изучает повторяемость и позволяет провести сравнительную оценку несчастных случаев, используя относительные показатели – коэффициенты частоты, тяжести и потерь производства. 

Коэффициент частоты травматизма показывает число несчастных случаев, приходящихся на 1000 работающих за определенный промежуток времени и рассчитывается по формуле

\[K = \frac{1000 N}{R}\]
N -- число учтенных несчастных случаев за анализируемый период, R -- среднесписочное число работающих за этот же период.

Коэффициент тяжести травматизма характеризует среднюю потерю трудоспособности на одного пострадавшего за анализируемый период и рассчитывается по формуле

\[KT = \frac{D}{N}\]
где D –общее число дней нетрудоспособности (кроме несчастных случаев с летальным исходом).

Коэффициент потерь производства представляет среднюю потерю трудоспособности на 1000 работающих и выражается произведением коэффициентов частоты и тяжести:

\[KP = K \times KT = \frac{1000 D}{R}\]


\section{Практическая часть}
\subsubsection*{Входные данные}
\(N = 10\) -- число учтенных несчастных случаев за анализируемый период

\(R = 4000\) -- среднесписочное число работающих за этот же период

\(D = 7\) -- общее число дней нетрудоспособности (кроме несчастных случаев с летальным исходом)

\subsubsection*{Коэффициент частоты травматизма}
\[K = \frac{1000N}{R} = \frac{1000 \times 10}{4000} = 2.5\]
\subsubsection*{Коэффициент тяжести травматизма}
\[KT = \frac{D}{N} = \frac{7}{10} = 0.7\]
\subsubsection*{Коэффициент потерь производства}
\[KP = K \times KT = \frac{1000 D}{R} = \frac{1000 \times 7}{4000} = 1.75\]

%\section{Ответы на контрольные вопросы}
%\begin{enumerate}
%    \item {\bfseries Какие нарушения, приведшие к несчастным случаям, объединены в технические причины?}
%    \item []
%    \item {\bfseries Какие нарушения, приведшие к нечастным случаям, объединены в организационные причины?}
%    \item []
%    \item {\bfseries Какие нарушения, приведшие к несчастным случаям, объединены в санитарно-гигиенические мероприятия?}
%    \item []
%    \item {\bfseries По каким признакам несчастные случаи распределяются по группам, от чего это зависит?}
%    \item []
%    \item {\bfseries Что выявляет топографический метод при исследовании несчастных случаев?}
%    \item []
%    \item {\bfseries ри анализе причин травматизма монографическим методом, что выявляется дополнительно в процессе исследования?}
%    \item []
%    \item {\bfseries Что можно выявить при исследовании несчастных случаев, используя коэффициент частоты и коэффициент тяжести, при статистическом методе анализа несчастных случаев?}
%    \item []
%\end{enumerate}


\subsection*{Вывод}
В ходе выполнения данной практической работы я ознакомился с краткими теоретическими сведениями о методах оценки состояния техники безопасности на производственном объекте, а также выполнил расчет и выполнение анализа причин и уровня травматизма статистическим методом.
\end{document}