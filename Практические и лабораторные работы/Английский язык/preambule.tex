% !TeX spellcheck = ru_RU
%Автор - Краснов Александр 2022
\usepackage[utf8]{inputenc}%Кодировка
\usepackage[left=20mm, right=15mm,
           top=15mm, bottom=15mm]{geometry}%Параметры страницы, поля, отступы, переплет
\usepackage{cmap}%Ссылки в PDF
\usepackage[T2A]{fontenc}%Стандартные шрифты
\usepackage[russian, english]{babel}%Грамматика
\usepackage{microtype}%Микротипографические эффекты
%\usepackage{extsizes}%Добавляет поддержу дополнительных размеров текста 8pt, 9pt, 10pt, 11pt, 12pt, 14pt, 17pt, and 20pt
%\usepackage{graphicx,svg}\DeclareGraphicsExtensions{.pdf,.png,.jpg}%Поддержка изображений
\usepackage[nodisplayskipstretch]{setspace}\setstretch{1.3}%Настройка межстрочного интервала
\usepackage{indentfirst}\setlength{\parindent}{1.25cm}%Красная строка
%\frenchspacing%Отключает большой пробел между предложениями
\usepackage{enumitem}\setlist{noitemsep}%Убираем расстояние между элементами списка
%\usepackage{parskip}\setlength{\parindent}{1.5pt}%Интервал между абзацами
%\usepackage{pdfpages}%Вставка PDF как приложений
\usepackage{pscyr}%Собственные шрифты, может не работать
\renewcommand{\rmdefault}{ftm}
\renewcommand{\bfseries}{\relax}%Отключение жирного шрифта
\clubpenalty=5000\widowpenalty=5000%Висячие строки
\sloppy%Принудительное включение автоматического переноса

%Макросы
\newcommand{\B}[1]{\textbf{#1}}
\newcommand{\I}[1]{\textit{#1}}
\newcommand{\Y}[1]{\underline{#1}}
\newcommand{\F}[1]{\fbox{#1}}

%Нумерация заголовков буквами
\renewcommand{\thesection}{\Alph{section}}
\renewcommand{\thesubsection}{\Alph{section}.}

\newcommand{\maketitlepage}{
    %\begin{titlepage}
        \author{Краснов Александр МР--19}
        \maketitle
        \tableofcontents
        \newpage
    %\end{titlepage}
    }

%\selectlanguage{english}