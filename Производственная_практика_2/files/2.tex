\chapter{ТЕХНИКА БЕЗОПАСНОСТИ}
К работе с моделями промышленных механизмов допускаются только лица, ознакомленные с их устройством, принципом действия, программным обеспечением и мерами безопасности в соответствии с требованиями, изложенными в настоящем разделе.

Для подключения модулей ручного управления и программируемых логических модулей должны использоваться только кабели, входящие в комплект поставки.

При обнаружении повреждений изоляции соединительных проводов необходимо работу с моделями прекратить и отключить их от питающей сети. Повторное включение разрешается только после устранения повреждений изоляции проводов или их замены.

Техническое обслуживание и ремонтные работы производить только после полного отключения моделей от питающей сети переменного тока ~220В и при отсутствии давления сжатого воздуха в пневмосистеме.

%Никогда не работайте с установкой, если находитесь в состоянии алкогольного или наркотического опьянения. И не допускайте к работе других лиц, если они ведут себя не адекватно или пьяны.

Во время работы установка находится под высоким давлением и электрическим напряжением, что может являться потенциально опасным и причинить травмы.

\section{Меры безопасности при работе с пневматической системой}
Все манипуляции с пневматической системой производить только при отключенной подачи давления. Перед включением проверить исправность, правильность, надежность и герметичность всех соединений пневматической магистрали, чтобы исключить утечки воздуха. Если не работаете с установкой, отключите подачу давления.


Периодически проверяйте надежность соединений пневматической\\магистрали, так как при эксплуатации возможно ослабление креплений. Не пользуйтесь устройствами, в которых отсутствуют какие-либо части.

Эксплуатируйте установку согласно температурному режиму во избежании поломок пневматической системы, обеспечьте подогрев компрессора и используйте смазку в соответствии с температурным режимом при необходимости.

Всегда производите техническое обслуживайте, ремонт и монтаж установки согласно инструкции. Своевременная смазка, чистка и обслуживание установки увеличивает его ресурс и уменьшает вероятность поломки.

\section{Меры безопасности при работе с электрической системой}
Все манипуляции с электрической системой производить только при отключенном питании. Перед включением проверить правильность, надежность и полярность соединений, чтобы исключить короткое замыкание или выход из строя электрических приборов. Если не работаете с электрическими устройствами долгое время, отключите питание.

Не прикасайтесь к не изолированным контактам переменного тока ~220В и не производите их подключение под напряжением.

Не подключайте устройства с низковольтным питанием и логикой к сети переменного тока ~220В.